\startcomponent 02-04-01_Typeset_MainFlow

\environment introCTX_env_00


\startchapter
  [title=Composition des éléments du flux principal,
  reference=cap:typset:mainflow]

\TocChap

Pour rappel tous les éléments dont nous allons parler maintenant
ont vocation à être indiqués dans zone préambule du fichier source
(voir \in{section}[sec:srcpreambule]) et mieux encore dans un fichier
environnement (voir \in{section}[sec:srcenvfile]) qui sera lui-même
appelé en préambule. Il est biensûr possible de les indiquer localement
dans le corps de texte, pour un élément très spécifique, 
mais ce n'est pas une bonne pratique.

\component        02-04-01-01_Typeset_Paragraphs
\component        02-04-01-02_Typeset_WordEmphasis
\component        02-04-01-03_Typeset_ParagraphsEmphasis
\component        02-04-01-04_Typeset_Framed
\component        02-04-01-05_Typeset_LinesRules
\component        02-04-01-06_Typeset_Quotations
\component        02-04-01-07_Typeset_Itemize
\component        02-04-01-08_Typeset_DescriptionEnumeration
\component        02-04-01-09_Typeset_Tabulate
\component        02-04-01-10_Typeset_Tables
\component        02-04-01-11_Typeset_ImagesAndCombinaisons
\component        02-04-01-12_Typeset_Floats
\component        02-04-01-13_Typeset_Sections
\component        02-04-01-14_Typeset_Macrostructure 
\component        02-04-01-15_Typeset_Covers
\component        02-04-01-16_Typeset_Maths

\component        02-04-01-21_Typeset_Columns

% * ===========================================================================

\stopchapter

\stopcomponent

%%% Local Variables:
%%% mode: ConTeXt
%%% mode: auto-fill
%%% coding: utf-8-unix
%%% TeX-master: "../introCTX_fra.tex"
%%% End:
%%% vim:set filetype=context tw=75 : %%%

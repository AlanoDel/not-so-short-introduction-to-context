\startcomponent 02-04-02-02_Typeset_Notes

\environment introCTX_env_00

%==============================================================================

\startsection
  [title=Notes,
   reference=sec:confnotes]


\TocChap

% *** Subsection 
\startsubsection
  [title=Configurer les notes]

\placefigure [force,here,none] [] {}
{\startDemoI
\definenote[MesNotes][Modèle][Configuration]
\stopDemoI}   
   
\startitemize

\item {\em MesNotes} est le nom que nous attribuons à notre nouveau type de note.

\item {\em Modèle} est le modèle de note qui sera utilisé initialement, soit par exemple {\tt footnote} et {\tt endnote}. Dans le premier cas, notre modèle de note fonctionnera comme des notes de bas de page, et dans le second cas, comme des notes de fin de document. Nous les définirons dans le fichier source avec la commande \tex{MesNotes} et nous utliseront \tex{placenotes[MesNotes]} (le nom que nous avons attribué à ces types de notes) à l'endriot où nous souhaitons les afficher dans le document final.


\item {\em Configuration} est un argument facultatif qui nous permet de distinguer notre nouveau type de notes de son modèle~: soit en définissant un format différent, soit un type de numérotation différent, soit les deux.

  \startSmallPrint

Selon la liste officielle des commandes \ConTeXt\ (voir \in{section}[sec:qrc-setup-fr]), les paramètres qui peuvent être fournis lors de la création du nouveau type de note sont basés sur ceux qui pourraient être fournis plus tard avec \tex{setupnote}. Cependant, comme nous le verrons bientôt, il existe en fait deux commandes possibles pour configurer les notes : \tex{setupnote} et \cmd{setupnotation} (voir \in{section}[sec:confnotes]). Je pense donc qu'il est préférable d'omettre cet argument lors de la création du type de note, puis de configurer nos nouvelles notes à l'aide des commandes appropriées. Au moins, c'est plus facile à expliquer.

  \stopSmallPrint

\stopitemize

Par exemple, pour créer un nouveau type de note appelé \quotation{MesNotesBleues} qui sera similaire aux notes de bas de page mais dont le contenu sera imprimé en gras et en bleu, nous pouvons définir ainsi~: 

\placefigure [force,here,none] [] {}
{\startDemoI
\definenote    [MesNotes]
\setupnotation [MesNotesBleues]  [color=blue, style=bf]
\stopDemoI}

Une fois créée, nous disposonss de la commande \tex{MesNotesBleues} dont la syntaxe est similaire à \tex{footnote}.

%-----------------------------------------

\startbuffer[a7-testfootnoteD]
\setuppapersize[A7,landscape]
\setupbodyfont[8pt]
\definenote[MesNotesBleues]
\setupnotation 
  [MesNotesBleues] 
  [color=blue, style=bf]
\starttext
Coucou \MesNotesBleues[refA]{coucou ici}
\placenotes[MesNotesBleues]
\stoptext
\stopbuffer

\savebuffer[list=a7-testfootnoteD,file=ex_footnoteD.tex,prefix=no]
\placefigure [force,here,none] [] {}{\typesetbuffer[a7-testfootnoteD][frame=on,page=1,background=color,backgroundcolor=white]
\attachment
  [file={ex_footnoteD.tex},
   title={exemple testfootnoteD}]}




\PlaceMacro{setupnote}
\PlaceMacro{setupnotation}

La configuration des notes (notes de bas de page ou de fin de page, notes créées avec \tex{definenote} et aussi notes de ligne mises en place avec \tex{linenote}) est réalisée avec deux commandes : \tex{setupnote} et \tex{setupnotation}.

\startSmallPrint 
\tex{setupnote} possède 35 options de configuration {\em directes} et 45 options supplémentaires héritées de \tex{setupframed} ; \tex{setupnotation} possède 45 options de configuration  {\em directes} et 23 autres héritées de \tex{setupcounter} \tex{setupcounter}. Comme ces options ne sont pas documentées et, bien que pour beaucoup d'entre elles nous puissions deviner leur utilité à partir de leur nom, nous devons vérifier si notre intuition est vraie ou non ; et aussi en tenant compte du fait que beaucoup de ces options permettent un certain nombre de valeurs et qu'elles doivent toutes être testées... Vous verrez que pour écrire cette explication j'ai dû faire un certain nombre de tests ; et bien que faire un test soit rapide, faire beaucoup de tests est lent et ennuyeux. J'espère donc que le lecteur m'excusera si je lui dis qu'en dehors des deux commandes de configuration générale des notes que je mentionne dans le texte principal et sur lesquelles je me concentre dans l'explication suivante, je laisserai de côté quatre autres possibilités de configuration dans l'explication :

\startitemize

\item \PlaceMacro{setupnotes} \tex{setupnotes} et \PlaceMacro{setupnotations}\tex{setupnotations} : En d'autres termes, le même nom mais au pluriel. Le wiki dit que les versions singulière et plurielle de la commande sont synonymes, et je le crois.

  \item \PlaceMacro{setupfootnotes} \tex{setupfootnotes} et \PlaceMacro{setupendnotes} \tex{setupendnotes} : Nous supposons qu'il s'agit d'applications spécifiques pour, respectivement, les notes de bas de page et les notes de fin. Il serait peut-être plus facile d'expliquer la configuration des notes sur la base de ces commandes, cependant, puisque je n'ai pas réussi à faire fonctionner la première option ({\tt numberconversion}) que j'ai essayée avec \tex{setupfootnotes}, bien que je sache que les autres options de ces commandes fonctionnent... J'étais trop paresseux pour ajouter les tests nécessaires pour inclure ces deux commandes dans l'explication aux nombreux tests que j'ai déjà dû faire pour écrire ce qui suit.\blank[small]

Mais je suis d'avis (d'après les quelques tests aléatoires que j'ai effectués) que tout ce qui fonctionne dans ces deux commandes, mais dont je laisse l'explication de côté, fonctionne également dans les commandes pour lesquelles je donne une explication.

\stopitemize

\stopSmallPrint 

La syntaxe est similaire dans les deux cas :

\placefigure [force,here,none] [] {}
{\startDemoI
\setupnote[NoteType][Configuration]
\setupnotation[NoteType][Configuration]
\stopDemoI}

% TODO Garulfo : il va falloir nettoyer pour éclaircir le propos

où {\em NoteType} fait référence au type de note que nous configurons ({\tt footnote}, {\tt endnote} ou le nom d'un type de note que nous avons nous-mêmes créé), et {\em configuration} contient les options de configuration particulières de la commande.

Le problème est que les noms de ces deux commandes n'indiquent pas clairement la différence entre elles ni ce que chacune d'elles configure ; et le fait que de nombreuses options de ces commandes ne soient pas documentées n'aide pas beaucoup non plus. Après de nombreux tests, je n'ai pas été en mesure de parvenir à une conclusion qui me permettrait de comprendre pourquoi certaines choses sont configurées avec l'une, tandis que d'autres le sont avec l'autre,\footnote{Il existe une page dans le \goto{ \ConTeXt\ wiki} [url(https://wiki.contextgarden.net/Unexpected\_behavior)] que j'ai découverte par hasard (puisqu'elle n'est pas spécifiquement dédiée aux notes), qui suggère que la différence est que \tex{setupnotation} contrôle le texte de la note à insérer, et \tex{setupnote} l'environnement de la note dans laquelle elle sera placée ( ? ) Mais ceci n'est pas cohérent avec le fait que, par exemple, la largeur du texte de la note (qui a à voir avec son {\em insertion}) est contrôlée par l'option {\tt width} de \tex{setupnote} et non par l'option \tex{setupnotation} du même nom. Ce qui est contrôlé dans ce cas c'est la largeur de l'espace pris pour afficher la marque, avant l'affichage du texte de la note.} sauf peut-être que, en raison des choix que j'ai faits pour le faire fonctionner, \tex{setupnotation} affecte toujours le texte de la note, ou l'ID qui est imprimé avec le texte de la note, alors que \tex{setupnote} a quelques options qui affectent la marque de la note insérée dans le texte principal.

Je vais maintenant essayer d'organiser ce que j'ai découvert après avoir fait quelques tests avec les différentes options des deux commandes. Je laisse de côté la plupart des options des deux commandes, car elles ne sont pas documentées et je n'ai pas pu tirer de conclusions quant à leur utilité ou aux conditions dans lesquelles elles doivent être utilisées :


\startitemize

\head {\bf ID utilisé pour la note:} Les notes sont toujours identifiées par un numéro. Ce que nous pouvons configurer ici est :

  \startitemize

  \item {\em Le premier nombre} : contrôlé par {\tt start} dans \tex{setupnotation}. Sa valeur doit être un nombre entier, que \ConTeXt\ utilise comme point de départ pour comptabiliser les notes.

  \item {\em Le système de numérotation}, qui dépend de l'option {\tt numberconversion} dans \tex{setupnotation}. Ses valeurs peuvent être :

  \startitemize [packed]

  \item {\em chiffres arabes} : {\tt n, N} ou {\tt numbers}.

  \item {\em chiffres romains} : {\tt I, R, Romannumerals, i, r, romannumerals}. Les trois premiers sont des chiffres romains en majuscules et les trois derniers en minuscules.

  \item {\em Numérotation avec des lettres} : {\tt A, Character, Characters, a, character, characters} selon que l'on souhaite que les lettres soient en majuscules (les trois premières options) ou en minuscules (le reste).

  \item {\em Numérotation avec des mots}. En d'autres termes, on écrit le mot qui désigne le nombre et ainsi, par exemple, \quote{3} devient \quote{trois}. Deux méthodes sont possibles. L'option {\tt Words} écrit les mots en majuscules et {\tt words} en minuscules.

  \item {\em Numérotation avec symboles} : on peut utiliser quatre jeux de symboles différents selon l'option choisie : {\tt set~0, set~1, set~2} o {\tt set~3}. Sur \at{page} [exemples de jeux de conversion], vous trouverez un exemple des symboles utilisés dans chacune de ces options.

  \stopitemize

\item {\em L'événement qui détermine la remise à zéro de la numérotation des notes} : Cela dépend de l'option {\tt way} dans \tex{setupnotation}. Lorsque la valeur est {\tt bytext}, toutes les notes du document seront numérotées séquentiellement sans que la numérotation soit réinitialisée. Lorsqu'elle vaut {\tt bychapter, bysection, bysubsection, etc.}, le compteur de notes sera réinitialisé chaque fois qu'un nouveau chapitre, ou nouvelle section ou sous-section est créé, tandis que lorsqu'elle vaut {\tt byblock}, la numérotation sera réinitialisée chaque fois que nous changerons de bloc dans la macrostructure du document (voir \in{section}[sec:macrostructure]). La valeur {\tt bypage} fait redémarrer le compteur de notes à chaque fois que la page est changée.

  \stopitemize

\head {\bf  Configurer la marque de la note~:}

  \startitemize

  \item Affichage ou non : Option {\tt number} dans \tex{setupnotation}.

  \item Positionnement de la marque par rapport au texte de la note : L'option {\tt alternative} dans \tex{setupnotation} : elle peut prendre l'une des valeurs suivantes : {\tt left, inleft, leftmargin, right, inright, rightmargin, inmargin, margin, innermargin, outermargin, serried, hanging, top, command}. Par défaut la marque est dans l'espace entre la marge et le texte. Avec {\tt hanging}
la marque est intégré dans la zone de texte, de même avec {\tt serried} qui en plus intégre une identation lors des retours à la ligne du texte de la note .

  \item Style et couleur à appliquer à la marque dans le corps du texte : L'option {\tt textstyle} et {\tt textcolor} dans \tex{setupnote}.

  \item Style et couleur à appliquer à la marque dans la note elle-même : L'option {\tt headstyle} et {\tt headcolor} dans \tex{setupnotation}.
  
  \item Commande à appliquer à la marque dans le corps du texte : L'option {\tt textcommand} dans \tex{setupnote}.

  \item Commande à appliquer à la marque dans la note elle-même : L'option {\tt numbercommand} dans \tex{setupnotation}.
   
\startSmallPrint
      
Les options {\tt numbercommand} et {\tt textcommand} doivent consister en une commande qui prend le contenu de la marque comme argument. Il peut s'agir d'une commande auto-définie. Cependant, j'ai constaté que les commandes de formatage simples (\tex{bf}, \tex{it}, etc.) fonctionnent, bien qu'elles ne soient pas des commandes devant prendre un argument.

\stopSmallPrint

\item Distance entre la marque et le texte (dans la note elle-même) : l'option {\tt distance} dans \tex{setupnotation}. Elle est complétée par l'option {\tt width} qui indique la largeur à accorder à l'affichage de la marque elle-même.

\item Existence ou non d'un lien hypertexte permettant de sauter entre la marque dans le texte principal et la marque dans la note elle-même : L'option {\tt interaction} dans \tex{setupnote}. Avec {\tt yes} comme valeur, il y aura un lien, et avec {\tt no} il n'y en aura pas.

\stopitemize

\head {\bf Configuration du texte de la note elle-même.}
   Nous pouvons influencer les aspects suivants~:

  \startitemize

  \item Positionnement : cela dépend de l'option {\tt location} dans \tex{setupnote}. 

    \startSmallPrint
      
En principe, nous savons déjà que les notes de bas de page sont placées en bas de la page ({\tt location=page}) et les notes de fin de page au point où la commande \tex{placenotes[endnote]} ({\tt location=text}) est trouvée. ({\tt location=text}), mais nous pouvons ajuster cette fonction et définir les notes de bas de page, par exemple, en tant que {\tt location=text}. Les notes de bas de page fonctionneront alors de la même manière que les notes de fin de document et apparaîtront à l'endroit du document où se trouve la commande \tex{placenotes[footnote]}, ou la commande spécifique aux notes de bas de page \tex{placefootnotes}. Avec cette procédure, nous pourrions, par exemple, imprimer les notes sous le paragraphe dans lequel elles se trouvent.

    \stopSmallPrint


  \item Séparation des paragraphes entre les notes : par défaut, chaque note est imprimée dans son propre paragraphe, mais on peut faire en sorte que toutes les notes d'une même page soient imprimées dans le même paragraphe en définissant l'option {\tt paragraph} de \tex{setupnote} sur \MyKey{yes}. 

  \item Style et couleur dans lequel le texte de la note sera écrit : l'option {\tt style} et {\tt color} dans \tex{setupnotation}.

  \item Taille de la lettre : l'option {\tt bodyfont} dans \tex{setupnote}.

    \startSmallPrint

Cette option ne concerne que le cas où l'on souhaite définir manuellement une taille de police pour les notes de bas de page. Ce n'est presque jamais une bonne idée de le faire car, par défaut, \ConTeXt\ ajuste la taille de la police des notes de bas de page pour qu'elle soit plus petite que celle du texte principal, mais avec une taille {\em proportionnelle} à celle de la police du corps principal.

    \stopSmallPrint

\item Marge gauche pour le texte de la note : l'option {\tt margin} dans \tex{setupnotation}.

  \item Largeur maximale : l'option {\tt width} dans \tex{setupnote}.

  \item Nombre de colonnes : l'option {\tt n} de \tex{setupnote} détermine si le texte de la note sera sur deux colonnes ou plus. La valeur \quote{n} doit être un nombre entier.

\stopitemize


\head {\bf Espace entre les notes ou entre les notes et le texte~:} ici, nous avons les options suivantes :
 
\startitemize
 
\item {\tt rule}, dans \tex{setupnote} détermine s'il y aura ou non une ligne (règle) entre la zone des notes et la zone de la page contenant le texte principal. Ses valeurs possibles sont {\tt yes, on, no} et {\tt off}. Les deux premières valeurs activent la règle et la dernière la désactive. Il est possible de configurer la ligne plus précisément en indiquant {\tt rule=command} et en définisant {\tt rulecommand} par exemple 

\placefigure [force,here,none] [] {}
{\startDemoI
rule=command,
rulecommand={\blackrule[width=0.5\textwidth,color=green,height=3pt,depth=-2pt]},
\stopDemoI}
 
  \item {\tt before}, dans \tex{setupnotation} : commande ou commandes à exécuter avant l'insertion du texte de la note. Sert à insérer un espacement supplémentaire, des lignes de séparation entre les notes, etc.
 
   \item {\tt after}, dans \tex{setupnotation} : commande ou commandes à exécuter après l'insertion du texte de la note.
 
 \stopitemize

\stopitemize

Finissons par un exemple~:

\startbuffer[a7-testfootnoteE]
\setuppapersize[A7,landscape]
\setupbodyfont[8pt]
%\showframe
%\showboxes
\definenote    [MesNotesBleuesA]

\setupnotation [MesNotesBleuesA]
  [color=middleblue,
   style=bf,
   headcolor=middlecyan,
   way=bypage,
   align=flushleft,
   alternative=hanging,
   width=broad,
   distance=0.5cm,
   numberconversion=Words,
   width=1.5cm,]

\setupnote     [MesNotesBleuesA]
  [textcolor=middlered,
   textstyle=it,
   rule=command,
   rulecommand={\blackrule[width=0.5\textwidth,color=middlegreen,height=3pt,depth=-2pt]},
   width=5cm,]

\definenote [MesNotesBleuesB] [MesNotesBleuesA] [alternative=inleft,textcolor=middleyellow]

\setupnotation [MesNotesBleuesB]
  [headcolor=middlemagenta,
   alternative=serried,]

\starttext
Ce début de phrase \MesNotesBleuesA[refA]{coucou ici ceci est une note de bas de page qui s'étend tout au long de la page} est suivi d'une fin de phrase \MesNotesBleuesB[refB]{coucou là ceci est une note de bas de page qui s'étend tout au long de la page}.
\stoptext
\stopbuffer

\savebuffer[list=a7-testfootnoteE,file=ex_footnoteE.tex,prefix=no]
\placefigure [force,here,none] [] {}{\typesetbuffer[a7-testfootnoteE][frame=on,page=1,background=color,backgroundcolor=white]
\attachment
  [file={ex_footnoteE.tex},
   title={exemple testfootnoteE}]}



\stopsubsection

% *** Subsection 
\startsubsection
  [title={Exclusion temporaire des notes lors de la compilation}]

\PlaceMacro{notesenabledfalse}
\PlaceMacro{notesenabledtrue}


Les commandes \tex{notesenabledfalse} et \tex{notesenabledtrue} indiquent à \ConTeXt\ d'activer ou de désactiver la compilation des notes respectivement. Cette fonction peut être utile si l'on souhaite obtenir une version sans notes lorsque le document comporte des notes nombreuses et étendues. Dans mon expérience personnelle, par exemple, lorsque je corrige une thèse de doctorat, je préfère la lire une première fois en une seule fois, sans les notes, puis faire une seconde lecture avec les notes incorporées.

\stopsubsection



\stopsection

%==============================================================================

\stopcomponent

%%% TeX-master: "../introCTX_fra.tex"

\startcomponent 02-04-01-07_Typeset_Itemize

\environment introCTX_env_00

%==============================================================================

\startsection
  [title=Itemize,
  reference=sec:typset:itemize]


\TocChap


% *** Subsection 
\startsubsection
  [
    reference=sec:itemize_select-list-type,
    title={Sélection du type de liste et du séparateur entre les {\em items} de la liste},
  ]

% **** Subsubsection
\startsubsubsection
  [title={Listes non ordonnées}]

Par défaut, la liste générée par {\tt itemize} est une liste non ordonnée, dans laquelle le séparateur sera automatiquement sélectionné en fonction du niveau d'imbrication :

\startitemize[packed, columns, two]\switchtobodyfont[small]
\sym{\convertnumber{set 0}{1}} Pour le premier niveau d'imbrication.
\sym{\convertnumber{set 0}{2}} Pour le second niveau d'imbrication.
\sym{\convertnumber{set 0}{3}} Pour le troisième niveau d'imbrication.
\sym{\convertnumber{set 0}{4}} Pour le quatrième niveau d'imbrication.
\sym{\convertnumber{set 0}{5}} Pour le cinquième niveau d'imbrication.
\sym{\convertnumber{set 0}{6}} Pour le sixième niveau d'imbrication.
\sym{\convertnumber{set 0}{7}} Pour le septième niveau d'imbrication.
\sym{\convertnumber{set 0}{8}} Pour le huitième niveau d'imbrication.
\sym{\convertnumber{set 0}{9}} Pour le neuvième niveau d'imbrication.

\stopitemize

Cependant, nous pouvons indiquer expressément que nous voulons que le symbole associé à un niveau particulier soit utilisé, simplement en passant le numéro du niveau comme argument. Ainsi, \tex{startitemize[4]} générera une liste non ordonnée dans laquelle le caractère \triangleright, sera utilisé comme séparateur, quel que soit le niveau d'imbrication de la liste.

Nous pouvons également modifier le symbole prédéterminé pour chaque niveau avec \PlaceMacro{definesymbol} \tex{definesymbol} :

\placefigure [force,here,none] [] {}
{\startDemoHN
% definesymbol
%  [Level]
%  [Symbol associated with the level]
\definesymbol [symA] [\diamond]
\definesymbol [symB] [{\blackrule[height=1.3ex, width=0.9ex, depth=-0.4ex,]}] 
\setupitemize [1]    [packed] [color=middlegreen, symbol=symA]
\setupitemize [2]    [packed] [color=middlered,   symbol=symB]
\startitemize
\item Texte 1  \item Texte 2
\startitemize  \item Texte 3 \item Texte 4 \stopitemize
\item Texte 5
\stopitemize
\stopDemoHN}

\stopsubsubsection

% **** Subsubsection
\startsubsubsection
  [title=Listes ordonnées]

Pour générer une liste ordonnée, nous devons indiquer à {\tt itemize} le type d'ordre que nous voulons. Cela peut être~:

\startitemize[intro, packed, 2*broad, columns, three]
\switchtobodyfont[small]

\sym{{\bf n}} 1, 2, 3, 4, ...

\sym{{\bf m}} {\os 1}, {\os 2}, {\os 3}, {\os 4}, ...

\sym{{\bf g}} \alpha, \beta, \gamma, \delta, ...

\sym{{\bf G}} \Alpha, \Beta, \Gamma, \Delta, ...

\sym{{\bf a}} a, b, c, d, ...

\sym{{\bf A}} A, B, C, D, ...

\sym{{\bf KA}} \cap{a, b, c, d, ...}

\sym{}

\sym{{\bf r}} i, ii, iii, iv, ...

\sym{{\bf R}} I, II, III, IV, ...

\sym{{\bf KR}} \cap{i, ii, iii, iv, ...}

\stopitemize

La différence entre {\tt n} et {\tt m} réside dans la police utilisée pour représenter le nombre : {\tt n} utilise la police activée à ce moment-là, tandis que {\tt m} utilise une police différente, plus élégante, presque calligraphique.

\startSmallPrint

Je ne connais pas le nom de la police que {\tt m} utilise, et donc dans la liste ci-dessus je n'ai pas pu représenter exactement le type de chiffres que cette option génère. Je suggère aux lecteurs de la tester par eux-mêmes.

\stopSmallPrint

\stopsubsubsection

\stopsubsection



% *** Subsection 
\startsubsection
  [
    reference=sec:itemize_arg1,
    title={Configuration basique des listes},
  ]

Nous rappelons que \MyKey{itemize} permet deux arguments. Nous avons déjà vu comment le premier argument nous permet de sélectionner le type de liste que nous voulons. Mais nous pouvons également l'utiliser pour indiquer d'autres caractéristiques de la liste ; ceci est fait par les options suivantes pour \MyKey{itemize} dans son premier argument :

\startitemize

\item {\tt columns} : cette option détermine que la liste est composée de deux colonnes ou plus. Après l'option colonnes, le nombre de colonnes souhaité doit être écrit sous forme de mots séparés par une virgule : {\tt two, three, four, five, six, seven, eight or nine}. {\tt columns} non suivi d'un nombre quelconque génère deux colonnes.

\item {\tt intro} : cette option tente de ne pas séparer la liste, par un saut de ligne, du paragraphe qui la précède.  

\item {\tt continue} : dans les listes ordonnées (numériques ou alphabétiques), cette option permet à la liste de poursuivre la numérotation à partir de la dernière liste numérotée. Si l'option {\tt continue} est utilisée, il n'est pas nécessaire d'indiquer le type de liste que l'on souhaite, car on suppose qu'elle sera identique à la dernière liste numérotée.

\item {\tt packed} : est l'une des options les plus utilisées. Son utilisation permet de réduire au maximum l'espace vertical entre les différents {\em items} de la liste.

\item {\tt nowhite} produit un effet similaire à celui de {\tt packed}, mais plus radical : il réduit non seulement l'espace vertical entre les éléments, mais aussi l'espace vertical entre la liste et le texte environnant.

\placefigure [force,here,none] [] {}
{\startDemoVN
Texte introductif.
\startitemize[packed]
\item Premier élément
\item Second élément
\item Troisième élément
\stopitemize
Texte introductif.
\startitemize[nowhite]
\item Premier élément
\item Second élément
\item Troisième élément
\stopitemize
Texte introductif.
\stopDemoVN}

\item {\tt after, before:} rajoutent au contraire des espaces respectivement après ou avant la liste

\placefigure [force,here,none] [] {}
{\startDemoVN
Texte introductif.
\startitemize[nowhite,after]
\item Premier élément
\item Second élément
\item Troisième élément
\stopitemize
Texte introductif.
\startitemize[nowhite,before]
\item Premier élément
\item Second élément
\item Troisième élément
\stopitemize
Texte introductif.
\stopDemoVN}

\item {\tt joinedup :} assure le rôle de {\tt nowhite} mais entre un niveau d'item et son parent dans le cas de listes imbriquées.

\placefigure [force,here,none] [] {}
{\startDemoVW%
\setupitemgroup[itemize] [1] [nowhite]%
\setupitemgroup[itemize] [2] [nowhite]%
\startitemize  
\item item 1.1
\startitemize
\item item 2.1 \item item 2.2
\stopitemize   
\item item 1.2
\stopitemize

\setupitemgroup[itemize] [1] [joinedup]%
\startitemize  
\item item 1.1
\startitemize
\item item 2.1 \item item 2.2
\stopitemize   
\item item 1.2
\stopitemize
\stopDemoVW}



\item {\tt broad} : augmente l'espace horizontal entre le séparateur d'élément et le texte de l'élément. L'espace peut être augmenté en multipliant un nombre par {\tt broad} comme dans, par exemple : \type{\startitemize[2*broad]}.

\placefigure [force,here,none] [] {}
{\startDemoVN
\startitemize[packed,2*broad]
\item Premier élément
\item Second élément
\stopitemize
\stopDemoVN}

\item {\tt serried} : supprime l'espace horizontal entre le séparateur d'élément et le texte.

\item {\tt intext} : supprime le retrait suspendu.

\item {\tt text} : supprime le retrait suspendu et réduit l'espace vertical entre les éléments.

\item {\tt repeat} : dans les listes imbriquées, la numérotation d'un niveau enfant {\em repeat} devient le même niveau que le niveau précédent. Ainsi, nous aurions, au premier niveau : 1, 2, 3, 4 ; au deuxième niveau : 1.1, 1.2, 1.3, etc. L'option doit être indiquée pour la liste intérieure, pas pour la liste extérieure.

\item {\tt margin, inmargin} : par défaut, le séparateur de liste est imprimé à gauche, mais à l'intérieur de la zone de texte elle-même ({\tt atmargin}). Les options {\tt margin} et {\tt inmargin} permettent de déplacer le séparateur vers la marge.

\stopitemize

\stopsubsection

% *** Subsection 
\startsubsection
  [
    reference=sec:itemize_arg2,
    title={Configuration complémentaire des listes},
  ]

Le deuxième argument, également facultatif, de \tex{startitemize} permet une configuration plus détaillée et plus approfondie des listes.

\startitemize

\item {\tt before, after} : commandes à exécuter respectivement avant le démarrage ou après la fermeture de l'environnement itemize (à ne pas confondre avec ces options de l'argument 1, qui font référence aux espacements verticaux voir ci-dessus).

\item {\tt inbetween} : commande à exécuter entre deux {\tt items}.

\item {\tt beforehead, afterhead} : commande à exécuter avant ou après un élément saisi avec la commande \tex{head}.

\item {\tt left, right} : caractère à imprimer à gauche ou à droite du séparateur. Par exemple, pour obtenir des listes alphabétiques dans lesquelles les lettres sont entourées de parenthèses, il faudrait écrire \tex{startitemize[a][left=(, right=)]}

\item {\tt stopper} : indique un caractère à écrire après le séparateur. Ne fonctionne que dans les listes ordonnées.

\item {\tt width, maxwidth} : largeur de l'espace réservé au séparateur et donc au retrait suspendu.

\item {\tt factor} : nombre représentatif du facteur de séparation entre le séparateur et le texte.

\item {\tt distance} : mesure de la distance entre le séparateur et le texte.

\placefigure [force,here,none] [] {}
{\startDemoVW%
\setupitemgroup[itemize]   [1]
               [nowhite,joinedup,3*broad]%
\setupitemgroup[itemize]   [2]
               [nowhite] [width=5mm]%
\startitemize  
\item item 1.1
\startitemize
\item item 2.1 \item item 2.2
\stopitemize   
\item item 1.2
\stopitemize
\setupitemgroup[itemize] [1] [distance=5mm]%
\setupitemgroup[itemize] [2] [distance=5mm]%
\startitemize  
\item item 1.1
\startitemize
\item item 2.1 \item item 2.2
\stopitemize   
\item item 1.2
\stopitemize
\stopDemoVW}


\item {\tt leftmargin, rightmargin, margin} : marge à ajouter à gauche (leftmargin) ou à droite (rightmargin) des éléments. 



\item {\tt start} : numéro à partir duquel la numérotation des éléments commencera.

\item {\tt symalign, itemalign, align} : alignement des éléments. Permet les mêmes valeurs que \tex{setupalign}. {\tt symalign} contrôle l'alignement du séparateur, {\tt itemalign} celui du texte de l'élément et {\tt align} contrôle les deux.

\item {\tt identing} : indentation de la première ligne des paragraphes
à l'intérieur de l'environnement. Voir \in{section}[sec:indentation].

\item {\tt indentnext} : indique si le paragraphe suivant l'environnement doit être indenté ou non. Les valeurs sont {\em yes, no} et {\tt auto}.

\item {\tt items} : dans les éléments saisis en entrée avec \tex{its}, indique le nombre de fois que le séparateur doit être reproduit.

\item {\tt style, color ; headstyle, headcolor ; marstyle, marcolor ; symstyle, symcolor} : ces options contrôlent le style et la couleur des éléments lors de leur saisie dans l'environnement avec les commandes \tex{item}, \tex{head}, \tex{mar} ou \tex{sym}.

\stopitemize

\stopsubsection

\stopsection

%==============================================================================

\stopcomponent

%%% TeX-master: "../introCTX_fra.tex"

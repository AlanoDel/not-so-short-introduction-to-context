\startcomponent 02-04-01-14_Typeset_Macrostructure

\environment introCTX_env_00

%==============================================================================

\startsection
  [title=Macro-structure du document,
  reference=sec:typset:macrostr,]

\TocChap

\startsubsection


Les quatre environnements permettent les quatre mêmes options : \MyKey{page}, \MyKey{before}, \MyKey{after} et \MyKey{number}, et leurs valeurs et utilité sont les mêmes que celles trouvées dans \tex{setuphead} (voir \in{section}[sec:setuphead]), bien que nous devons noter qu'ici l'option \MyKey{number=no} éliminera la numérotation de toutes les commandes de sectionnement dans l'environnement.

L'inclusion de l'une de ces macro-sections (ou bloc) dans notre document n'a de sens que si elle permet d'établir une certaine différenciation entre elles. Il peut s'agir d'en-têtes ou d'une numérotation de pages dans le corps du texte. La configuration de chacun de ces blocs est réalisée par \PlaceMacro{setupsectionblock} \tex{setupsectionblock} dont la syntaxe est :


\placefigure [force,here,none] [] {}{
\startDemoI
\setupsectionblock [NomMacroSection] [Options]
\stopDemoI}

où {\em NomMacroSection} peut être {\tt frontpart}, {\tt bodypart}, {\tt appendix} ou {\tt backpart} et les options peuvent être les mêmes que celles mentionnées précédemment : \MyKey{page}, \MyKey{number}, \MyKey{before} et \MyKey{after}. 


% TODO Garulfo  added begin ---------------------------------------

\ConTeXt permet d'attribuer des configurations différentes aux commandes selon que l'on fait appel à elles dans telle où telle macro-section (ou bloc). La fonction pour cela est \PlaceMacro{startsectionblockenvironment} \tex{startsectionblockenvironment}. Ainsi, par exemple, pour s'assurer que dans {\em frontmatter} les pages sont numérotées avec des chiffres romains, nous devrions écrire dans le préambule de notre document :

\placefigure [force,here,none] [] {}{
\startDemoI
\startsectionblockenvironment[frontpart]
\setupuserpagenumber
   [numberconversion=romannumerals]
\stopsectionblockenvironment
\stopDemoI}

%-----------

\startbuffer[demomacrosection]

\startsectionblockenvironment[frontpart]
\setupuserpagenumber
   [numberconversion=romannumerals]
\setuphead 
   [section] 
   [style=italic]
\stopsectionblockenvironment

\startsectionblockenvironment[bodypart]
\setuphead
  [section]
  [style=bold]
\stopsectionblockenvironment

\starttext

\startfrontmatter
  \startsection[title=Titre premier]
  Texte \quotation{front matter}.
  \stopsection
\stopfrontmatter

\startbodymatter
  \startsection[title=Titre second]
  Texte \quotation{body matter}.
  \stopsection
\stopbodymatter

\stoptext
\stopbuffer

\savebuffer[list=demomacrosection,file=ex_macrosection.tex,prefix=no]
\attachment
  [file={ex_macrosection.tex},
   title={ex_macrosection}]


% TODO Garulfo  end  ---------------------------------------


Les configurations par défaut de ces quatre macro-sections (ou bloc) impliquent notamment que~:

\startitemize[packed]

  \item Les quatre macro-section commencent une nouvelle page.

  \item La numérotation des sections dépend de la macro-section~:

  \startitemize

    \item Dans {\tt frontmatter} et {\tt backmatter} toutes les sections numérotées deviennent non numérotées par défaut.

    \item Dans {\tt bodymatter} les sections ont une numérotation arabe.

    \item Dans les {\tt appendices}, les sections sont numérotées en majuscules.

  \stopitemize

\stopitemize

Il est également possible de créer de nouveaux macro-section avec \PlaceMacro{definesectionblock} \tex{definesectionblock}.


\stopsubsection


\stopsection

%==============================================================================

\stopcomponent

%%% TeX-master: "../introCTX_fra.tex"

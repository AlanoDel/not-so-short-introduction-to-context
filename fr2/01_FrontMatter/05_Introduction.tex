\startcomponent 02-01_Introduction

\environment introCTX_env_00

\startchapter
  [title=Introduction\\~et vue d'ensemble,reference=cap:panorama]

%==============================================================================

\ConTeXt\ est un {\em système de composition et de mise en page} 
de documents électroniques. Il vise à donner à son utilisateur un contrôle le plus complet 
et précis possible sur l'impression (sur papier) ou l'affichage (sur écran)
des documents électroniques.

\startitemize
\item {\em La composition} au sens de l'imprimerie, consiste à mettre en forme
      le contenu textuel pour préparer son impression, son affichage.
\item {\em La mise en page}, toujours au sens de l'imprimerie, consiste à réunir 
      et assembler tous les éléments qui constitueront la page finalisée
      (textes mais aussi images, tableaux, couleurs...)
\stopitemize

Pour être à l'aise avec l'utilisation de \ConTeXt , comme avec toute technologie
informatique, il est bon d'en comprendre la logique sous-jacente. 
Celle de \ConTeXt\ s'appuie sur trois concepts~:

\startitemize[n]
\item les fichiers sources,
\item le balisage,
\item la composition et la mise en page.
\stopitemize

\startsubsubsubject [title={Les fichiers sources}]

Les fichiers sources constituent la donnée d'entrée.
Il s'agit de ce qu'il y a de plus simple en informatique~: des fichiers textes. 
Ce sont des fichiers lisibles sur tous les ordinateurs du monde, avec un simple éditeur de texte.

Cette universalité se fait au prix de deux conditions~: (1)
Ces fichiers ne contiennent que des suites de caractères 
(pas de couleur, pas de mise en page, pas d'images, pas de sons… rien que du texte), 
et (2) ils utilisent un codage particulier des caractères 
(c'est à dire des règles standards de représentation informatique des caractères)
que l'on appelle \goto{UTF-8}[url(https://fr.wikipedia.org/wiki/UTF-8)]. 
Ce codage, dorénavant utilisé par plus de 97\% des sites web, 
permet de représenter les milliers de caractères du répertoire universel des caractères codés
(c'est à dire la très grande majorité des caractères utilisés par les différentes langues du monde).

\ConTeXt\ prend donc en entrée des fichiers textes. 
Ces derniers servent de {\em contenant} au {\em contenu} informationnel de votre document. 
Pour le texte de votre document (les mots, les paragraphes), c'est assez facile à comprendre. 
Mais la question se pose pour les autres informations que vous souhaitez exprimer, 
par exemple \quotation{ici, il faut telle image}, \quotation{ce texte est un titre de chapitre}, \quotation{ce texte doit être en rouge}, \quotation{la page doit avoir un format A4}…

Comme le fichier ne contient que des suites de caractères, 
toutes ces informations (sémantique, structure, couleurs, composition, mise en page...) 
vont devoir être décrites par des suites de caractères. Cela se fait en deux temps, avec les deux concepts suivants.

\stopsubsubsubject

\startsubsubsubject [title={Le balisage}]

L'utilisateur doit distinguer les différents type d'information au sein des fichiers sources.
La science informatique a développé pour cela des techniques et langages dits de {\em balisage} tels que 
\goto{HTML}[url(https://fr.wikipedia.org/wiki/Hypertext_Markup_Language)],   
\goto{Markdown}[url(https://daringfireball.net/projects/markdown/syntax)], 
\goto{Org-mode}[url(https://orgmode.org/)], 
et \goto{\ConTeXt}[url(https://wiki.contextgarden.net)].

Apprendre \ConTeXt\ implique donc d'apprendre le langage de balisage 
de \ConTeXt\ ce qui nécessite un peu d'effort et de rigueur.

\stopsubsubsubject 

\startsubsubsubject [title={La composition et mise en page}]

Une fois les différents types d'information repérés,
l'utilisateur doit encore indiquer les règles de composition et de mise en page à appliquer à chacun.
La science informatique a également développé pour cela des techniques et langages dédiés tels que 
\goto{CSS}[url(https://fr.wikipedia.org/wiki/Feuilles_de_style_en_cascade)] (language de composition pour les éléments balisés avec HTML),
et... \goto{\ConTeXt}[url(https://wiki.contextgarden.net)].

Le langage \ConTeXt\ permet donc, au-delà du balisage, de configurer l'ensemble des règles de composition et de mise en page à appliquer.

Ce qui est très appréciable et apprécié, c'est que la configuration par défaut donne d'excellents résultats.
Il {\em suffit} ensuite de la personnaliser, la compléter, la détailler.
On découvre alors que l'art de la composition est particulièrement riche et vaste 
-- \quotation{tout est permis} et \quotation{l'imagination est la seule limite} --
et le vocabulaire de \ConTeXt\ l'est également même s'il bénéficie d'un effort continu de standardisation.

\stopsubsubsubject

On peut estimer au nombre de 100 les mots de vocabulaire
qui couvriront la très grande majorité de vos besoins -- même pour des documents complexes -- 
à la fois de balisage, de composition et de mise en page.

Pour progresser dans cet apprentissage, ce document procéde en 4 parties.

\startsubsection [title={Les fichiers sources}]

Ils font l'objet de la \in{Partie}[part:src].
Pour fixer rapidement les idées, nous commencerons par un premier exemple au \in{Chapitre}[cap:firstdoc]
et nous découvrirons une première balise~: \tex{starttext}.
La logique et les règles syntaxiques seront ensuite explicitées au \in{Chapitre}[cap:commands].
Puis nous verrons comment organiser nos fichiers sources au \in{Chapitre}[cap:sourcefile] avec notamment~:

\start
\tfx
\bTABLE[option=stretch]
\setupTABLE[frame=off]
\setupTABLE[r][1][style=\bf,width=4cm,topframe=on]
\bTR \bTD Organisation des fichiers sources \eTD \eTR
\bTR \bTD \tex{startproject}                \eTD \eTR
\bTR \bTD \tex{startproduct}                \eTD \eTR
\bTR \bTD \tex{product}                     \eTD \eTR
\bTR \bTD \tex{startcomponent}              \eTD \eTR
\bTR \bTD \tex{component}                   \eTD \eTR
\bTR \bTD \tex{startenvironment}            \eTD \eTR
\bTR \bTD \tex{environment}                 \eTD \eTR
\bTR \bTD \tex{usepath}                     \eTD \eTR
\eTABLE
\stop


\stopsubsection 

% * Section 2 - la composition et mise en page ================================

\startsection [title=La composition d'ensemble]

Nous verrons en \in{Partie}[part:global] les éléments de composition d'ensemble. 
Ils sont désignés ainsi car ils affectent l'ensemble de la composition et mise en page du document final.
Il s'agit des 7 sujets suivants~:

\startcolumns[n=2]
\tfx
\bTABLE[frame=off,option=stretch]
\setupTABLE[r][1][style=\bf,width=4cm,topframe=on,width=5cm]
\bTR \bTD Mise en page                           \eTD \eTR 
\bTR \bTD \tex{definepapersize}                  \eTD \eTR
\bTR \bTD \tex{setuppapersize}                   \eTD \eTR
\bTR \bTD \tex{setuplayout}                      \eTD \eTR
\bTR \bTD \tex{setupbackgrounds}                 \eTD \eTR
\bTR \bTD \tex{showframe}                        \eTD \eTR
\eTABLE

\bTABLE[frame=off,option=stretch]
\setupTABLE[r][1][style=\bf,width=4cm,topframe=on,width=5cm]                                                                                         
\bTR \bTD En-tête, pied et numérotation de page  \eTD \eTR
\bTR \bTD \tex{setupfootertexts}                 \eTD \eTR
\bTR \bTD \tex{setupheadertexts}                 \eTD \eTR
\bTR \bTD \tex{setuppagenumbering}               \eTD \eTR                                           
\eTABLE

\bTABLE[frame=off,option=stretch]
\setupTABLE[r][1][style=\bf,width=4cm,topframe=on,width=5cm]                                                
\bTR \bTD Paragraphes                            \eTD \eTR                 
\bTR \bTD \tex{par}                              \eTD \eTR
\bTR \bTD \tex{crlf}                             \eTD \eTR
\bTR \bTD \tex{setupinterlinespace}              \eTD \eTR
\bTR \bTD \tex{setupwhitespace}                  \eTD \eTR
\bTR \bTD \tex{setupalign}                       \eTD \eTR
\bTR \bTD \tex{setuptolerance}                   \eTD \eTR
\eTABLE

\bTABLE[frame=off,option=stretch]
\setupTABLE[r][1][style=\bf,width=4cm,topframe=on,width=5cm]                                                                                                              
\bTR \bTD Polices                                \eTD \eTR
\bTR \bTD \tex{setupbodyfont}                    \eTD \eTR
\bTR \bTD \tex{switchtobodyfont}                 \eTD \eTR
\bTR \bTD \tex{definedfont}                      \eTD \eTR
\bTR \bTD \tex{definefontfeature}                \eTD \eTR
\eTABLE

\bTABLE[frame=off,option=stretch]
\setupTABLE[r][1][style=\bf,width=4cm,topframe=on,width=5cm]                                                                                                              
\bTR \bTD Couleurs                               \eTD \eTR
\bTR \bTD \tex{usecolors}                        \eTD \eTR
\bTR \bTD \tex{color}                            \eTD \eTR
\bTR \bTD \tex{definecolor}                      \eTD \eTR
\eTABLE

\bTABLE[frame=off,option=stretch]
\setupTABLE[r][1][style=\bf,width=4cm,topframe=on,width=5cm]                                                                                                             
\bTR \bTD Langue                                 \eTD \eTR
\bTR \bTD \tex{language}                         \eTD \eTR
\bTR \bTD \tex{mainlanguage}                     \eTD \eTR
\bTR \bTD \tex{setcharacterspacing}              \eTD \eTR
\eTABLE

\bTABLE[frame=off,option=stretch]
\setupTABLE[r][1][style=\bf,width=4cm,topframe=on,width=5cm]                                                                                                 
\bTR \bTD Interactivité                          \eTD \eTR
\bTR \bTD \tex{setupattachments}                 \eTD \eTR
\bTR \bTD \tex{setupinteraction}                 \eTD \eTR
\eTABLE
\stopcolumns


\stopsection

% * Section 2 ===================================================

\startsection [title=Éléments du flux principal]

Nous verrons en \in{Partie}[part:mainflow] le balisage et la composition des éléments du {\bf flux principal} d'information qui se déroule linéairement de page en page, ou bien d'écran en écran. Il s'agit essentiellement des 17 éléments suivants, en partant du plus petit /  moins structurant, au plus macroscopique / plus structurant~:

\startcolumns[n=2]
\tfx
\bTABLE[frame=off,option=stretch]
\setupTABLE[r][1][style=\bf,width=4cm,topframe=on,width=5cm]                                                                             
\bTR \bTD Emphase de mots               \eTD \eTR
\bTR \bTD \tex{em}                      \eTD \eTR
\bTR \bTD \tex{definehighlight}         \eTD \eTR
\bTR \bTD \tex{cap}                     \eTD \eTR
\eTABLE

\bTABLE[frame=off,option=stretch]
\setupTABLE[r][1][style=\bf,width=4cm,topframe=on,width=5cm] 
\bTR \bTD Emphase de paragraphes        \eTD \eTR
\bTR \bTD \tex{startalign}              \eTD \eTR
\bTR \bTD \tex{setupnarrower}           \eTD \eTR
\bTR \bTD \tex{startnarrower}           \eTD \eTR
\eTABLE

\bTABLE[frame=off,option=stretch]
\setupTABLE[r][1][style=\bf,width=4cm,topframe=on,width=5cm] 
\bTR \bTD Encadrement                   \eTD \eTR         
\bTR \bTD \tex{startframed}             \eTD \eTR  
\eTABLE

\bTABLE[frame=off,option=stretch]
\setupTABLE[r][1][style=\bf,width=4cm,topframe=on,width=5cm] 
\bTR \bTD Lignes et traits              \eTD \eTR                 
\bTR \bTD \tex{starttextrule}           \eTD \eTR   
\bTR \bTD \tex{blackrule}               \eTD \eTR   
\eTABLE

\bTABLE[frame=off,option=stretch]
\setupTABLE[r][1][style=\bf,width=4cm,topframe=on,width=5cm] 
\bTR \bTD Citations                     \eTD \eTR        
\bTR \bTD \tex{quote}                   \eTD \eTR  
\bTR \bTD \tex{quotation}               \eTD \eTR  
\bTR \bTD \tex{startquotation}          \eTD \eTR  
\eTABLE

\bTABLE[frame=off,option=stretch]
\setupTABLE[r][1][style=\bf,width=4cm,topframe=on,width=5cm] 
\bTR \bTD Mathématiques                 \eTD \eTR
\bTR \bTD \tex{startformula}            \eTD \eTR  
\bTR \bTD \tex{startcases}              \eTD \eTR  
\bTR \bTD \tex{stopitemize}             \eTD \eTR  
\eTABLE

\bTABLE[frame=off,option=stretch]
\setupTABLE[r][1][style=\bf,width=4cm,topframe=on,width=5cm] 
\bTR \bTD Listes structurées            \eTD \eTR
\bTR \bTD \tex{startitemize}            \eTD \eTR  
\bTR \bTD \tex{item}                    \eTD \eTR  
\eTABLE

\bTABLE[frame=off,option=stretch]
\setupTABLE[r][1][style=\bf,width=4cm,topframe=on,width=5cm] 
\bTR \bTD Description et énumération    \eTD \eTR
\bTR \bTD \tex{definedescription}       \eTD \eTR
\bTR \bTD \tex{defineenumeration}       \eTD \eTR
\eTABLE

\bTABLE[frame=off,option=stretch]
\setupTABLE[r][1][style=\bf,width=4cm,topframe=on,width=5cm] 
\bTR \bTD Textes tabulés                \eTD \eTR             
\bTR \bTD \tex{starttabulate}           \eTD \eTR
\bTR \bTD \tex{NR}                      \eTD \eTR
\bTR \bTD \tex{NC}                      \eTD \eTR
\eTABLE

\bTABLE[frame=off,option=stretch]
\setupTABLE[r][1][style=\bf,width=4cm,topframe=on,width=5cm] 
\bTR \bTD Tableaux                      \eTD \eTR      
\bTR \bTD \tex{bTABLE}                  \eTD \eTR
\bTR \bTD \tex{bTABLEbody}              \eTD \eTR
\bTR \bTD \tex{bTABLEhead}              \eTD \eTR
\bTR \bTD \tex{bTD}                     \eTD \eTR
\bTR \bTD \tex{bTH}                     \eTD \eTR
\bTR \bTD \tex{bTR}                     \eTD \eTR
\eTABLE

\bTABLE[frame=off,option=stretch]
\setupTABLE[r][1][style=\bf,width=4cm,topframe=on,width=5cm] 
\bTR \bTD Images et Combinaisons        \eTD \eTR
\bTR \bTD \tex{externalfigure}          \eTD \eTR
\bTR \bTD \tex{setupexternalfigure}     \eTD \eTR
\bTR \bTD \tex{startcombination}        \eTD \eTR
\eTABLE

\bTABLE[frame=off,option=stretch]
\setupTABLE[r][1][style=\bf,width=4cm,topframe=on,width=5cm] 
\bTR \bTD Objets flottants              \eTD \eTR                
\bTR \bTD \tex{startplacetable}         \eTD \eTR
\bTR \bTD \tex{startplacefigure}        \eTD \eTR
\bTR \bTD \tex{setupcaption}            \eTD \eTR
\eTABLE

\bTABLE[frame=off,option=stretch]
\setupTABLE[r][1][style=\bf,width=4cm,topframe=on,width=5cm] 
\bTR \bTD Colonnes                      \eTD \eTR        
\bTR \bTD \tex{startcolumns}            \eTD \eTR
\eTABLE

\bTABLE[frame=off,option=stretch]
\setupTABLE[r][1][style=\bf,width=4cm,topframe=on,width=5cm]
\bTR \bTD Section                        \eTD \eTR
\bTR \bTD \tex{startpart}                \eTD \eTR
\bTR \bTD \tex{startchapter}             \eTD \eTR
\bTR \bTD \tex{startsection}             \eTD \eTR
\bTR \bTD \tex{startsubsection}          \eTD \eTR
\bTR \bTD \tex{startsubsubsection}       \eTD \eTR
\bTR \bTD \tex{startsubsubsubsection}    \eTD \eTR
\bTR \bTD \tex{setuphead}                \eTD \eTR
\eTABLE

\bTABLE[frame=off,option=stretch]
\setupTABLE[r][1][style=\bf,width=4cm,topframe=on,width=5cm]
\bTR \bTD Macro-structure                \eTD \eTR
\bTR \bTD \tex{startfrontmatter}         \eTD \eTR
\bTR \bTD \tex{startbodymatter}          \eTD \eTR
\bTR \bTD \tex{startappendices}          \eTD \eTR
\bTR \bTD \tex{startbackmatter}          \eTD \eTR
\eTABLE

\bTABLE[frame=off,option=stretch]
\setupTABLE[r][1][style=\bf,width=4cm,topframe=on,width=5cm]
\bTR \bTD Page de couverture et de titre \eTD \eTR
\bTR \bTD \tex{definemakeup}             \eTD \eTR
\bTR \bTD \tex{startmakeup}              \eTD \eTR
\eTABLE

\bTABLE[frame=off,option=stretch]
\setupTABLE[r][1][style=\bf,width=4cm,topframe=on,width=5cm]
\bTR \bTD Autres éléments spécialisés    \eTD \eTR
\bTR \bTD \tex{startbuffer}              \eTD \eTR
\bTR \bTD \tex{getbuffer}                \eTD \eTR
\bTR \bTD \tex{startsetups}              \eTD \eTR
\bTR \bTD \tex{setups}                   \eTD \eTR
\eTABLE

\stopcolumns

\stopsection

% * Section 2 ===================================================

\startsection [title=Compléments au flux principal]

Nous verrons en \in{Partie}[part:complements] le balisage et la composition des {\bf compléments au flux principal}. Ils permettent de l'enrichir, d'y naviguer, de faire des connexions complexes à d'autres informations. Ces éléments permettent de sortir des limites du flux linéaire principal et sont en ce sens tout à fait essentiels~:


\startcolumns[n=2]
\tfx
\bTABLE[frame=off,option=stretch]
\setupTABLE[r][1][style=\bf,width=4cm,topframe=on,width=5cm]
\bTR \bTD Table des matières             \eTD \eTR
\bTR \bTD \tex{setupcombinedlist}        \eTD \eTR
\bTR \bTD \tex{placecontent}             \eTD \eTR
\eTABLE

\bTABLE[frame=off,option=stretch]
\setupTABLE[r][1][style=\bf,width=4cm,topframe=on,width=5cm]
\bTR \bTD Abréviations et glossaire      \eTD \eTR                         
\bTR \bTD \tex{abbreviation}             \eTD \eTR
\bTR \bTD \tex{placelistofabbreviations} \eTD \eTR
\eTABLE

\bTABLE[frame=off,option=stretch]
\setupTABLE[r][1][style=\bf,width=4cm,topframe=on,width=5cm]
\bTR \bTD Notes de bas de page           \eTD \eTR                    
\bTR \bTD \tex{footnote}                 \eTD \eTR
\bTR \bTD \tex{setupfootnotedefinition}  \eTD \eTR
\bTR \bTD \tex{setupfootnotes}           \eTD \eTR
\eTABLE

\bTABLE[frame=off,option=stretch]
\setupTABLE[r][1][style=\bf,width=4cm,topframe=on,width=5cm]
\bTR \bTD Notes marginales               \eTD \eTR                
\bTR \bTD \tex{margintext}               \eTD \eTR
\eTABLE

\bTABLE[frame=off,option=stretch]
\setupTABLE[r][1][style=\bf,width=4cm,topframe=on,width=5cm]
\bTR \bTD Pièces jointes                 \eTD \eTR              
\bTR \bTD \tex{attachment}               \eTD \eTR
\eTABLE

\bTABLE[frame=off,option=stretch]
\setupTABLE[r][1][style=\bf,width=4cm,topframe=on,width=5cm]
\bTR \bTD Références internes            \eTD \eTR                   
\bTR \bTD \tex{reference}                \eTD \eTR
\bTR \bTD \tex{at}                       \eTD \eTR
\bTR \bTD \tex{in}                       \eTD \eTR
\eTABLE

\bTABLE[frame=off,option=stretch]
\setupTABLE[r][1][style=\bf,width=4cm,topframe=on,width=5cm]
\bTR \bTD Références externes             \eTD \eTR                  
\bTR \bTD \tex{from}                     \eTD \eTR
\bTR \bTD \tex{goto}                     \eTD \eTR
\bTR \bTD \tex{useurl}                   \eTD \eTR
\bTR \bTD \type{url()}                   \eTD \eTR
\eTABLE

\bTABLE[frame=off,option=stretch]
\setupTABLE[r][1][style=\bf,width=4cm,topframe=on,width=5cm]
\bTR \bTD Références bibliographiques    \eTD \eTR                           
\bTR \bTD \tex{cite}                     \eTD \eTR
\bTR \bTD \tex{placelistofpublications}  \eTD \eTR
\bTR \bTD \tex{usebtxdataset}            \eTD \eTR
\eTABLE

\bTABLE[frame=off,option=stretch]
\setupTABLE[r][1][style=\bf,width=4cm,topframe=on,width=5cm]
\bTR \bTD Listes des images, tableaux... \eTD \eTR                              
\bTR \bTD \tex{placelistoffigures}       \eTD \eTR
\bTR \bTD \tex{placelistoftables}        \eTD \eTR
\eTABLE

\bTABLE[frame=off,option=stretch]
\setupTABLE[r][1][style=\bf,width=4cm,topframe=on,width=5cm]
\bTR \bTD Index                          \eTD \eTR     
\bTR \bTD \tex{index}                    \eTD \eTR
\bTR \bTD \tex{placeindex}               \eTD \eTR
\eTABLE

\stopcolumns


\stopsection


\startsection[title=Galerie]

Pour finir cette introduction, la galerie suivante vous donnera un aperçu visuel
de l'ensemble des différents éléments évoqués durant cette introduction.

% pdfjam demonstration.pdf  --nup 2x22 --frame true   --delta "1mm 2mm" --outfile resultat.pdf
% pdfcrop resultat.pdf resultat.pdf

%\showframe 

\start
\setupexternalfigure[frame=on,height=6cm,interaction=yes,background=color,backgroundcolor=white]

\startcombination[2*3]
{\externalfigure[demonstration.pdf][page=1] }{}
{\externalfigure[demonstration.pdf][page=2] }{}
{\externalfigure[demonstration.pdf][page=3] }{}
{\externalfigure[demonstration.pdf][page=4] }{}
{\goto{\externalfigure[demonstration.pdf][page=5]}[cap:pages]        }{}
{\goto{\externalfigure[demonstration.pdf][page=6]}[sec:headerfooter] }{}
\stopcombination

\page

\startcombination[2*3]
{\externalfigure[demonstration.pdf][page=7] }{}
{\externalfigure[demonstration.pdf][page=8] }{}
{\externalfigure[demonstration.pdf][page=9] }{}
{\externalfigure[demonstration.pdf][page=10]}{}
{\externalfigure[demonstration.pdf][page=11]}{}
{\externalfigure[demonstration.pdf][page=12]}{}
\stopcombination

\page

\startcombination[2*3]
{\externalfigure[demonstration.pdf][page=13]}{}
{\externalfigure[demonstration.pdf][page=14]}{}
{\externalfigure[demonstration.pdf][page=15]}{}
{\externalfigure[demonstration.pdf][page=16]}{}
{\externalfigure[demonstration.pdf][page=17]}{}
{\externalfigure[demonstration.pdf][page=18]}{}
\stopcombination

\page

\startcombination[2*3]
{\externalfigure[demonstration.pdf][page=19]}{}
{\externalfigure[demonstration.pdf][page=20]}{}
{\externalfigure[demonstration.pdf][page=21]}{}
{\externalfigure[demonstration.pdf][page=22]}{}
{\externalfigure[demonstration.pdf][page=23]}{}
{\externalfigure[demonstration.pdf][page=24]}{}
\stopcombination

\page

\startcombination[2*3]
{\externalfigure[demonstration.pdf][page=25]}{}
{\externalfigure[demonstration.pdf][page=26]}{}
{\externalfigure[demonstration.pdf][page=27]}{}
{\externalfigure[demonstration.pdf][page=28]}{}
{\externalfigure[demonstration.pdf][page=29]}{}
{\externalfigure[demonstration.pdf][page=30]}{}
\stopcombination

\page

\startcombination[2*3]
{\externalfigure[demonstration.pdf][page=31]}{}
{\externalfigure[demonstration.pdf][page=32]}{}
{\externalfigure[demonstration.pdf][page=33]}{}
{\externalfigure[demonstration.pdf][page=34]}{}
{\externalfigure[demonstration.pdf][page=35]}{}
{\externalfigure[demonstration.pdf][page=36]}{}
\stopcombination

\page

\startcombination[2*3]
{\externalfigure[demonstration.pdf][page=37]}{}
{\externalfigure[demonstration.pdf][page=38]}{}
{\externalfigure[demonstration.pdf][page=39]}{}
{\externalfigure[demonstration.pdf][page=40]}{}
{\externalfigure[demonstration.pdf][page=41]}{}
{\externalfigure[demonstration.pdf][page=42]}{}
\stopcombination

\page

\startcombination[2*1]
{\externalfigure[demonstration.pdf][page=43]}{}
{\externalfigure[demonstration.pdf][page=44]}{}
\stopcombination

\stop

\stopsection


%%%%%%%%%%%%%%%%%%

\stopchapter                                            
                                                        
\stopcomponent                                          
                                                        
%%% Local Variables:
%%% mode: ConTeXt
%%% mode: auto-fill
%%% TeX-master: "../introCTX_fra.tex"
%%% coding: utf-8-unix
%%% End:

 


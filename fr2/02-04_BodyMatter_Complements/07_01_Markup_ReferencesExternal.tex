\startcomponent 02-02-02-06_Markup_ReferencesExternal

\environment introCTX_env_00

%==============================================================================

\startsection
  [title=Références externes,
  reference=sec:mkp:refext]


\TocChap


\tex{useURL}
\tex{url}
\tex{from}
\tex{goto}

Je distinguerai les commandes qui ne créent pas le lien mais aident à composer l'URL du lien, et les commandes qui créent l'hyperlien à proprement parler. Examinons-les séparément~:

% ***  Section Commands that help typeset the hyperlinks

% TODO Garulfo modifier la formulation (définir + afficher)
\startsubsection
[title={Commandes de définition d'hyperlien (sans les introduire dans le document)}]

Les URL ont tendance à être très longues, et comprennent des caractères de tous types, même des caractères réservés en \ConTeXt\ et qui ne peuvent pas être utilisés directement. De plus, lorsque l'URL doit être affichée dans le document, il est très difficile de composer le paragraphe, car l'URL peut dépasser la longueur d'une ligne et ne comporte jamais d'espaces vides pouvant être utilisés pour insérer un saut de ligne. De plus, dans une URL, il n'est pas raisonnable de césurer les mots pour insérer des sauts de ligne, car le lecteur pourrait difficilement savoir si la césure fait ou non partie de l'URL.

C'est pourquoi \ConTeXt\ fournit deux utilitaires pour {\em la composition} des URL. Le premier est principalement destiné aux URL qui seront utilisées en interne, mais qui ne seront pas réellement affichées dans le document. Le second est destiné aux URL qui doivent être écrits dans le texte du document. Examinons-les séparément~:


\startdescription{\tex{useURL}}\PlaceMacro{useURL}

Cette commande nous permet d'écrire une URL dans le préambule du document, en l'associant à un nom, de sorte que lorsque nous voulons l'utiliser ensuite dans notre document, nous pouvons l'invoquer par le nom qui lui est associé. Elle est particulièrement utile pour les URL qui seront utilisées plusieurs fois dans le document.

Cette commande permet deux utilisations :

\startitemize[n, packed]

\item \type{\useURL [NomAssocié] [URL]}
\item \type{\useURL [NomAssocié] [URL] [] [Texte du lien]}

\stopitemize

\startitemize

\item Dans la première version, l'URL est simplement associée au nom par lequel elle sera invoquée dans notre document. Mais alors, pour utiliser l'URL, nous devrons indiquer d'une manière ou d'une autre, en l'invoquant, quel texte cliquable sera affiché dans le document.

Dans la deuxième version, le dernier argument inclut le texte cliquable. Le troisième argument existe dans le cas où nous voulons diviser une URL en deux parties, de sorte que la première partie contienne l'adresse d'accès et la deuxième partie le nom du document ou de la page spécifique que nous voulons ouvrir. Par exemple, avec l'adresse du document qui explique ce qu'est \ConTeXt\~:

\placefigure [force,here,none] [] {}
{\startDemoVW
\setupinteraction[state=start]
\useURL [WhatIsCTXa]
  [http://www.pragma-ade.com/general/manuals/what-is-context.pdf]
  []
  [What is \ConTeXt?]

\useURL [WhatIsCTXb]
  [http://www.pragma-ade.com/general/manuals]
  [what-is-context.pdf]
  [What is \ConTeXt?]

\url[WhatIsCTXa]\\
\url[WhatIsCTXb]

\from[WhatIsCTXa]
\stopDemoVW}

Dans les deux cas, nous aurons associé cette adresse au mot \MyKey{WhatIsCTX}. Si, à un moment quelconque du texte, nous voulons reproduire une URL que nous avons associée à un nom à l'aide de la commande \tex{useURL}, nous pouvons utiliser la commande \tex{url[NomAssocié]} qui insère l'URL associée à ce nom dans le document. Mais cette commande, bien qu'elle écrive l'URL, ne crée aucun lien. La commande \tex{from} insère le texte associé au lien (voir \in{section}[sec:createlinks]).

\startSmallPrint

Le format d'affichage des URL écrites à l'aide de \tex{url} n'est pas celui établi de manière générale au moyen de \tex{setupinteraction}, mais celui établi spécifiquement pour cette commande au moyen de \PlaceMacro{setupurl} \tex{setupurl}, qui nous permet de définir le style (option {\tt style}) et la couleur (option {\tt color}).

\stopSmallPrint

\stopitemize

\stopdescription


\startdescription{\tex{hyphenatedurl}}\PlaceMacro{hyphenatedurl}

Cette commande est destinée aux URL qui seront écrits dans le texte de notre document, et \ConTeXt\  doit inclure des sauts de ligne dans l'URL, si nécessaire, afin de composer correctement le paragraphe. Son format est le suivant :

\placefigure [force,here,none] [] {}
{\startDemoI
\hyphenatedurl{AdresseURL}
\stopDemoI}

Malgré le nom de la commande \PlaceMacro{hyphenatedurl} \tex{hyphenatedurl}, elle ne met pas de trait d'union dans le nom de l'URL. Ce qu'elle fait, c'est considérer que certains caractères courants dans les URL sont de bons points pour insérer un saut de ligne avant ou après eux. Nous pouvons ajouter les caractères que nous voulons à la liste des caractères pour lesquels un saut de ligne est autorisé. Nous disposons de trois commandes pour cela :

\placefigure [force,here,none] [] {}
{\startDemoI
\sethyphenatedurlnormal{caractères}
\sethyphenatedurlbefore{caractères}
\sethyphenatedurlafter{caractères}
\stopDemoI}

\PlaceMacro{sethyphenatedurlnormal}
\PlaceMacro{sethyphenatedurlbefore}
\PlaceMacro{sethyphenatedurlafter}


Ces commandes ajoutent, respectivement, les caractères qu'elles prennent comme arguments à la liste des caractères qui supportent les sauts de ligne avant et après, uniquement avant, et uniquement après.

\tex{hyphenatedurl} peut être utilisée lorsqu'il faut écrire une URL qui apparaîtra telle quelle dans le document final. Elle peut même être utilisée comme dernier argument de \tex{useURL} dans la version de cette commande où le dernier argument récupère le texte cliquable à afficher dans le document final.


Dans l'argument \tex{hyphenatedurl}, tous les caractères réservés peuvent être utilisés, sauf trois qui doivent être remplacés par des commandes :

\startitemize[packed]

\item \%{} doit être remplacé par \PlaceMacro{letterpercent} \tex{letterpercent}

\item \#{} doit être remplacé par \PlaceMacro{letterhash} \tex{letterhash}

\item \backslash{} doit être remplacé par \PlaceMacro{letterescape} \tex{letterescape} ou \PlaceMacro{letterbackslash} \tex{letterbackslash}.

\stopitemize

Chaque fois que \tex{hyphenatedurl} insère un saut de ligne, il exécute la commande \PlaceMacro{hyphenatedurlseparator} \tex{hyphenatedurlseparator}, qui, par défaut, ne fait rien. Mais si nous la redéfinissons, un caractère représentatif est inséré dans l'URL de manière similaire à ce qui se passe avec les mots normaux, où un trait d'union est inséré pour indiquer que le mot continue sur la ligne suivante. Par exemple :

\placefigure [force,here,none] [] {}
{\startDemoVW
\setupinteraction[state=start]
\useURL [WhatIsCTXa]
[http://www.pragma-ade.com/general/manuals]
[what-is-context.pdf]
[http://www.pragma-ade.com/general/manuals/what-is-context.pdf]

\useURL [WhatIsCTXb]
[http://www.pragma-ade.com/general/manuals]
[what-is-context.pdf]
[\hyphenatedurl{http://www.pragma-ade.com/general/manuals/what-is-context.pdf}]

\from[WhatIsCTXa]\\
\from[WhatIsCTXb]\\
\def\hyphenatedurlseparator{\curvearrowright}
\from[WhatIsCTXb]
\stopDemoVW}

\stopdescription

\stopsubsection

% ***  Section Commands that establish the link

% TODO Garulfo la présentation me semble améliorable et simplifiable
% Donner un exemple et ensuite expliquer ?
% ne peut on pas faire l'explication en 2 pages max ?

\startsubsection
[
reference=sec:createlinks,
title={Commandes d'insertion d'hyperlien dans le document}]

Pour établir des liens vers des URL prédéfinis à l'aide de \tex{useURL}, nous pouvons utiliser la commande \PlaceMacro{from} \tex{from}, qui se limite à établir le lien, mais n'écrit aucun texte cliquable. Le texte par défaut de \tex{useURL} sera utilisé comme texte du lien. Sa syntaxe est la suivante :

\placefigure [force,here,none] [] {}
{\startDemoI
\from[Nom] 
\stopDemoI}

où {\em Nom} est le nom précédemment associé à une URL à l'aide de \tex{useURL}.

\placefigure [force,here,none] [] {}
{\startDemoVW
\setupinteraction[state=start]
\useURL [WhatIsCTX]
  [http://www.pragma-ade.com/general/manuals]
  [what-is-context.pdf]
  [What is \ConTeXt]

\from[WhatIsCTX]
\stopDemoVW}

Pour créer des liens et les associer à un texte cliquable qui n'a pas été préalablement défini, nous disposons de la commande \PlaceMacro{goto} \tex{goto} qui est utilisée à la fois pour générer des liens internes et externes. Sa syntaxe est la suivante :

\placefigure [force,here,none] [] {}
{\startDemoI
\goto{Texte utilisé en lien}[Cible]
\stopDemoI}

où {\em Texte utilisé en lien}  est le texte à afficher dans le document final
et sur lequel un clic de souris conduira le lecteur à la cible, et {\em Cible} peut être :

\startitemize

\item Une étiquette de notre document. Dans ce cas, \tex{goto} générera le saut de la même manière que, par exemple, les commandes \tex{in} ou \tex{at} déjà examinées. Mais contrairement à ces commandes, aucune information associée à l'étiquette ne sera récupérée.

\item L'URL elle-même. Dans ce cas, il faut indiquer expressément qu'il s'agit d'une URL en écrivant la commande comme suit :

\placefigure [force,here,none] [] {}
{\startDemoI
\goto{Texte utilisé en lien}[url(URL)]
\stopDemoI}

où URL, à son tour, peut être le nom précédemment associé à une URL au moyen de \tex{useURL}, ou l'URL elle-même, auquel cas, lors de l'écriture de l'URL, nous devons nous assurer que les caractères réservés de \ConTeXt sont écrits correctement dans \ConTeXt. L'écriture de l'URL selon les règles de \ConTeXt\ n'affectera pas la fonctionnalité du lien.

\stopitemize

\stopsubsection



\stopsection


%==============================================================================

\stopcomponent

%%% TeX-master: "../introCTX_fra.tex"

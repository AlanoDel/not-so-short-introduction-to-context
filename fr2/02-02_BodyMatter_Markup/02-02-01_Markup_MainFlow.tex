\startcomponent 02-02-01_Markup_MainFlow

\environment introCTX_env_00


\ConTeXt\ offre par défaut un grand nombre d'éléments de balisage ainsi que
des outils pour en faire nos propres variations avec les commandes de type \tex{definexxx}.

Nous allons les présenter dans un ordre de force et d'ampleur de structuration et croissante,
c'est à dire en partant de simples mots mis en valeur, à listes, à tableaux, à des sections,
jusqu'à la macrostructure du document (pages liminaires, annexes...). 

La dernière section rassemble des éléments moins usités dans l'objectif
de vous faire prendre conscience de leur existence.


\startchapter
  [title=Balisage du flux principal,
  reference=cap:mkp:mainflow]

\TocChap

\component        02-02-01-01_Markup_Paragraphs
\component        02-02-01-02_Markup_WordEmphasis
\component        02-02-01-03_Markup_ParagraphsEmphasis
\component        02-02-01-04_Markup_Framed
\component        02-02-01-05_Markup_LinesRules
\component        02-02-01-06_Markup_Quotations
\component        02-02-01-07_Markup_Itemize
\component        02-02-01-08_Markup_DescriptionEnumeration
\component        02-02-01-09_Markup_Tabulate
\component        02-02-01-10_Markup_Tables
\component        02-02-01-11_Markup_ImagesAndCombinaisons
\component        02-02-01-12_Markup_Floats
\component        02-02-01-13_Markup_Sections
\component        02-02-01-14_Markup_Macrostructure
\component        02-02-01-15_Markup_Covers
\component        02-02-01-16_Markup_Maths

\component        02-02-01-50_Markup_Others

% * ===========================================================================

\stopchapter

\stopcomponent

%%% Local Variables:
%%% mode: ConTeXt
%%% mode: auto-fill
%%% coding: utf-8-unix
%%% TeX-master: "../introCTX_fra.tex"
%%% End:
%%% vim:set filetype=context tw=75 : %%%

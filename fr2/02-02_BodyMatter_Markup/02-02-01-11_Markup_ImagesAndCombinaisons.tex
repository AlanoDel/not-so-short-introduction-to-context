\startcomponent 02-02-01-11_Markup_ImagesAndCombinaisons

\environment introCTX_env_00

%==============================================================================

\startsection
  [title={Images, figures, combinaison},
  reference=sec:mkp:images]

\TocChap


\tex{externalfigure}
\tex{useexternalfigure}


% *** Subsection
\startsubsection
  [title={Insertion directe d'images}]

\PlaceMacro{externalfigure}
\PlaceMacro{useexternalfigure}

Pour insérer directement une image (pas comme un objet flottant), nous utilisons la commande \tex{externalfigure} 

\setup{externalfigure}

\startitemize

\item {\em 1} et {\em 2} peut être soit le nom du fichier contenant l'image, soit l'adresse Web d'une image sur Internet, soit un nom symbolique que nous avons précédemment associé à une image à l'aide de la commande \PlaceMacro{useexternalfigure} \tex{useexternalfigure} dont le format est similaire à celui de \tex{externalfigure} bien qu'elle prenne un premier argument avec le nom symbolique qui sera associé à l'image en question.

\item {\em 3} est un argument facultatif qui nous permet d'appliquer certaines transformations à l'image avant qu'elle ne soit insérée dans notre document. Nous examinerons cet argument de plus près dans la \in{section} [sec:configimage].

\stopitemize

Les formats d'image autorisés sont {\tt pdf, mps, jpg, png, jp2, jbig, jbig2, jb2, svg, eps, gif, tif}. \ConTeXt\ peut gérer directement huit d'entre eux, tandis que les autres ({\tt svg, eps, gif, tif}) doivent être convertis avec un outil externe avant de les ouvrir, qui change en fonction du format et doit donc être installé sur le système pour que \ConTeXt\ puisse manipuler ce type de fichiers.


\startSmallPrint

Parmi les formats pris en charge par \tex{externalfigure} figurent également certains formats vidéo. En particulier : QuickTime (extension .mov), Flash Video (extension .flv) et MPeg~4 (extension .mp4). Mais la plupart des lecteurs de PDF ne savent pas comment traiter les fichiers PDF contenant de la vidéo. Je ne peux pas me prononcer sur ce point, car je n'ai pas fait de tests.

\stopSmallPrint


Il n'est pas nécessaire d'indiquer l'extension du fichier : \ConTeXt\ recherchera un fichier avec le nom spécifié et l'une des extensions des formats d'image connus. S'il y a plusieurs candidats, c'est d'abord le format PDF qui est utilisé s'il existe, et en son absence le format MPS (graphiques générés par MetaPost). En l'absence de ces deux formats, l'ordre suivant est suivi : jpeg, png, jpeg~2000, jbig et jbig2.

\startSmallPrint

Si le format réel de l'image ne correspond pas à l'extension du fichier qui la stocke, \ConTeXt\ ne peut pas l'ouvrir à moins que nous n'indiquions le format réel de l'image à l'aide de l'option {\tt method}.

\stopSmallPrint


Si l'image n'est pas placée seule en dehors d'un paragraphe, mais qu'elle est intégrée à un paragraphe de texte, et que sa hauteur est supérieure à l'interligne, la ligne sera ajustée pour éviter que l'image ne chevauche les lignes précédentes~ :

\placefigure [force,here,none] [] {}{
\startDemoVN
Si l'image est intégrée à un paragraphe de texte, la hauteur de ligne sera ajustée pour éviter que l'image \externalfigure [cow-brown] [width=3em] ne chevauche les lignes précédentes et que l'ensemble reste lisible.
\stopDemoVN}

Par défaut, \ConTeXt\ cherche les images dans le répertoire de travail, dans son répertoire parent et dans le répertoire parent de ce répertoire. Nous pouvons indiquer l'emplacement d'un répertoire contenant les images avec lesquelles nous allons travailler en utilisant l'option {\tt directory} de la commande \tex{setupexternalfigures}, qui ajoutera ce répertoire au chemin de recherche. Si nous voulons que la recherche soit effectuée uniquement dans le répertoire d'images, nous devons également définir l'option {\tt location}. Ainsi, par exemple, pour que notre document recherche toutes les images dont nous avons besoin dans le répertoire \MyKey{img}, nous devons écrire :

\PlaceMacro{setupexternalfigures}

Par défaut, \ConTeXt\ cherche les images dans le répertoire de travail, dans son répertoire parent et dans le répertoire parent de ce dernier. Nous pouvons indiquer l'emplacement d'un répertoire contenant les images avec lesquelles nous allons travailler en utilisant l'option {\tt directory} de la commande \tex{setupexternalfigures}, qui ajoutera ce répertoire au chemin de recherche. Si nous voulons que la recherche soit effectuée uniquement dans le répertoire d'images, nous devons également définir l'option {\tt location}. Ainsi, par exemple, pour que notre document recherche toutes les images dont nous avons besoin dans le répertoire \MyKey{~/Images}, nous devons écrire :


\placefigure [force,here,none] [] {}{
\startDemoI
\setupexternalfigures [directory=~/Image]
\stopDemoI}


\startSmallPrint

Dans l'option {\tt directory} de \tex{setupexternalfigures}, nous pouvons inclure plus d'un répertoire, en les séparant par des virgules ; mais dans ce cas, nous devons mettre les répertoires entre accolades. Par exemple : \MyKey{directory=\{img, \lettertilde/images\}}.

Dans {\tt directory}, nous utilisons toujours le caractère \quote{/} comme séparateur entre les répertoires ; y compris dans Microsoft Windows dont le système d'exploitation utilise le caractère \quote{\backslash} comme séparateur de répertoires.

\stopSmallPrint

\tex{externalfigure} est également capable d'utiliser des images hébergées sur Internet. Ainsi, par exemple, l'extrait suivant insère le logo CervanTeX directement depuis Internet dans le document. Il s'agit du groupe d'utilisateurs hispanophones de \TeX\ :

\placefigure [force,here,none] [] {}{
\startDemoHN
\externalfigure
  [http://www.cervantex.es/files/cervantex/cervanTeXcolor-small.jpg]
\stopDemoHN}


\startSmallPrint

Lorsqu'un document contenant un fichier distant est compilé pour la première fois, il est téléchargé du serveur et stocké dans le répertoire de cache de \LuaTeX . Ce fichier en cache est utilisé lors des compilations suivantes. Normalement, l'image distante est téléchargée à nouveau si l'image dans le cache est plus ancienne que 1~jour. Pour modifier ce seuil, consultez le \goto{wiki \ConTeXt}[url(https://wiki.contextgarden.net/Using_Graphics)].


\stopSmallPrint

Si \ConTeXt\ ne trouve pas l'image qui devrait être insérée, aucune erreur n'est générée, mais à la place de l'image, un bloc de texte sera inséré avec des informations sur l'image qui devrait s'y trouver. La taille de ce bloc sera celle de l'image (si elle est connue de \ConTeXt\) ou, sinon, une taille standard.

\placefigure [force,here,none] [] {}{
\startDemoVN
\externalfigure 
  [image-perdueA.jpg]
\externalfigure  
  [image-perdueB.jpg]
  [height=2cm]
\stopDemoVN}


\stopsubsection

% *** Subsection
\startsubsection
  [
    reference=sec:placefigure,
    title={Insertion d'une image avec \tex{placefigure}},
  ]
\PlaceMacro{placefigure}

Les images peuvent être insérées directement. Mais il est préférable de le faire avec \tex{placefigure}. Cette commande à plusieurs implications vis à vis de \ConTeXt. Elle permet~ :

\startitemize

\item  d'indiquer que l'on insère une image qui doit être incorporée à la liste des images du document qui pourra ensuite être utilisée, si on le souhaite, pour produire un index des images.

\item d'attribuer un numéro à l'image, facilitant ainsi les références internes à celle-ci.

\item d'ajouter un titre à l'image, en créant un bloc combinant l'image et son titre, ce qui les rend indissociables.

\item de définir automatiquement l'espace blanc (horizontal et vertical) entourant l'image et nécessaire sa bonne visualisation.

\item de positionner l'image à l'endroit indiqué, en faisant {\em circuler} le texte autour d'elle si nécessaire.

\item pour convertir l'image en objet flottant si cela est possible, en tenant compte de ses spécifications de taille et d'emplacement.\footnote{Cette dernière est ma conclusion, étant donné que parmi les options de placement, il y a celles comme {\tt force} ou \Conjecture {\tt split} qui vont à l'encontre de la véritable notion d'objet flottant.}

% TODO Garulfo trouvver une réponse au point ci-dessus

\stopitemize
  

La syntaxe de cette commande est la suivante :

\placefigure [force,here,none] [] {}{
\startDemoI
\placefigure[Options] [Etiquette] {Titre} {CommandeInsererImage}
\stopDemoI}

Elle hérite de la commande \tex{placefloat}.
\setup{placefloat}


The various arguments have the following meanings:

\startitemize

\item {\em Options} sont un ensemble d'indications qui font généralement référence à l'endroit où placer l'image. Comme ces options sont les mêmes dans cette commande et dans d'autres, je les expliquerai ensemble plus tard (dans \in{section}[sec:placingobjects]). Pour l'instant, je vais utiliser l'option {\tt here} à titre d'exemple. Elle indique à \ConTeXt\ que, dans la mesure du possible, il doit placer l'image exactement à l'endroit du document où se trouve la commande qui l'insère.

\item {\em Etiquette} est une chaîne de texte permettant de se référer en interne à cet objet afin de pouvoir y faire référence (voir \in{section}[sec:references]).

\item {\em Titre} est le texte du titre à ajouter à l'image.

\item {\em CommandeInsererImage} est la commande qui insère l'image (on peut aussi juste mettre... du texte.)

\stopitemize

Par exemple~ :

\placefigure [force,here,none] [] {}{
\startDemoHW
\setupfloat[figure][location=center]
Imaginons un texte introductif.
\placefigure
  [here]
  [fig:cm2016group]
  {\ConTeXt\ Meeting Group 2016}
  {\externalfigure[https://meeting.contextgarden.net/2016/img/cm2016-group-photo.jpg]}
Avez vous vu la \in{figure}[fig:cm2016group] ?
\blank[big]
{\bfa Liste des figures}
\blank[small]
\placelistoffigures
\stopDemoHW}

Comme nous pouvons le voir dans l'exemple, en insérant l'image (ce qui, soit dit en passant, a été fait directement à partir d'une image hébergée sur Internet), il y a quelques changements par rapport à ce qui se passe en utilisant directement la commande \tex{externalfigure}. Un espace vertical est ajouté, l'image est centrée et un titre est ajouté. Il s'agit de modifications {\em externes} évidentes à première vue. D'un point de vue interne, la commande a également produit d'autres effets non moins importants :


\startitemize

\item Tout d'abord, l'image a été insérée dans la \quotation{liste d'images} que \ConTeXt\ maintient en interne pour les objets insérés dans le document. Cela signifie donc que l'image apparaîtra dans l'index des images qui peut être généré avec \tex{placelist[figure]} (voir \in{section}[sec:lists]), bien qu'il existe deux commandes spécifiques pour générer l'index d'images, à savoir
  \PlaceMacro{placelistoffigures} \tex{placelistoffigures} ou
  \PlaceMacro{complettelistoffigures} \tex{complettelistoffigures}.

\item Ensuite, l'image a été liée à l'étiquette qui a été ajoutée comme deuxième argument de la commande \tex{placefigure}, ce qui signifie qu'à partir de maintenant, nous pouvons faire des références internes à cette image en utilisant cette étiquette (voir \in{section}[sec:références]).

\item Enfin, l'image est devenue un flottant, ce qui signifie que si, pour des besoins de composition (pagination), elle devait être déplacée, \ConTeXt\ modifierait son emplacement.

\stopitemize

En fait, \tex{placefigure}, malgré son nom, n'est pas seulement utilisé pour insérer des images. Nous pouvons insérer n'importe quoi avec elle, y compris du texte. Cependant, le texte ou les autres éléments insérés dans le document avec \tex{placefigure}, seront traités {\em comme s'il s'agissait d'une image}, même s'ils ne le sont pas ; ils seront ajoutés à la liste des images gérée en interne par \ConTeXt, et nous pouvons appliquer des transformations similaires à celles que nous utilisons pour les images telles que la mise à l'échelle ou la rotation, le cadrage, etc. Ainsi, l'exemple suivant :


\placefigure [force,here,none] [] {}{
\startDemoHN
\setupfloat [figure] [location=center]
\placefigure
  [here, force]
  [fig:testtext]
  {\tex{placefigure} avec des transformations textuelles.}
  {\rotate[rotation=35]{\color[darkred]{\bfd Excellent !!!}}}
\stopDemoHN}


\stopsubsection

% *** Subsection Insertion d'images intégrées dans un bloc de texte
\startsubsection
  [title={Insertion d'images intégrées dans un bloc de texte}]

À l'exception des très petites images, qui peuvent être intégrées dans une ligne sans trop perturber l'espacement des paragraphes, les images sont généralement insérées dans un paragraphe qui ne contient qu'elles (ou dit autrement, l'image peut être considérée comme un paragraphe à part entière). Si l'image est insérée avec \tex{placefigure} et que sa taille le permet, en fonction de ce que nous avons indiqué concernant son placement (voir \in{section}[sec:placingobjects]), \ConTeXt\ permettra au texte du paragraphe précédent et des paragraphes suivants de circuler autour de l'image. Cependant, si nous voulons nous assurer qu'une certaine image ne sera pas séparée d'un certain texte, nous pouvons utiliser l'environnement {\tt figuretext} \PlaceMacro{startfiguretext} dont la syntaxe est la suivante :

\placefigure [force,here,none] [] {}{
\startDemoHN
Texte
\setupfloat [figure] [location=center]
\startfiguretext
  [left]
  [fig:testtext]
  {\tex{placefigure} avec des transformations textuelles.}
  {\rotate[rotation=20]{\color[darkred]{\bfd Excellent ce texte image!!!}}}
Texte d'accompagnement qui peut contenir beaucoup d'information selon l'inspiration de l'auteur.
\stopfiguretext
\stopDemoHN}


Les arguments de l'environnement sont exactement les mêmes que pour \tex{placefigure} et ont la même signification. Mais ici, les options ne sont plus des options de placement d'un objet flottant, mais des indications concernant l'intégration de l'image dans le paragraphe ; ainsi, par exemple, \MyKey{left} signifie ici que l'image sera placée sur la gauche tandis que le texte s'écoulera vers la droite, tandis que \MyKey{left, bottom} signifiera que l'image doit être placée sur le côté inférieur gauche du texte qui lui est associé.

Le texte écrit dans l'environnement est ce qui circulera autour de l'image.

\placefigure [force,here,none] [] {}{
\startDemoHN
Texte
\setupfloat [figure] [location=center]
\startfiguretext
  [right,low]    % or middle
  [fig:testtext]
  {\tex{placefigure} avec des transformations textuelles.}
  {\rotate[rotation=20]{\color[darkred]{\bfd Excellent ce texte image!!!}}}
Texte d'accompagnement qui peut contenir beaucoup d'information selon l'inspiration de l'auteur.
\stopfiguretext
\stopDemoHN}

\stopsubsection

% *** Subsection
\startsubsection
  [
    reference=sec:startcombination,
    title={Insertion combinée de deux ou plusieurs objets},
  ]

Pour insérer deux ou plusieurs objets différents dans le document, de telle sorte que \ConTeXt\ les garde ensemble et les traite comme un seul objet, nous disposons de l'environnement \PlaceMacro{startcombination} \tex{startcombination} dont la syntaxe est :

\placefigure [force,here,none] [] {}{
\startDemoI
\startcombination[ModeAffichage] ... \stopcombinaison
\stopDemoI}

où {\em ModeAffichage} indique comment les objets doivent être ordonnés : s'ils doivent tous être ordonnés horizontalement, {\em ModeAffichage} indique seulement le nombre d'objets à combiner. Mais si l'on veut combiner les objets sur deux ou plusieurs lignes, il faudra indiquer le numéro de l'objet par ligne, suivi du nombre de lignes, et séparer les deux numéros par le caractère *. Par exemple :


\placefigure [force,here,none] [] {}{
\startDemoHN
\useMPlibrary [dum]
\startcombination[3*2]
  {\externalfigure [dummy] [height=2cm,width=3cm]}
  {\externalfigure [dummy] [height=2cm,width=3cm]}
  {\externalfigure [dummy] [height=2cm,width=3cm]}
  {\externalfigure [dummy] [height=2cm,width=3cm]}
  {\externalfigure [dummy] [height=2cm,width=3cm]}
  {\externalfigure [dummy] [height=2cm,width=3cm]}
\stopcombination
\stopDemoHN}


Dans l'exemple précédent, les images que j'ai combinées n'existent pas en réalité, c'est pourquoi, au lieu des images, \ConTeXt\ a généré des images aléatoires (avec \tex{useMPlibrary [dum]} et \tex{externalfigure [dummy]} ).

Voyez, d'autre part, comment chaque élément à combiner dans \tex{startcombination}, est entre accolades. L'ensemble des accolades constitue une liste d'arguments.

En fait, \tex{startcombination} nous permet non seulement de connecter et d'aligner des images, mais aussi n'importe quel type de {\em boîte} comme des textes dans un environnement \tex{startframedtext}, des tableaux, etc. Pour configurer la combinaison, nous pouvons utiliser la commande \tex{setupcombination} et nous pouvons également créer des combinaisons préconfigurées en utilisant \PlaceMacro{definecombination} \tex{definecombination}.



\stopsubsection

\stopsection

%==============================================================================

\stopcomponent

%%% TeX-master: "../introCTX_fra.tex"

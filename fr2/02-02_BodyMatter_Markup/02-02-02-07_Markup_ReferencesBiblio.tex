\startcomponent 02-02-02-07_Markup_ReferencesBiblio

\environment introCTX_env_00

%==============================================================================

\startsection
  [title=Références bibliographiques,
  reference=sec:mkp:biblio]


\TocChap


\tex{usebtxdataset}
\tex{showbtxdatasetfields}
\tex{cite}
\tex{placelistofpublications}


% **   Section 7 Bibliographie

% TODO Garulfo added 

Partons ici de l'hypothèse que vous avez un fichier BiBTeX {\tt .bib} qui contient les références bibliographiques. Les principales commandes à connaitre sont~: \tex{usebtxdataset}, \tex{showbtxdatasetfields}, \tex{cite} et \tex{placelistofpublications}. Voyez plutôt~:

\PlaceMacro{usebtxdataset}
\PlaceMacro{showbtxdatasetfields}
\PlaceMacro{cite}
\PlaceMacro{placelistofpublications}

\attachment
  [Exemple de bibliographie]
  [file={introCTX_biblio.bib},
   author={bibliographie exemple}]

\placefigure [force,here,none] [] {}
{\startDemoHN%
\usebtxdataset [introCTX_biblio.bib]%
\showbtxdatasetfields%
%\showbtxdatasetcompleteness

Une première citation~: \cite[title][Hagen2017metafun]
de \cite[author][Hagen2017metafun] (\cite[Hagen2017metafun])

Une seconde citation~: \cite[title][ConTeXtExcursion]
de \cite[author][ConTeXtExcursion] (\cite[ConTeXtExcursion])

\startsubject[title=Biblio]
\placelistofpublications
\stopsubject
\stopDemoHN}


\usebtxdataset [introCTX_biblio.bib]%

Une première citation~: \cite[title][Hagen2017metafun]
de \cite[author][Hagen2017metafun] (\cite[Hagen2017metafun])

Une seconde citation~: \cite[title][ConTeXtExcursion]
de \cite[author][ConTeXtExcursion] (\cite[ConTeXtExcursion])


Cette section méritera d'être complétée. Pour le moment, vous pouvez consulter
\goto{la présentation de 2014}
[url(https://meeting.contextgarden.net/2014/talks/2014-09-09-alan-bibtex/From_BibTeX_to_ConTeXt_MKVI.pdf)] 
de Alan Braslau,
ainsi que le manuel office
\goto{Bibliographies the \ConTeXt\ way}
[url(https://pragma-ade.nl/general/manuals/mkiv-publications.pdf)] qui indique notamment comment personaliser l'affichage des informations.

\page

\startsubsubsubject[title=Biblio]
\placelistofpublications
\stopsubsubsubject



\stopsection


%==============================================================================

\stopcomponent

%%% TeX-master: "../introCTX_fra.tex"

\startcomponent 02-02-01-07_Markup_Itemize

\environment introCTX_env_00

%==============================================================================

\startsection
  [title=Listes itemize,reference=sec:itemize,]
\TocChap

\tex{startitemize}


Lorsque les informations sont présentées de manière ordonnée, elles sont plus faciles à saisir pour le lecteur. Mais si l'agencement est également perceptible visuellement, cela souligne pour le lecteur le fait que nous avons ici un texte structuré. C'est pourquoi il existe certaines {\em constructions} ou {\em mécanismes} qui tentent de mettre en évidence la disposition visuelle du texte, contribuant ainsi à sa structuration. Parmi les outils que \ConTeXt\ met à la disposition de l'auteur à cette fin, le plus important, qui fait l'objet de cette section, est l'environnement {\tt itemize} qui permet de développer ce que nous pourrions appeler des {\em listes structurées}.

Les listes sont des séquences d'{\em éléments de texte} (que j'appellerai {\em items}), chacun d'entre eux étant précédé d'un caractère qui permet de le mettre en évidence en le différenciant du reste, et que j'appellerai le \quotation{séparateur}. Le séparateur peut être un chiffre, une lettre ou un symbole. Généralement (mais pas toujours), les {\em items} sont des paragraphes et la liste est formatée de manière à assurer la  {\em visibilité} du séparateur pour chaque élément, généralement en appliquant un retrait suspendu qui le met en évidence. 
\footnote{En typographie, un retrait qui s'applique à toutes les lignes d'un paragraphe sauf la première est appelé un {\em tiret suspendu}, ce qui permet de trouver facilement le premier mot ou caractère du paragraphe.}
Dans le cas de listes imbriquées, l'indentation de chacune d'elles augmente progressivement. Le langage HTML appelle généralement les listes où le séparateur est un nombre ou un caractère qui augmente de manière séquentielle, des {\em listes ordonnées}, ce qui signifie que chaque {\em item} de la liste aura un séparateur différent qui nous permettra de nous référer à chaque élément par son numéro ou son identifiant ; et il donne le nom de {\em listes non ordonnées} à celles où le même caractère ou symbole est utilisé pour chaque élément de la liste.

\ConTeXt\ gère automatiquement la numérotation ou l'ordre alphabétique du séparateur dans les listes numérotées, ainsi que l'indentation que doivent avoir les listes imbriquées ; et, dans le cas de listes non ordonnées imbriquées, il s'occupe également de la sélection d'un caractère ou d'un symbole différent qui permet de distinguer d'un coup d'œil le niveau d'un {\em item} dans la liste en fonction du symbole qui le précède.


\startSmallPrint

Le manuel de référence dit que le niveau maximum d'imbrication dans les listes est de 4, mais je suppose que c'était le cas en 2013, lorsque le manuel a été écrit. D'après mes tests, il ne semble pas y avoir de limite à l'imbrication des listes {\em ordonnées} (dans mes tests, j'ai atteint jusqu'à 15 niveaux d'imbrication). Pour les listes non ordonnées, il ne semble pas y avoir de limite non plus, dans le sens où peu importe le nombre d'imbrications que nous incluons, aucune erreur ne sera générée ; mais, pour les listes non ordonnées, \ConTeXt\ applique seulement des symboles par défaut pour les neuf premiers niveaux d'imbrication. 

Quoi qu'il en soit, il convient de souligner qu'un nombre excessif d'imbrications de listes peut avoir l'effet inverse de celui recherché, à savoir que le lecteur se sent perdu, incapable de situer chaque élément dans la structure générale de la liste. C'est pourquoi je pense personnellement que, bien que les listes soient un outil puissant pour structurer un texte, il n'est presque jamais bon de dépasser le troisième niveau d'imbrication ; et même le troisième niveau ne devrait être utilisé que dans certains cas où nous pouvons le justifier.

\stopSmallPrint

L'outil général pour écrire des listes dans \ConTeXt\ est l'environnement \tex{itemize} dont la syntaxe est la suivante :

The general tool for writing lists in \ConTeXt\ is the \tex{itemize} environment whose syntax is as follows:

\PlaceMacro{startitemize}
\placefigure [force,here,none] [] {}
{\startDemoI
\startitemize[Options][Configuration] ... \stopitemize
\stopDemoI}

où les deux arguments sont facultatifs. Le premier permet d'utiliser des noms symboliques comme contenu auquel une signification précise a été attribuée par \ConTeXt\ ; le second argument, rarement utilisé, permet d'attribuer des valeurs spécifiques à certaines variables qui affectent le fonctionnement de l'environnement.



% *** Subsection 
\startsubsection
  [
    reference=sec:itemize_item-type,
    title={Saisie des éléments d'une liste},
  ]

En règle générale, les éléments d'une liste créée avec \tex{startitemize} sont saisis avec la commande \PlaceMacro{item}\tex{item} qui possède également une version sous forme d'environnement plus adaptée au style Mark~IV :
\PlaceMacro{startitem} \tex{startitem ... \stopitem}. Ainsi, observez l'exemple suivant :

\placefigure [force,here,none] [] {}
{\startDemoVN
\startitemize[a, packed]
\startitem Premier élément \stopitem
\startitem Second élément \stopitem
\startitem Troisième élément \stopitem
\stopitemize

\startitemize[a, packed]
\item Premier élément
\item Second élément
\item Troisième élément
\stopitemize
\stopDemoVN}

\tex{item} ou \tex{startitem} est la commande {\em générale} pour insérer un élément dans la liste. Elle s'accompagne des commandes supplémentaires suivantes pour obtenir un résultat particulier :


\startitemize[3*broad]

\sym{\PlaceMacro{head}\tex{head}} Cette commande doit être utilisée à la place de \tex{item} lorsque l'on veut éviter d'insérer un saut de page après l'élément en question. Son formatage peut être indiqué avec l'option {\tt headstyle=}  (bold par exemple).

\startSmallPrint

Une construction courante consiste à inclure une liste imbriquée ou un bloc de texte immédiatement sous un élément de liste, de sorte que l'élément de liste fonctionne, en un sens, comme le {\em titre} de la sous-liste ou du bloc de texte. Dans ce cas, il est déconseillé d'insérer un saut de page entre cet élément et les paragraphes suivants. Si nous utilisons \tex{head} au lieu de \tex{item} pour saisir ces éléments, \ConTeXt\ {\em s'efforcera} (dans la mesure du possible) de ne pas séparer cet élément du bloc suivant.

\stopSmallPrint

\sym{\PlaceMacro{nop}\tex{nop}} annule l'affichage du séparateur. 

\sym{\PlaceMacro{sym}\tex{sym}} La commande \type{\sym{Texte}} permet de saisir un élément dans lequel le texte utilisé comme argument de \tex{sym} est utilisé comme {\em séparateur}, et non comme nombre ou symbole. Cette liste, par exemple, est construite avec des éléments saisis au moyen de \tex{sym} au lieu de \tex{item}. 

\placefigure [force,here,none] [] {}
{\startDemoVN
\startitemize[a, packed]
\item Premier élément
\sym{x} Second élément
\nop Troisième élément
\stopitemize
\stopDemoVN}

\sym{\PlaceMacro{sub}\tex{sub}} Cette commande, qui ne doit être utilisée que dans les listes ordonnées (où chaque élément est précédé d'un numéro ou d'une lettre dans l'ordre alphabétique), fait en sorte que l'élément saisi avec elle conserve le numéro de l'élément précédent et, afin d'indiquer que le numéro est répété et qu'il ne s'agit pas d'une erreur, le signe \quote{+} est imprimé à gauche. Cela peut être utile si l'on se réfère à une liste antérieure pour laquelle on suggère des modifications mais où, pour des raisons de clarté, il convient de conserver la numérotation de la liste originale.

\sym{\PlaceMacro{mar}\tex{mar}} Cette commande conserve la numérotation des éléments, mais ajoute une lettre ou un caractère dans la marge (qui lui est passé en argument, entre crochets). Je ne suis pas sûr de son utilité.

% TODO Garulfo vérifier la compréhension de mar et sub 

\placefigure [force,here,none] [] {}
{\startDemoVN
\startnarrower[left]
\startitemize[n, packed]
\item Premier élément
\mar{o} Second élément (un o en marge)
\sub  Second élément
\item Troisième élément
\stopitemize
\stopnarrower
\stopDemoVN}

\stopitemize

Il existe deux commandes supplémentaires pour la saisie d'éléments, dont la combinaison produit des effets très {\em intéressants} et, si je peux me permettre, je pense qu'il est préférable de les expliquer par un exemple. \PlaceMacro{ran} \tex{ran} (abréviation de {\em gamme}) et \PlaceMacro{its} \tex{its}, abréviation de {\em items}. La première prend un argument (entre crochets) et la seconde répète le symbole utilisé comme séparateur dans la liste un nombre x de fois (par défaut 4 fois, mais nous pouvons modifier cela en utilisant l'option {\tt items}). L'exemple suivant montre comment ces deux commandes peuvent fonctionner ensemble pour créer une liste qui imite un questionnaire :


\placefigure [force,here,none] [] {}
{\startDemoHN
Après avoir lu l'introduction suivante, répondez aux questions suivantes~:
\startitemize[5, packed][width=8em, distance=2em, items=5]
\ran{No \hss Yes} % \hss = un espace infiniment extensible et écrasable
\its Je n'utiliserai jamais ConTeXt, c'est trop difficile.
\its Je ne l'utiliserai que pour écrire de gros livres.
\its Je l'utiliserai toujours.
\its Je l'aime tellement que j'appellerai mon prochain enfant \quotation{Hans}, en hommage à Hans Hagen.
\stopitemize
\stopDemoHN}

\stopsubsection


% *** Subsection 
\startsubsection
  [
    reference=sec:items command,
    title={Listes simples à l'aide de la commande \tex{items}.},
  ]

\PlaceMacro{items}

Une alternative à l'environnement {\tt itemize} pour les listes non numérotées très simples, où les éléments ne sont pas trop gros, est la commande \tex{items} dont la syntaxe est :

\placefigure [force,here,none] [] {}
{\startDemoI
\items[Configuration]{Item 1, Item 2, ..., item n}
\stopDemoI}

Les différents éléments de la liste sont séparés les uns des autres par des virgules. Par exemple :

\placefigure [force,here,none] [] {}
{\startDemoVN
Les fichiers graphiques peuvent avoir, entre autres, les extensions suivantes :
\items{png, jpg, tiff, bmp}
\stopDemoVN}

Les options de configuration prises en charge par cette commande sont un sous-ensemble de celles de la commande {\tt itemize}, à l'exception de deux options spécifiques à cette commande : 

\startitemize
  
\item {\tt symbol} : cette option détermine le type de liste qui sera généré. Elle prend en charge les mêmes valeurs que celles utilisées pour {\tt itemize} pour sélectionner un type de liste.
 
\item {\tt n} : cette option indique à partir de quel numéro d'élément il y aura un séparateur.

\stopitemize

\stopsubsection

% *** Subsection 
\startsubsection
  [
    reference=sec:setupitemize,
    title={Prédétermination du comportement des listes et création de nos propres types de listes},
  ]
  
Dans les sections précédentes, nous avons vu comment indiquer quel type de liste nous voulons et quelles caractéristiques elle doit avoir. Mais faire cela à chaque fois qu'une liste est appelée est inefficace et produira généralement un document incohérent dans lequel chaque liste a sa propre apparence, mais sans que les différentes apparences ne répondent à aucun critère.

Résultat préférable pour cela :

\startitemize

\item Prédéterminer le comportement {\em normal} de {\tt itemize} et \tex{items} dans le préambule du document.

\item Créer nos propres listes personnalisées. Par exemple : une liste numérotée alphabétiquement que nous voulons appeler {\em ListAlpha}, une liste numérotée avec des chiffres romains ({\em ListRoman}), etc.
 
\stopitemize

Nous réalisons la première avec les commandes \tex{setupitemize} et \tex{setupitems}. La seconde nécessite l'utilisation de la commande \PlaceMacro{defineitemgroup} \tex{defineitemgroup}, ou \PlaceMacro{defineitems} \tex{defineitems}. La première créera un environnement de liste similaire à {\tt itemize} et la seconde une commande similaire à {\tt items}.

\stopsubsection

\stopsection

%==============================================================================

\stopcomponent

%%% TeX-master: "../introCTX_fra.tex"

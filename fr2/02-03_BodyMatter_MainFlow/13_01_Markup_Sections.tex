\startcomponent 02-02-01-13_Markup_Sections

\environment introCTX_env_00

%==============================================================================

\startsection
  [title=Sections,    
  reference=cap:structure,]


\TocChap

\tex{startpart}
\tex{startchapter}
\tex{startsection}
\tex{startsubsection}
\tex{startsubsubsection}
\tex{startsubsubsubsection}


\startsubsection
  [title=Les divisions structurelles d'un document]


À l'exception des textes très courts (comme une lettre, par exemple), un document est généralement structuré en blocs ou en groupements de textes qui suivent généralement un ordre hiérarchique. Il n'y a pas de manière standard de nommer ces blocs : dans les romans, par exemple, les divisions structurelles sont généralement appelées \quotation{chapitres} bien que certains -- les plus longs -- aient des blocs plus grands généralement appelés \quotation{parties} qui regroupent un certain nombre de chapitres.  Les œuvres théâtrales font la distinction entre les \quotation{actes} et les \quotation{scènes}. Les manuels universitaires sont divisés (parfois) en \quotation{parties} et \quotation{leçons}, \quotation{sujets} ou \quotation{chapitres} qui, à leur tour, ont souvent des divisions internes également ; le même type de divisions hiérarchiques complexes existe souvent dans d'autres documents universitaires ou techniques (tels que des textes comme le présent texte consacré à l'explication d'un programme ou d'un système informatique. Même les lois sont structurées en \quotation{livres}, (les plus longs et les plus complexes, comme les Codes), \quotation{titres}, \quotation{chapitres}, \quotation{sections}, \quotation{sous-sections}. Les documents scientifiques et techniques peuvent également atteindre jusqu'à six, sept ou même parfois huit niveaux de profondeur.


Ce chapitre se concentre sur l'analyse du mécanisme que propose \ConTeXt\ pour mettre en oeuvre ces divisions structurelles. Je les désignerai par le terme général de \quotation{sections}.



\startSmallPrint

Il n'existe pas de terme clair qui nous permette de nous référer de manière générique à tous ces types de divisions structurelles. Le terme \quotation{section}, pour lequel j'ai opté, se concentre sur la division structurelle plutôt que sur autre chose, bien qu'un inconvénient soit que l'une des divisions structurelles prédéterminées de \ConTeXt\ soit justement appelée une \quotation{section}. J'espère que cela ne créera pas de confusion, car je pense qu'il sera assez facile de déterminer à partir du contexte si nous parlons de section en tant que référence générique et globale aux divisions structurelles, ou d'une division spécifique que \ConTeXt\ appelle une section.


\stopSmallPrint

Chaque \quotation{section} (de manière générique) implique :

\startitemize

\item Une {\em division structurelle du document} raisonnablement grande d'un document qui peut, à son tour, inclure d'autres divisions de niveau inférieur. Dans cette perspective, les {\em sections} impliquent des blocs de texte avec une relation hiérarchique entre eux. Du point de vue de ses sections, le document dans son ensemble peut être considéré comme un arbre. Le document {\em en soi} est le tronc, chacun de ses chapitres une branche, qui à son tour peut avoir des rameaux qui peuvent aussi se subdiviser et ainsi de suite.

  Il est très important d'avoir une structure claire pour que le document puisse être lu et compris. Cette tâche incombe toutefois à l'auteur, et non au compositeur. Et bien qu'il ne revienne pas à \ConTeXt\ de faire de nous de meilleurs auteurs que nous ne le sommes, la gamme complète de commandes de section qu'il inclut, où la hiérarchie entre elles est très claire, pourrait nous aider à écrire des documents mieux structurés.

  \item Un {\em nom de structure} que nous pourrions appeler son \quotation{titre}   ou \quotation{label}. Ce nom de structure est affiché~:


  \startitemize

    \item Toujours (ou presque toujours) à l'endroit du document où commence la division structurelle.

    \item Parfois aussi dans la table des matières, dans l'en-tête ou le pied de page des pages occupées par la section en question. 


  \stopitemize

  \ConTeXt\ nous permet d'automatiser toutes ces tâches de telle sorte que les caractéristiques de formatage avec lesquelles le titre d'une unité structurelle doit être affiché (où que ce soit et notamment dans la table des matières, ou dans les en-têtes ou les pieds de page) ne doivent être indiquées qu'une seule fois. Pour ce faire, \ConTeXt\ a seulement besoin de savoir où commence et finit chaque unité structurelle, comment elle s'appelle et à quel niveau hiérarchique elle se situe.

\stopitemize

\stopsubsection

% * Section Section types et hierarchies
\startsubsection
  [
    reference=sec:sectiontypes,
    title=Types et hiérarchie des sections,
  ]

\ConTeXt\ fait la distinction entre les sections {\em numérotées} et {\em non numérotées}. Les premières, comme leur nom l'indique, sont numérotées automatiquement et envoyées à la table des matières, ainsi que, parfois, aux en-têtes et/ou pieds de page.

\ConTeXt\ a des commandes de section prédéfinies et hiérarchisées qui se trouvent dans la \in{table}[tbl:sectioned].


\placetable
  [here]
  [tbl:sectioned]
  {Section commands in \ConTeXt}
{
\starttabulate[|l|l|l|]
\HL
\NC {\bf Niveau}
\NC {\bf Sections numérotées}
\NC {\bf Sections non numérotées}
\NR
\HL
\NC 1
\NC \tex{part}    \PlaceMacro{part}\PlaceMacro{startpart}
\NC --
\NR
\NC 2
\NC \tex{chapter} \PlaceMacro{chapter}\PlaceMacro{startchapter}
\NC \tex{title}   \PlaceMacro{title}\PlaceMacro{starttitle}
\NR
\NC 3
\NC \tex{section} \PlaceMacro{section}\PlaceMacro{startsection}
\NC \tex{subject} \PlaceMacro{subject}\PlaceMacro{startsubject}
\NR
\NC 4
\NC \tex{subsection} \PlaceMacro{subsection}\PlaceMacro{startsubsection}
\NC \tex{subsubject} \PlaceMacro{subsubject}\PlaceMacro{startsubsubsubject}
\NR
\NC 5
\NC \tex{subsubsection}\PlaceMacro{subsubsection}\PlaceMacro{startsubsubsection}
\NC \tex{subsubsubject}\PlaceMacro{subsubsubject}\PlaceMacro{startsubsubsubject}
\NR
\NC 6
\NC \tex{subsubsubsection}\PlaceMacro{subsubsubsection}\PlaceMacro{startsubsubsubsection}
\NC \tex{subsubsubsubject}\PlaceMacro{subsubsubsubject}\PlaceMacro{startsubsubsubsubject}
\NR
\NC ...
\NC ...
\NC ...
\NR
\HL
\stoptabulate
}

En ce qui concerne les sections prédéfinies, les précisions suivantes doivent être apportées~:

\startitemize

\item Dans le \in{tableau}[tbl:sectioned], les commandes de section sont présentées sous leur forme traditionnelle. Mais nous verrons tout de suite qu'elles peuvent également être utilisées comme des {\em environnements} (\tex{startchapter ... \stopchapter}, par exemple) et que c'est l'approche qui est réellement recommandée.

\item Le tableau ne contient que les 6 premiers niveaux de section. Dans mes tests, cependant, j'ai trouvé jusqu'à 12 niveaux : Après \tex{subsubsubsection} vient \tex{subsubsubsubsection}, et ainsi de suite jusqu'à \tex{subsubsubsubsubsubsubsubsubsubsection}, ou \tex{subsubsubsubsubsubsubsubsubsubsubsubject}.

\startSmallPrint

Mais il ne faut pas oublier que les niveaux inférieurs (trop profonds) indiqués ci-dessus ne sont guère susceptibles d'améliorer la compréhension d'un texte ! Tout d'abord, nous risquons d'avoir de grandes sections traitant inévitablement de plusieurs sujets, ce qui rendra difficile pour le lecteur d'en {\em saisir} le contenu. En outre, si l'on approfondit excessivement les niveaux, le lecteur risque de perdre le sens global du texte, et l'effet produit est celui d'une fragmentation excessive du matériel concerné. Je crois savoir qu'en général, quatre niveaux sont suffisants ; très occasionnellement, il peut être nécessaire d'aller jusqu'à six ou sept niveaux, mais une plus grande profondeur est rarement une bonne idée.

Du point de vue de l'écriture du fichier source, le fait que la création de sous-niveaux supplémentaires signifie l'ajout d'une autre \quotation{sub} au niveau précédent peut rendre le fichier source presque illisible : ce n'est pas une blague d'essayer de déterminer le niveau d'une commande nommée \quotation{subsubsubsubsubsection} puisque je dois compter toutes les \quotation{subs} ! Mon conseil est donc que si nous avons vraiment besoin de tant de niveaux de profondeur, à partir du cinquième niveau (sous-sous-section), nous ferions mieux de définir nos propres commandes de section (voir \in{section}[sec:definehead]) en leur donnant des noms plus clairs que les noms prédéfinis.

\stopSmallPrint

  \item Le niveau de section le plus élevé (\tex{part}) n'existe que pour les titres numérotés et a la particularité que le titre de la partie n'est pas imprimé (par défaut, mais cela peut être modifié). Cependant, même si le titre n'est pas imprimé, une page blanche est introduite (sur laquelle on peut supposer que le titre est imprimé une fois que l'utilisateur a reconfiguré la commande) et la numérotation de la {\em partie} est prise en compte pour calculer la numérotation des chapitres et autres sections.


  \startSmallPrint

La raison pour laquelle la version par défaut de \tex{part} n'imprime rien est que, selon le wiki \ConTeXt\, presque toujours le titre à ce niveau nécessite une mise en page spécifique ; et bien que cela soit vrai, cela ne me semble pas une raison suffisante, puisque, dans la pratique, les chapitres et les sections sont aussi souvent redéfinis, et le fait que les parties n'impriment rien oblige l'utilisateur novice à {\em plonger} dans la documentation pour voir ce qui ne va pas.

    
  \stopSmallPrint

  \item Bien que le premier niveau de sectionnement soit la \quotation{part}, ceci n'est que théorique et abstrait. Dans un document spécifique, le premier niveau de sectionnement sera celui qui correspond à la première commande de sectionnement du document. C'est-à-dire que dans un document qui ne comprend pas de parties mais des chapitres, le chapitre sera le premier niveau.  Mais si le document ne comprend pas non plus de chapitres, mais uniquement des sections, la hiérarchie de ce document commencera par les sections.

\stopitemize

\stopsubsection

% * Section Syntaxe commune

\startsubsection
  [
    reference=sec:sectionsyntax,
    title=Syntaxe commune des commandes liées aux sections,
  ]

Toutes les commandes de section, y compris les niveaux créés par l'utilisateur (voir \in{section}[sec:definehead]), permettent les formes alternatives de syntaxe suivantes (si, par exemple, nous utilisons le niveau \MyKey{section}) :

\placefigure [force,here,none] [] {}{
\startDemoI
\section [Label] {Title}
\section [Options]
\startsection [Options] [Variables] ... \stopsection
\stopDemoI}


Dans les trois méthodes ci-dessus, les arguments entre crochets sont facultatifs et peuvent être omis. Nous les examinerons séparément, mais il convient tout d'abord de préciser que dans Mark~IV, c'est la troisième de ces trois méthodes qui est recommandée.


\startitemize

  \item Dans la première forme syntaxique, que l'on pourrait appeler la \quotation{\em historique}, la commande prend deux arguments, l'un facultatif entre crochets, l'autre obligatoire entre accolades. L'argument facultatif sert à associer la commande à une étiquette qui sera utilisée pour les références internes (voir \in{section}[sec:références]). L'argument obligatoire entre crochets est le titre de la section.

  \item Les deux autres formes de syntaxe sont plutôt du style de \ConTeXt\ : tout ce que la commande doit savoir est communiqué par des valeurs et des options introduites entre crochets.


  \startSmallPrint

Rappelez-vous que dans \in{sections}[sec:command scope] et \in{}[sec:command options] j'ai dit que dans \ConTeXt, la portée de la commande est indiquée entre crochets, et ses options entre crochets. Mais si l'on y réfléchit, le titre d'une commande de section particulière n'est pas le champ d'application de celle-ci, donc pour être cohérent avec la syntaxe générale, il ne devrait pas être introduit entre crochets, mais comme une option.  \ConTeXt\ permet cette exception car il s'agit de la façon historique de faire les choses dans \TeX, mais il fournit les formes alternatives de syntaxe qui sont plus cohérentes avec sa conception générale. 

  \stopSmallPrint

  Les options sont du type affectation de valeur (OptionName=Value), et sont les suivantes~:

  \startitemize

  \item {\tt\bf reference} : étiquette, ou référence, pour les références croisées.

  \item {\tt\bf title} : titre de la section qui sera utilisé dans le corps du document. 

  \item {\tt\bf list} : Le titre de la section qui sera utilisé dans la table des matières.

  \item {\tt\bf marking} : Le titre de la section qui sera utilisé dans les en-têtes ou les pieds de page.

  \item {\tt\bf bookmark} : Le titre de la section qui sera utilisé en {\em signet} dans le fichier PDF.

  \item {\tt\bf ownnumber} : Cette option est utilisée dans le cas d'une section qui n'est pas automatiquement numérotée ; dans ce cas, cette option prendra le numéro attribué à la section en question.

  \stopitemize

Bien entendu, les options \MyKey{list}, \MyKey{marking} et \MyKey{bookmark} ne doivent être utilisées que si nous voulons utiliser un titre différent pour remplacer le titre principal défini avec l'option \MyKey{title} . Ceci est très utile, par exemple, lorsque le titre est trop long pour l'en-tête ; bien que pour y parvenir, nous puissions également utiliser l'option \PlaceMacro{nomarking} \tex{nomarking} et \PlaceMacro{nolist} \tex{nolist} (quelque chose de très similaire). D'autre part, nous devons garder à l'esprit que si le texte du titre (l'option \quotation{title}) comprend des virgules, il devra être placé entre accolades, à la fois le texte complet et la virgule, afin que \ConTeXt\ sache que la virgule fait partie du titre. Il en va de même pour les options : \quotation{list}, \quotation{marking} et \quotation{bookmark}. Par conséquent, pour ne pas avoir à surveiller s'il y a ou non des virgules dans le titre, je pense que c'est une bonne idée de prendre l'habitude de toujours enfermer la valeur de l'une de ces options entre des accolades.

\stopitemize


Ainsi, par exemple, les lignes suivantes créeront un chapitre intitulé \quotation{Un Chapitre de test} associé à l'étiquette \quotation{chap:test} pour les références croisées, tandis que l'en-tête sera \quotation{Chapitre test} au lieu de \quotation{Un Chapitre de test}. 


\placefigure [force,here,none] [] {}{
\startDemoI
\chapter
  [title={Un Chapitre de test},
   reference={chap:test},
   marking={Chapitre test}]
\stopDemoI}


La syntaxe \tex{startSectionType} transforme la section en un {\em environnement}. Elle est plus cohérente avec le fait que, comme je l'ai dit au début, en arrière-plan, chaque section est un bloc de texte différencié, bien que \ConTeXt, par défaut, ne considère pas les {\em environnements} générés par les commandes de section comme des {\em groupes}. Néanmoins, cette procédure est celle que Mark~IV recommande, probablement parce que cette façon d'établir les sections nous oblige à indiquer expressément où commence et finit chaque section, ce qui facilite la cohérence de la structure et offre très probablement un meilleur support pour les sorties XML et EPUB. En fait, pour la sortie XML, c'est essentiel.

Lorsque nous utilisons \tex{startsection}, une ou plusieurs variables sont autorisées comme arguments entre crochets. Leur valeur peut ensuite être utilisée ultérieurement à d'autres endroits du document grâce à la commande \PlaceMacro{namedstructurevariable} \tex{structureuservariable}.

% TODO Garulfo modification :namedstructurevariable vs structureuservariable

\placefigure [force,here,none] [] {}{
\startDemoVN
\startsection[title={Mon joli titre}]
Mon texte dans 
\namedstructurevariable{section}{title}
\stopsection
\stopDemoVN}


\startSmallPrint

Le fait, pour l'utilisateur, de disposer ou accéder à des variables permet des utilisations très avancées de \ConTeXt\ en raison du fait que des décisions peuvent être prises concernant la compilation ou non d'un fragment, ou de quelle manière le faire, ou avec quel modèle en fonction de la valeur d'une variable particulière. Ces utilitaires \ConTeXt, cependant, dépassent le cadre du matériel que je souhaite traiter dans cette introduction.

\stopSmallPrint

\stopsubsection

% * Section Définir une nouvelle commande

\startsubsection
  [
    reference=sec:definehead,
    title=Définir de nouvelles commandes de section,
  ]
\PlaceMacro{definehead}


Nous pouvons définir nos propres commandes de section avec \tex{definehead} dont la syntaxe est~:

\placefigure [force,here,none] [] {}{
\startDemoI
\definehead[NomCommande][Modèle][Configuration]
\stopDemoI}

où

\startitemize

  \item {\bf NomCommande} représente le nom que portera la nouvelle commande de section.

  \item {\bf Modèle} est le nom d'une commande de section existante qui sera utilisée comme modèle à partir duquel la nouvelle commande héritera initialement de toutes ses caractéristiques.

  \startSmallPrint

En fait, la nouvelle commande hérite du modèle bien plus que ses caractéristiques initiales : elle devient une sorte d'instance personnalisée du modèle, mais partage avec lui, par exemple, le compteur interne qui contrôle la numérotation.

  \stopSmallPrint

 \item {\bf Configuration} est la configuration personnalisée de notre nouvelle commande. Ici, nous pouvons utiliser exactement les mêmes options que dans \tex{setuphead}.

\stopitemize


Il n'est pas nécessaire de configurer la nouvelle commande au moment de sa création. Cela peut être fait plus tard avec \tex{setuphead} et, en fait, dans les exemples donnés dans les manuels \ConTeXt\ et son wiki, cela semble être la manière habituelle.


\stopsubsection



\stopsection
%==============================================================================

\stopcomponent

%%% TeX-master: "../introCTX_fra.tex"

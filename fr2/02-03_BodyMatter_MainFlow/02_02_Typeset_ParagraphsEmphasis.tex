\startcomponent 02-04-01-03_Typeset_ParagraphsEmphasis

\environment introCTX_env_00

%==============================================================================

\startsection
  [title=ParagraphsEmphasis,
  reference=sec:typset:emphasep]


\TocChap

% *** Subsection Narrower 

\startsubsection
  [title=Emphase de paragraphe par mise en retrait]

\ConTeXt\ propose un environnement \MyKey{narrower}.


\PlaceMacro{startnarrower}
\placefigure [force,here,none] [] {}{
\startDemoI
\startnarrower [Options] ... \stopnarrower
\stopDemoI}

où {\em Options} peut être~:


\startitemize

\item {\tt\bf left} : met en retrait la marge de gauche.

\item {\tt\bf Num*gauche} : met en retrait la marge de gauche, en multipliant le retrait {\em normal} par {\em Num}. (par exemple, {\tt 2*left}).

\item {\tt\bf right} : met en retrait la marge de droite.

\item {\tt\bf Num*right} : met en retrait la marge de droite, en multipliant le retrait {\em normal} par {\em Num}. (par exemple, {\tt 2*right}).

\item {\tt\bf middle} : met en retrait les deux marges. Il s'agit de la valeur par défaut.

\item {\tt\bf Num*middle} : met en retrait les deux marges, en multipliant le retrait {\em normal} par {\em Num}.

\stopitemize

Lors de l'explication des options, j'ai mentionné l'indentation  {\em normale}, qui fait référence à la quantité d'indentation gauche et droite que \MyKey{narrower} applique par défaut. Cette quantité peut être configurée avec \PlaceMacro{setupnarrower} \tex{setupnarrower} qui permet les options de configuration suivantes :


\startitemize[packed]


  \item {\tt\bf left} : montant de l'indentation à appliquer à la marge gauche.

  \item {\tt\bf right} : montant du retrait à appliquer à la marge de droite.

  \item {\tt\bf middle} : montant du retrait à appliquer aux deux marges.

  \item {\tt\bf before} : commande à exécuter avant d'entrer dans l'environnement.

  \item {\tt\bf after} : commande à exécuter après avoir existé dans l'environnement.

\stopitemize

Si nous voulons utiliser différentes configurations de l'environnement plus étroit dans notre document, nous pouvons attribuer un nom différent à chacune d'entre elles avec \PlaceMacro{definenarrower} \type{\definenarrower [Name] [Configuration]} où {\em Name} est le nom lié à cette configuration et où {\em Configuration} autorise les mêmes valeurs que \tex{setupnarrower}.

\PlaceMacro{startnarrower}
\placefigure [force,here,none] [] {}{
\startDemoHN%
\definenarrower[ConfigA][left=1cm,right=3cm,default={right,left}]
\definenarrower[ConfigB][left=3cm,right=1cm,default={right,left}]
\definenarrower[ConfigC][middle=2cm,default=middle]

Ceci est un petit texte sans effet de narrower.

\startConfigA
Ceci est un petit texte pour visualiser l'effet de narrower en ConfigA. Ceci est un petit texte pour visualiser l'effet de narrower en ConfigA. Ceci est un petit texte pour visualiser l'effet de narrower en ConfigA.
\stopConfigA

Ceci est un petit texte sans effet de narrower.

\startConfigB
Ceci est un petit texte pour visualiser l'effet de narrower en ConfigB. Ceci est un petit texte pour visualiser l'effet de narrower en ConfigB. Ceci est un petit texte pour visualiser l'effet de narrower en ConfigB.
\stopConfigB

Ceci est un petit texte sans effet de narrower.

\startConfigC
Ceci est un petit texte pour visualiser l'effet de narrower en ConfigB. Ceci est un petit texte pour visualiser l'effet de narrower en ConfigB. Ceci est un petit texte pour visualiser l'effet de narrower en ConfigB.
\stopConfigC
\stopDemoHN}


\stopsubsection


\stopsection

%==============================================================================

\stopcomponent

%%% TeX-master: "../introCTX_fra.tex"

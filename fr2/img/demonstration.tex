
\definepapersize[sheet][width=65mm,height=60mm]
\setuppapersize[sheet]

\setuplayout[backspace=12mm,
    width=53mm,
    margindistance=1mm,
    rightmargin=0mm,
    headerdistance=1mm,
    footerdistance=1mm,
    topspace=0.5mm,
    backspace=10mm,
    margin=8mm,
    header=4mm,
    footer=4mm,
    height=59mm]

%\showframe

\setuppagenumbering[location=none]

%------------------------------------------------------------------------------         
\setupbodyfont[pagella]    
\setupbodyfont[8pt]


%\setupinterlinespace[2.2ex]
\setupwhitespace[medium]

\mainlanguage[fr]
\language[fr]
\setcharacterspacing[frenchpunctuation] 

\setuptolerance[strict]               % how tolerant vs ‘ugly’ stretched spaces

\setupalign[hz,                                         % hz for font expansion
            hanging,                                   % hanging for protrusion
            lesshyphenated,                      % two instructions added by j.
            verytolerant,                       % to eliminate most hyphenation
            stretch]                 % and get better interword spacing as well

%------------------------------------------------------------------------------

\usecolors[x11]

\definecolor [orange]        [.6(orange)]  
\definecolor [maincolor]     [orange]
\definecolor [purple]        [.5(darkblue)]

%------------------------------------------------------------------------------

\definehighlight [importantA]       [style=\underbar,color=darkred]
\definehighlight [importantB]       [style={\bf\cap},color=darkred]
\definehighlight [importantC]       [style=\bf,color=darkred]
\definehighlight [importantD]       [style=\bfc,color=darkred]

%------------------------------------------------------------------------------

\definetext[ChapterFirstPage]
           [footer]
           [pagenumber]

           
\setuphead [part]
  [placehead=yes,
    bodypartlabel=part,
    before={\setupbackgrounds[page][background=color,backgroundcolor=purple]\strut\vfill},
    alternative=middle,
    after={\strut\vfill\page\setupbackgrounds[page][background=color,backgroundcolor=]},
%    conversion=I,
    style={\switchtobodyfont[20pt]\bf},
    color=white,
    header=high,
    footer=ChapterFirstPage,
  ]

\setuphead[section]
  [page=yes,
   align={flushleft, nothyphenated, verytolerant},
   color=maincolor,
   style=\bfb]
\setuphead[subsection]  [style=]

%------------------------------------------------------------------------------

\setupnarrower[middle=0.75cm]

%-------------------------------------------------------------------------------

\setupinteraction[state=start]

%------------------------------------------------------------------------------

\startuseMPgraphic{myattachmentsymbol}  % #1 - Draw a symbol, here a paperclip!
path pa;
numeric u; u=0.5mm;
numeric Ra; Ra=u/2;  numeric Rb; Rb=2.2Ra;  numeric Rc; Rc=1.5Rb;
numeric La; La=3.5u; numeric Lb; Lb=-La/10; numeric Lc; Lc=1.2*La;

pair a[];
a0=(0,0);
a1=(La,0);        a2=(La+Ra,-Ra);         a3=(La,-2Ra);
a4=(Lb,-2Ra);     a5=(Lb-Rb,-2Ra+Rb);     a6=(Lb,-2Ra+2Rb);
a7=(Lc,-2Ra+2Rb); a8=(Lc+Rc,-2Ra+2Rb-Rc); a9=(Lc,-2Ra+2Rb-2Rc);
a10=(1u,-2Ra+2Rb-2Rc);

pa := (a0--a1..a2..a3--a4..a5..a6--a7..a8..a9--a10) rotated 45;

path pb;
pb := fullcircle scaled (2.8*(Lc-Lb-Rb)) shifted (center(pa)) ;
fill pb withcolor \MPcolor{darkred};
pickup pencircle scaled (1.3*(Rb-Ra))   ;
draw pa  withcolor white;
\stopuseMPgraphic

%------------------------------------------------------------------------------

\definesymbol                             % #2 - Define the drawing as a symbol
  [myattachmentsym]
  [\useMPgraphic{myattachmentsymbol}]

%------------------------------------------------------------------------------

\setupattachments                          % #3 - Use the symbol for attachment
  [symbol=myattachmentsym]

%------------------------------------------------------------------------------

\startbuffer
Ceci est un petit texte pour illustrer les différentes possibilités  \ConTeXt\ de balisage et de composition.
\stopbuffer

%-------------------------------------------------------------------------------

\definemakeup[custom][align=middle]

%-------------------------------------------------------------------------------

\setupheadertexts[]
%\setupfootertexts[margin][pagenumber][]

\startsetups[showframe]
\setupbackgrounds [header,text,footer] [leftmargin,text,rightmargin] 
[frame=on,framecolor=darkred]
\stopsetups

\startsetups[showheaderfooter]
\setupheadertexts[{\color[darkred]{Texte d'en-tête}}][{\color[darkred]{ici}}]
\setupfootertexts[{\color[darkred]{Texte de pied de page}}][{\color[darkred]{et là}}]
\setupfootertexts[margin][{\hfill\color[darkred]{\pagenumber}\hfill}][]
\stopsetups


\startsetups[hide]
   \setupbackgrounds [header,text,footer] [leftmargin,text,rightmargin] 
[frame=off]
\setupheadertexts[]
\setupheadertexts[][]
\setupfootertexts[][]
\setupfootertexts[margin][][]
\stopsetups
%-------------------------------------------------------------------------------



\starttext

%===============================================================================

\startfrontmatter

%-------------------------------------------------------------------------------

\setupbackgrounds[page][background=color,backgroundcolor=darkred]
\startmakeup[custom]
\switchtobodyfont[24pt]\bf\color[white]{Couverture}
\stopmakeup
\setupbackgrounds[page][backgroundcolor=ivory2]

%-------------------------------------------------------------------------------

\startsection[title=Introduction]
\index{introduction}
Un petit texte introductif avec un lien non visible qui sera utilisé plus tard\reference[MaRef]{référence}.
\stopsection



%-------------------------------------------------------------------------------

\startsection[title=Table des matières]

La table des matière est présentée \at{page}[toc].

\stopsection

\stopfrontmatter

%===============================================================================

\startbodymatter


\startpart [title={Composition d'ensemble}] 

%-------------------------------------------------------------------------------


\startsection[title={Mise en page}]
\setup[showframe]                         % <== pour ne plus visualiser le cadre
Il s'agit de définir la place et le positionnement des différentes zones de la page.
\stopsection

\startsection[title={En-tête, pied et numérotation de page}]
\setup[hide]
\setup[showheaderfooter]

\getbuffer \par \getbuffer

\stopsection

%-------------------------------------------------------------------------------

\startsection[title=Paragraphes]
\setup[hide]                              % <== pour ne plus visualiser le cadre
\index{paragraphes}
\getbuffer \par
\color[darkred]{\getbuffer} \par
\getbuffer
\stopsection

%-------------------------------------------------------------------------------

\startsection[title=Polices]
\definefontfeature
  [mesfeaturesA]
  [mode=node, 
   language=dflt, 
   protrusion=quality,  % for protrusion (dans les marges)
   expansion=quality,   % for expansions (expansion des lettres)
   script=latn, 
   kern=no,            % for kerning
   liga=no,            % ligatures communes
   dlig=no,            % ligatures spécifiques (exemple st) 
   calt=no,            % alternatives contextuelles
   lnum=yes,            % lining numbers
   onum=no,             % old style numbers,
   ccmp=no,            % petites majuscules
   ss04=no,            % stylistic swash
   ]
\definefontfeature
  [mesfeaturesB]
  [mode=node, 
   language=dflt, 
   protrusion=quality,  % for kerning
   expansion=quality,   % for kerning
   script=latn, 
   kern=yes,            % for kerning
   liga=yes,            % ligatures communes
   dlig=yes,            % ligatures spécifiques (exemple st) 
   calt=yes,            % alternatives contextuelles
   lnum=no,            % lining numbers
   onum=yes,             % old style numbers,
   ccmp=yes,            % petites majuscules
   ss04=yes,            % stylistic swash
   ]
%definedfont [name:garamondpremrpro*mesfeaturesA at 30 pt]%
%t fi Qu 0123
\color[darkred]{%
\definedfont [name:texgyreadventor at 36 pt]%
Amour

\cap{\definedfont [name:garamondpremrpro*mesfeaturesB at 36 pt] Amour}

\definedfont [name:garamondpremrpro*mesfeaturesB at 30 pt]%
st fi Qu 0123
}
\stopsection


%-------------------------------------------------------------------------------

\startsection[title=Couleurs]
\showcolorcomponents[gold,darkorange,firebrick,violetred,mediumorchid,royalblue3,seagreen3]
\stopsection

%-------------------------------------------------------------------------------

\startsection[title=Langue]
\start
\definedfont [name:garamondpremrpro*mesfeaturesB at 24 pt]
\startalign[flushleft]
\mainlanguage	[de]	\quotation{Inhalt}\par
\color[darkred]{\mainlanguage	[fr]	\quotation{Contenu}\par}
\mainlanguage	[en]	\quotation{Content}\par
\stopalign

\stop
\stopsection

%-------------------------------------------------------------------------------

\startsection[title=Interactivité]
Cela permet d'activer par exemple :
\startitemize[packed]
\item les liens dans la table des matières,
\item les liens d'une page à une autre,
\item les liens vers Internet,
\item les pièces jointes,
\item les signets PDF.
\stopitemize

\stopsection

\stoppart

%===============================================================================

\startpart[title={Éléments du flux principal}] 

%-------------------------------------------------------------------------------

\startsection[title=Emphase de mots]
Ceci est un \color[darkred]{\em petit texte} pour illustrer les différentes possibilités \ConTeXt\ de \importantA{balisage} et de \importantB{composition}.

Ceci est un {\em petit texte} pour illustrer les différentes possibilités \ConTeXt\ de \importantC{balisage} et de \importantD{composition}.     
\stopsection

%-------------------------------------------------------------------------------

\startsection[title=Emphase de paragraphes]
\getbuffer
\startnarrower
\color[darkred]{\getbuffer} 
\stopnarrower
\getbuffer
\stopsection

%-------------------------------------------------------------------------------

\startsection[title=Encadrement]
\index{encadrement}
\getbuffer \par
\startframed[align={normal,verytolerant},width=local,framecolor=darkred,rulethickness=2pt]
\getbuffer
\stopframed\par                 
\getbuffer
\stopsection

%-------------------------------------------------------------------------------

\startsection[title=Lignes et traits]
Coucou.\par
\blackrule [color=darkred,height=1pt, width=\textwidth]    
\getbuffer
\setuptextrules[color=darkred,rulecolor=darkred]
\starttextrule{Exemple}
\getbuffer
\stoptextrule
\stopsection

%-------------------------------------------------------------------------------

\startsection[title=Citations]
Ceci est un petit texte pour illustrer les différentes possibilités \ConTeXt\ de \color[darkred]{\quote{balisage}} et de \color[darkred]{\quotation{composition}}.     
\color[darkred]{%
\startquotation
\getbuffer
\stopquotation}
\getbuffer
\stopsection

%-------------------------------------------------------------------------------

\startsection[title=Mathématiques]

\color[darkred]{
\startformula \startalign
  \NC f(x) \NC =  \startcases
                      \NC 1,  \MC x >  0 \NR
                      \NC 0,  \MC x >= 0 \NR
                  \stopcases  \NR
  \NC g(x) \NC = x^2 +2x +1   \NR

\NC q \NC = \delta \frac{\partial p}{\partial x} = \delta(\phi) p_{vsat}(\theta) \frac{\partial \phi}{\partial x} \NR
\NC   \NC = \left[ \frac{\delta_a}{\mu(\theta)} p_{vsat}(\theta)
\right] \frac{\partial \phi}{\partial x} \NR
\NC   \NC = k \frac{\partial \phi}{\partial x}\NR
\stopalign \stopformula}

\stopsection

%-------------------------------------------------------------------------------

\startsection[title=Listes structurées]
\getbuffer
\index{itemize}
\startitemize[packed][color=darkred]
\item premier élément
\item second élément
\startitemize[packed,n][color=darkred]
\item et ici
\item et là
\stopitemize
\item troisième élément
\stopitemize

\stopsection

%-------------------------------------------------------------------------------

\startsection[title={Description, énumération}]

\definedescription [Concept] [width=0.9cm,headcolor=darkred]
\defineenumeration [Exercice][alternative=serried,width=1.1cm,headcolor=darkred]

\startConcept{Coucou}
\getbuffer
\stopConcept

\startExercice
\getbuffer
\stopExercice

\startExercice
\getbuffer
\stopExercice

\stopsection

%-------------------------------------------------------------------------------

\startsection[title=Textes tabulés]
\getbuffer
\color[darkred]{%
\starttabulate
\NC Ligne 1 \NC  Petite texte A \NC en unité a\NC\NR
\NC Ligne 2 \NC  Petite texte B \NC en unité b\NC\NR
\NC Ligne 3 \NC  Petite texte C \NC en unité c\NC\NR
\NC Ligne 4 \NC  Petite texte D \NC en unité d\NC\NR
\stoptabulate}

Et voilà.
\stopsection

%-------------------------------------------------------------------------------

\startsection[title=Tableaux]
\getbuffer
\index{tableau}

\color[darkred]{%
\bTABLE
\bTABLEhead
\bTR 
\bTH Colonne 1 \eTH 
\bTH Colonne 2 \eTH 
\bTH Colonne 3 \eTH 
\eTR
\eTABLEhead
\bTABLEbody
\bTR 
\bTD Texte 1.1 \eTD 
\bTD Texte 2.1 \eTD 
\bTD Texte 3.1 \eTD
\eTR
\bTR 
\bTD Texte 1.2 \eTD 
\bTD Texte 2.2 \eTD 
\bTD Texte 3.3 \eTD
\eTR
\eTABLEbody
\eTABLE}

\getbuffer
\stopsection

%-------------------------------------------------------------------------------

\startsection[title=Images et Combinaisons]
\getbuffer \index{image}

\externalfigure[https://meeting.contextgarden.net/2016/img/cm2016-group-photo.jpg][width=\textwidth]

\useMPlibrary [dum]
\startcombination[2*2]
{\externalfigure [dummy]
 [height=2cm,width=2cm]}
{\externalfigure [dummy]
 [height=2cm,width=2cm]}
{\externalfigure [dummy]
 [height=2cm,width=2cm]}
{\externalfigure [dummy]
 [height=2cm,width=2cm]}
\stopcombination

\stopsection

%-------------------------------------------------------------------------------

\startsection[title=Objets flottants, reference=secdemo]
\getbuffer
\start
\setupcaption[headcolor=darkred,color=darkred]
\startplacefigure
  [title=Une jolie illustration,
   reference=figdemo]
\externalfigure[http://www.cervantex.es/files/cervantex/cervanTeXcolor-small.jpg][height=2.25cm]
\stopplacefigure

\startplacefigure
  [title=Une jolie photo,
   reference=figdemo]
\externalfigure[https://meeting.contextgarden.net/2016/img/cm2016-group-photo.jpg][width=0.75\textwidth]
\stopplacefigure

\startplacetable
  [title=Un joli tableau,
   reference=tabdemo]
\bTABLE
\bTABLEhead
\bTR 
\bTH Colonne 1 \eTH 
\bTH Colonne 2 \eTH 
\bTH Colonne 3 \eTH 
\eTR
\eTABLEhead
\bTABLEbody
\bTR 
\bTD Texte 1.1 \eTD 
\bTD Texte 2.1 \eTD 
\bTD Texte 3.1 \eTD
\eTR
\eTABLEbody
\eTABLE
\stopplacetable
\stop
\stopsection

%-------------------------------------------------------------------------------

\startsection[title=Colonnes]
\startcolumns[separator=rule,color=darkred]
{\tfxx \getbuffer \getbuffer \getbuffer \getbuffer \getbuffer \getbuffer}
\stopcolumns
\stopsection

%-------------------------------------------------------------------------------

\startsection[title=Section]

\setuphead[subsection][color=darkred,style=bold]
\startsubsection[title=Titre 1]
\startsubsubsection[title=Titre 1.1]
\startsubsubsubsection[title=Titre 1.1.1]
\stopsubsubsubsection
\stopsubsubsection
\stopsubsection
\startsubsection[title=Titre 2]
\stopsubsection
\stopsection

%-------------------------------------------------------------------------------

\startsection[title=Macro-structure]
Les \color[darkred]{pages préliminaires} sont les premières pages.

Les \color[darkred]{pages finales} sont les dernières pages.

Les \color[darkred]{pages du corps de texte} se trouvent entre les deux.

Les \color[darkred]{pages d'annexe} leur succèdent immédiatement.

Les distringuer permet des les composer différemment (numérotations, couleur, mise en page…).
\stopsection

%-------------------------------------------------------------------------------

\startsection[title=Page de couverture et de titre]
Regardez la première et la dernière page de cette galerie (sur \color[darkred]{fond rouge}), ainsi que les quatre pages de changement de partie (sur \color[darkred]{fond bleu}).

\stopsection

%-------------------------------------------------------------------------------

\startsection[title=Autres éléments spécialisés]

\stopsection

%-------------------------------------------------------------------------------

\stoppart

%===============================================================================

\startpart [title={Compléments au flux principal}] 

%-------------------------------------------------------------------------------

\startsection[title=Table des matières,reference=toc]

\start
\tfxx
\setupcombinedlist[content][list={part,section}]
\setupinterlinespace[2.8ex] 
%\startcolumns[n=2]
\placecontent[criterium=all,alternative=d]
%\stopcolumns
\stop

\stopsection

%-------------------------------------------------------------------------------

\startsection[title=Abréviations et glossaire]
\abbreviation{PRAGMA}{PRAGMA Advanced Document Engineering}
\abbreviation{LMTX}{\LuaMetaTex}
\abbreviation{FAQ}{Foire aux Questions}

Une \color[darkred]{\FAQ} est une \color[darkred]{\infull{FAQ}}. 
N'hésitez pas à aller sur le Wiki ainsi que sur le site de \color[darkred]{\PRAGMA} (\color[darkred]{\infull{PRAGMA}})

\startsubsubject[title=Glossaire]
\placelistofabbreviations
\stopsubsubject

\stopsection

%-------------------------------------------------------------------------------

\startsection[title=Notes de bas de page]
\getbuffer
\setupfootnotes[textcolor=darkred,rulecolor=darkred]
\setupfootnotedefinition[color=darkred, headcolor=darkred]
\footnote{Une note de bas de page}
\getbuffer
\footnote{Une seconde}
\stopsection

%-------------------------------------------------------------------------------

\startsection[title=Notes marginales]
\getbuffer
\margintext{\color[darkred]{Note en marge}}
\getbuffer
\getbuffer
\margintext{\color[darkred]{Une autre}}
\stopsection

%-------------------------------------------------------------------------------

\startsection[title=Pièces jointes] 
\getbuffer

\attachment[file=demonstration.tex,location=leftmargin]
\getbuffer

\stopsection

%-------------------------------------------------------------------------------

\startsection[title={Références internes vers des illustrations, des sections}]

Regardez la \in{figure}[figdemo] \at{page}[figdemo].

Regardez la \in{section}[secdemo] \at{page}[secdemo].

Regardez la \in[MaRef] de la \at{page}[MaRef].

\stopsection

%-------------------------------------------------------------------------------

\startsection[title=Références externes vers internet]

\goto{Wiki de \ConTeXt}[url(https://wiki.contextgarden.net)]

\useurl [wiki] [https://wiki.contextgarden.net]
        []     [https://wiki.contextgarden.net]
\from[wiki]

\stopsection

%-------------------------------------------------------------------------------

\startsection[title=Références bibliographiques]

\usebtxdataset [../introCTX_biblio.bib]

Cf. \cite[title][Hagen2017metafun]
de \cite[author][Hagen2017metafun] \cite[Hagen2017metafun]

Ainsi que \cite[title][ConTeXtExcursion]
de \cite[author][ConTeXtExcursion] \cite[ConTeXtExcursion]

\startsubsubject[title=Bibliographie]
\placelistofpublications
\stopsubsubject

\stopsection

%-------------------------------------------------------------------------------

\stoppart

\stopbodymatter

%===============================================================================

\startappendices

\startpart        [title=Annexes]

\setupbackgrounds[page][backgroundcolor=ivory2]

%-------------------------------------------------------------------------------

\startsection[title={Listes des images, tableaux…}]

\startsubsubject[title=Figures]
\placelistoffigures[criterium=all]
\stopsubsubject

\startsubsubject[title=Tableaux]
\placelistoftables[criterium=all]
\stopsubsubject
\stopsection

%-------------------------------------------------------------------------------
\startsection[title=Index]
\placeindex
\stopsection

\stoppart

\stopappendices

%===============================================================================


\startbackmatter

%-------------------------------------------------------------------------------

\setupbackgrounds[page][background=color,backgroundcolor=darkred]
\startmakeup[custom]
\switchtobodyfont[24pt]\bf\color[white]{4\high{ème} de couverture}
\stopmakeup
\setupbackgrounds[page][backgroundcolor=]

%-------------------------------------------------------------------------------

\stopbackmatter

\stoptext

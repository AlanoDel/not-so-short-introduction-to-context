\startproject ma-cb

  %-------------------------------------------------------------------------------
  % #01 - Chargement des modules et environnements
  %-------------------------------------------------------------------------------

  % \environment ma-cb-language

  % Les abbreviations utilisées et définies avec la commande \logo
  \environment ma-cb-abbreviations

  % Eléments graphiques utilisés, définis avec des commandes metapost / metafun
  \environment ma-cb-graphics

  \usemodule[s][abr-03]
  \usemodule[visual]
  \useMPlibrary [dum]

  % pour pouvoir utiliser \shortsetup{setupsectionblock} dans le chapitre structure
  \usemodule[setups]
  \usemodule[setups-basics]
  \usemodule[x-setups-basics]
  \loadsetups[context-en]

  \usecolors[xwi]

  \setupinteraction[state=start]

  \mainlanguage[fr]
  \language[fr]

  %-------------------------------------------------------------------------------
  % #02 - LAYOUT
  %-------------------------------------------------------------------------------

  \startsetups [mylayout]

    \setuppapersize[A4]

    \def\nbdiv{11}
    \def\nbdiva{7}
    \def\nbdivb{2}

    \newdimen\marge   \marge   = 4mm
    \def\myfooterdistance{\marge}
    \def\myheaderdistance{\marge}

    \def\myheader{\dimexpr(((1\paperheight)/\nbdiv)/2-\marge)\relax}
    \def\myfooter{\dimexpr(((\nbdivb\paperheight)/\nbdiv)/2-\marge)\relax}

    \def\myleftmargin{\dimexpr(((1\paperwidth)/\nbdiv)/2-\marge)\relax}
    \def\myrightmargin{\dimexpr((\nbdivb\paperwidth)/\nbdiv)-\marge\relax}

    \setuplayout[
      width={\dimexpr((\nbdiva\paperwidth)/\nbdiv)-\marge\relax},
      backspace={\dimexpr(1\paperwidth/\nbdiv)+(\marge/2)\relax},
      % topspace + header + headerdistance = 24,75mm
      header=\myheader,
      headerdistance=\myheaderdistance,
      topspace={\dimexpr(1\paperheight/\nbdiv)+(\marge/2)-\myheader-\myheaderdistance\relax},
      % height - ( header + headerdistance + footer + footerdistance ) = 222.75mm
      footer=\myfooter,
      footerdistance=\myfooterdistance,
      height={\dimexpr((8\paperheight)/\nbdiv)-\marge+\myfooter+\myfooterdistance+\myheader+\myheaderdistance\relax},
      %
      topdistance=\marge,
      top=\myheader,
      bottomdistance=\marge,
      bottom=\myfooter,
      % edge and margin distance
      margindistance=\marge,
      edgedistance=\marge,
      %
      % left
      leftmargin=\myleftmargin,
      leftedge=\myleftmargin,
      % rigth
      rightmargin=\myrightmargin,
      rightedge=\myrightmargin]

    \newdimen\wDemoHN \wDemoHN=\dimexpr(1.0\textwidth)
    \newdimen\wDemoHW \wDemoHW=\dimexpr(1.0\textwidth+1.0\rightmargintotal)

    \newdimen\wComba \wComba=\dimexpr(0.5\textwidth+0.5\rightmargintotal-0.5\marge)
    \newdimen\wCombb \wCombb=\dimexpr(\textwidth/3+\rightmargintotal/3-2\marge/3)
    \newdimen\wDemoV \wDemoV=\dimexpr(0.5\textwidth+0.5\rightmargintotal-\marge)

  \stopsetups

  \setups{mylayout}

  \showframe

  %-------------------------------------------------------------------------------
  % #03 - OVERLAY LAYOUT STRUCTURE FOR DEBUGGING
  %-------------------------------------------------------------------------------

  \startuseMPgraphic{page}
    StartPage ;
    pickup pencircle scaled 1pt ;

    nbdiv := 11;
    marge := 4mm;
    pickup pencircle scaled 0.25pt;
    path pv ; pv := (0,0) -- (0,PaperHeight);
    path ph ; ph := (0,0) -- (PaperWidth,0);
    for i=0 upto nbdiv:

    draw pv shifted ((i*PaperWidth/nbdiv) + (marge /2),0) withcolor  \MPcolor{lightblue};
    draw pv shifted ((i*PaperWidth/nbdiv) - (marge /2),0) withcolor  \MPcolor{lightblue};

    draw ph shifted (0,(i*PaperHeight/nbdiv) + (marge /2)) withcolor \MPcolor{lightblue};
    draw ph shifted (0,(i*PaperHeight/nbdiv) - (marge /2)) withcolor  \MPcolor{lightblue};
    endfor;

    StopPage ;
  \stopuseMPgraphic

  \defineoverlay[page][\useMPgraphic{page}]
  \setupbackgrounds[page][background={foreground,page},state=repeat]

  %-------------------------------------------------------------------------------
  % #04 - CONTINUE LAYOUT CONFIGURATION
  %-------------------------------------------------------------------------------

  \setuppagenumbering[alternative=doublesided,
    location={},
    left={\tfa\ss},
    right={},
    style=bold]

  \setupheadertexts[edge][][pagenumber][pagenumber][]

  %numérotation romaine des pages liminaires ?
  \definestructureconversionset[frontpart:pagenumber][][romannumerals]
  \definestructureconversionset[bodypart:pagenumber] [][numbers]

  %-------------------------------------------------------------------------------
  % #05 - FONTS
  %-------------------------------------------------------------------------------

  %\definefontfamily[palatino][rm][palatinoltstd][features={default, quality}]
  %% by default all \it \bf \bi \smallcaps \oldstyle styles are ready to use as well as ligatures
  %% but superscript requires extra settings
  %\definefontfeature[f:superscript][sups=yes]
  % % an extra \sup macro is defined for our convenience
  %  \define[1]\sup{\feature[+][f:superscript]#1}

  \definefontfeature
    [default]
    [default]
    [protrusion=quality,expansion=quality]

  \setupalign[hz,hanging]

\usetypescriptfile[type-myalegreya]


%  \setupbodyfont[palatino,9pt]
  \setupbodyfont[myalegreya]
  \setupbodyfont[10pt]
  

  \setupwhitespace[medium]
  \setupblank[medium]
  
  % defaut \setupindenting[yes,next,medium]
  \setupindenting[none]

  %pour forcer un peu la césure des mots
  \setuptolerance[horizontal,stretch]
  \setupalign                                    % two instructions added by j.
    [hz,                                        % to eliminate most hyphenation
      lesshyphenated,                 % and get better interword spacing as well
      verytolerant,
      stretch]
  %-------------------------------------------------------------------------------
  % #06 - SETUP DIVERS
  %-------------------------------------------------------------------------------

  \setupcombinations[distance=\the\marge,
    after={\blank[\the\marge]}]

  \setupcaptions[align=flushleft,location=none]

  \setuplist
    [part]
    [textstyle=bold,
      pagestyle=bold,
      numberstyle=bold,
      before={\blank[big]\blackrule[color=darkred, height=0.5pt,width=2cm]}]

  \setuplist
    [chapter]
    [textstyle=bold,
      pagestyle=bold]

  \setupfloat
    [figure]
    [location=inner]

  \setupfloat
    [table]
    [location=inner]

  \definefont[PartStyle][Serif at 30pt]

  \setuphead[part,chapter,section,subsection,subsubsection][color=brown]

  \setuphead
    [part]
    [page=right,
      placehead=yes,
      header=empty,
      alternative=middle,
      style=PartStyle]

  \setuphead
    [subsubsection]
    [number=no,
      color=brown,
%after={\blank[nowhite]\color[brown]{\hrule}\blank[line]}]
  before={\color[brown]{\vrule width 1mm height 4mm}~~}]

  \definemargindata
    [margintext]
    [location=outer,
      align=flushleft,
      stack=continue]

  \definedescription
    [zoom]
    [alternative=outermargin,
      headcolor=brown,
      distance=0pt,
      align=flushleft]

  %pour les itemize au changement de page
  \setupitemize[each][autointro]

  %pour la mise en page des texte de code source
  \setuplinenumbering[color=black,style=\tfxx\tt,step=1,location=text]

  %-------------------------------------------------------------------------------
  % #07 - SETUP : SIDE NOTE / FOOTNOTE MARGIN
  %-------------------------------------------------------------------------------

  \definecounter[marginnotes]

  \setupmargindata
    [inouter]
    [stack=continue,
      align=flushleft,
      style=\small,
      color=middlered]

  \starttexdefinition unexpanded marginnote #1
    \incrementcounter[marginnotes]
    \high{\convertedcounter[marginnotes]}
    \inouter{
      \convertedcounter[marginnotes][numberstopper={. }]
      #1
    }
  \stoptexdefinition

  %-------------------------------------------------------------------------------
  % #08 - MISE EN PAGE DES DEMONSTRATONS
  %-------------------------------------------------------------------------------

  \setupcolor[hex]
  \definecolor[hex-orange] [h=ED7D31]
  \definecolor[hex-vert]   [h=70AD47]
  \definecolor[hex-orangel][h=FEE8E1]
  \definecolor[hex-vertl]  [h=E2EDDB]

  % si besoin de mettre en forme par TeX des commentaires dans les codes sources
  % il faut les encadrer par [[ et ]] comme indiqué dans typebuffer, option escape

  \definestartstop[comment][style={\rm}]

  % startDemoHN ==> horizontal split, normal width
  % startDemoHW ==> horizontal split, wide
  % startDemoVN ==> vertical split, normal width
  % startDemoVW ==> vertical split, wide
  % startDemoI  ==> for input file
  % startDemoC  ==> for shell command

  \startluacode
    userdata = userdata or {}

    function userdata.Demolua(buffer,lettre)
    if lettre=="N" then
    Tlarg="{\\dimexpr\\textwidth-\\marge\\relax}"
    elseif lettre=="W" then
    Tlarg="{\\dimexpr\\textwidth+\\rightmargintotal-\\marge\\relax}"
    elseif lettre=="V" then
    Tlarg="{\\dimexpr0.5\\textwidth+0.5\\rightmargintotal-1.5\\marge\\relax}"
    end

    buffers.assign("tempo",buffers.getcontent(buffer))
    buffers.assign(buffer,"\\startTEXpage %\n")
    buffers.append(buffer,"\\project ma-cb %\n")
    buffers.append(buffer,"\\setupbodyfont[palatino,9pt] %\n")
    buffers.append(buffer,"\\framed[align=normal,\n width=" .. Tlarg .. ",\n offset=0pt,\n frame=off]{%debutZ\n")
      buffers.append(buffer,buffers.getcontent("tempo"))
      buffers.append(buffer,"\n%finZ \n}\n")
    buffers.append(buffer,"\\stopTEXpage\n")
    end
  \stopluacode

  %-------------------------------------------------------------------------------

  \def\startDemoHN{\dostartbuffer[DemoTbuff][startDemoHN][stopDemoHN]}

  \def\stopDemoHN{\ctxlua{userdata.Demolua('DemoTbuff',"N")}%
    \startplacetable[location=force]
      \bTABLE
      \setupTABLE[frame=off,offset=2pt,spaceinbetween=\the\marge]
      \setupTABLE[column][1][width=\wDemoHN,align=flushleft] %
      \bTR
      \bTD[topframe=on,rulethickness=3pt,framecolor=hex-orange,background=color,backgroundcolor=gray]
      \typebuffer[DemoTbuff][bodyfont=8pt,option=tex,range={debutZ,finZ},escape={[[,]]}]
      \eTD
      \eTR
      \bTR
      \bTD[bottomframe=on,rulethickness=3pt,framecolor=hex-vert,background=color,backgroundcolor=gray]
      \strut\typesetbuffer[DemoTbuff][page=1,frame=off]
      \eTD
      \eTR
      \eTABLE
    \stopplacetable}

  %-------------------------------------------------------------------------------

  \def\startDemoHW{\dostartbuffer[DemoTbuff][startDemoHW][stopDemoHW]}

  \def\stopDemoHW{\ctxlua{userdata.Demolua('DemoTbuff',"W")}%
    \startplacetable[location=force]
      \bTABLE
      \setupTABLE[frame=off,offset=2pt,spaceinbetween=\the\marge]
      \setupTABLE[column][1][width=\wDemoHW,align=flushleft] %
      \bTR
      \bTD[topframe=on,rulethickness=3pt,framecolor=hex-orange,background=color,backgroundcolor=gray]
      \typebuffer[DemoTbuff][bodyfont=8pt,option=tex,range={debutZ,finZ},escape={[[,]]}]
      \eTD
      \eTR
      \bTR
      \bTD[bottomframe=on,rulethickness=3pt,framecolor=hex-vert,background=color,backgroundcolor=gray]
      \strut\typesetbuffer[DemoTbuff][page=1,frame=off]
      \eTD
      \eTR
      \eTABLE
    \stopplacetable}

  %-------------------------------------------------------------------------------

  \def\startDemoVW{\dostartbuffer[DemoTbuff][startDemoVW][stopDemoVW]}

  \def\stopDemoVW{\ctxlua{userdata.Demolua('DemoTbuff',"V")}%
    \startplacetable[location=force]
      \bTABLE
      \setupTABLE[frame=off,offset=2pt]
      \setupTABLE[column][1][width={\dimexpr0.5\wDemoHW-0.5\marge\relax},align=flushleft]
      \setupTABLE[column][2][width=\marge,align=flushleft]
      \setupTABLE[column][3][width={\dimexpr0.5\wDemoHW-0.5\marge\relax},align=flushleft]
      \bTR
      \bTD[leftframe=on,rulethickness=3pt,framecolor=hex-orange,background=color,backgroundcolor=gray]
      \typebuffer[DemoTbuff][bodyfont=8pt,option=tex,range={debutZ,finZ},escape={[[,]]}]
      \eTD
      \bTD    \eTD
      \bTD[rightframe=on,rulethickness=3pt,framecolor=hex-vert,background=color,backgroundcolor=gray]
      \strut\typesetbuffer[DemoTbuff][page=1,frame=off]
      \eTD
      \eTR
      \eTABLE
    \stopplacetable}

  %-------------------------------------------------------------------------------

  \def\startDemoVN{\dostartbuffer[DemoTbuff][startDemoVN][stopDemoVN]}

  \def\stopDemoVN{\ctxlua{userdata.Demolua('DemoTbuff',"V")}%
    \startplacetable[location=force]
      \bTABLE
      \setupTABLE[frame=off,offset=2pt]
      \setupTABLE[column][1][width={\dimexpr0.5\wDemoHN-0.5\marge\relax},align=flushleft]
      \setupTABLE[column][2][width=\marge,align=flushleft]
      \setupTABLE[column][3][width={\dimexpr0.5\wDemoHN-0.5\marge\relax},align=flushleft]
      \bTR
      \bTD[leftframe=on,rulethickness=3pt,framecolor=hex-orange,background=color,backgroundcolor=gray]
      \typebuffer[DemoTbuff][bodyfont=8pt,option=tex,range={debutZ,finZ},escape={[[,]]}]
      \eTD
      \bTD    \eTD
      \bTD[rightframe=on,rulethickness=3pt,framecolor=hex-vert,background=color,backgroundcolor=gray]
      \strut\typesetbuffer[DemoTbuff][page=1,frame=off]
      \eTD
      \eTR
      \eTABLE
    \stopplacetable}

  %-------------------------------------------------------------------------------

  \def\startDemoI{\dostartbuffer[DemoTbuffI][startDemoI][stopDemoI]}

  \def\stopDemoI{\ctxlua{userdata.Demolua("DemoTbuffI","N")}%
    \startplacetable[location=force]
      \bTABLE
      \setupTABLE[frame=off,offset=2pt,spaceinbetween=\the\marge]
      \setupTABLE[column][1][width=\wDemoHN,align=flushleft]
      \bTR
      \bTD[leftframe=on,rulethickness=3pt,framecolor=darkblue,background=color,backgroundcolor=gray]
      \typebuffer[DemoTbuffI][bodyfont=8pt,option=tex,range={debutZ,finZ},escape={[[,]]}]
      \eTD
      \eTABLE
    \stopplacetable}

  %-------------------------------------------------------------------------------

  \def\startDemoC{\dostartbuffer[Demobuff][startDemoC][stopDemoC]}

  \def\stopDemoC{\ctxlua{userdata.Demolua("Demobuff","N")}%
    \startplacetable[location=force]
      \bTABLE
      \setupTABLE[frame=off,offset=2pt,spaceinbetween=\the\marge]
      \setupTABLE[column][1][width=\wDemoHN,align=flushleft]
      \bTR
      \bTD[leftframe=on,rulethickness=3pt,framecolor=darkred,background=color,backgroundcolor=gray]
      \typebuffer[Demobuff][bodyfont=8pt,option=tex,range={debut,fin}] %,escape={[[,]]}]
  \eTD
  \eTABLE
\stopplacetable}


  %-------------------------------------------------------------------------------
  % #09 - POUR AFFICHER LA LISTE DES COMMANDES EN FIN DE DOC
  %-------------------------------------------------------------------------------

  \defineregister
    [Command]

  \setupregister
    [Command]
    [\c!indicator=\v!off,
      \c!before={\blank[\v!line]}]

  % TODO : c'est quoi le séparator
  \definemixedcolumns
    [documentcolumns]
    [\c!n=2,
      \c!distance=36pt,
      \c!separator=ColumnRule]

  %-------------------------------------------------------------------------------
  % #10 - ANCIEN TRUC DU DOC PAR DEFAUT
  %-------------------------------------------------------------------------------

  % definecolor
  %   ShapeDarkLine
  %   ShapeDarkDotse
  %   ShapeDarkEnde
  %   ShapeLightLine
  %   ShapeLightDots
  %   ShapeLightFill
  %   ShapeLightFrame

  % startuseMPgraphic : basic-shape-dark / basic-shape-light | basic-shape | frame-shape | setup-shape
  % startuniqueMPpagegraphic: chapter-state | note-rule | column-rule
  % startreusableMPgraphic : pagenumber-state | manualsymbol

  %\environment ma-cb-style

  % setupframedtexts
  % setupexternalfigures
  % setuplayout
  % setupwhitespace
  % setupblank
  % setuptyping
  % setuptolerance
  % setupfootnotes
  % setuphead
  % setupfootertexts

  % startsetups chapterindicator
  % startsetups coverbackground
  % startsetups coverpage
  % startsetups backpage
  % startsetups frontpart
  % startsetups bodypart
  % startsetups appendix
  % startsetups backpart

  % \setupmixedcolumns
  % \setupregister

  %=============================================================================
  % Définition des liens / références vers des documents externes

  \environment ma-cb-links

  %=============================================================================
  % Mode écran ou papier

  %   \startmode[screen]
  %       \environment ma-cb-screen
  %   \stopmode

  %=============================================================================
  % Définition des produits

  \product ma-cb-en
  \product ma-cb-nl
  \product ma-cb-fr

\stopproject

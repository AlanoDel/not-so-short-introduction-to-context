%%% File:       b13_Floats
%%% Author:    Joaquín Ataz-López
%%% Begun:     August 2020
%%% Concluded: August 2020
%%% Contents:  Arriving at this chapter I was already tired of the
%%%            introduction. Hence, for example, there is only one 
%%%            kind of table explained. Doing tables (in TeX, LaTeX or            
%%%            ConTeXt) makes me lazier. The chpater's organisation
%%%            on the other hand is somewhat heterodox (I think)
%%%            It is clear to me, but I am not sure if the public
%%%            will think that what is an ingenious" círcle" for me  
%%%            is no more than some powerful mental disorder!
%%% Edited with: Emacs + AuTeX - And at times vim + context-plugin
%%%

\environment introCTX_env_00

\startcomponent b13_Floats

\startchapter
  [title={Images, tables and other floating objects},
  reference=cap:floats]

\TocChap

This chapter is mainly about floating objects (floats). But following up on this concept, it takes advantage of it to explain two object types that are not necessarily floats, although they are often configured as if they were: external images and tables. Looking at this chapter's table of contents, one might think this is all very untidy: it begins by talking about floating objects, then goes on to talk about images and tables, and finishes by once again talking about floating objects. The reasons for this untidiness are {\em pedagogical}: images and tables can be explained without insisting too much on the fact that they are normally floats; and yet, when we start examining them it helps a lot to discover that, surprise surprise, we already know about two floating objects.

\startsection
  [
    reference=sec:floating objects,
    title={What are floating objects and what do they do?},
  ]

If a document were to contain only {\em normal} text, paginating it would be relatively easy: knowing the maximum height of the text area of the page is enough to measure the height of the different paragraphs to know where to insert page breaks. The problem is that in many documents there are objects, fragments or indivisible blocks of text such as an image, a table, a formula, a framed paragraph, etc.

Sometimes these objects can occupy a large portion of the page, which in turn poses the problem that if you have to insert it at a particular point in the document, it may not fit on the current page, and has to be interrupted abruptly, leaving a large blank space at the bottom, so that the object in question, and the text that follows it, are moved to the next page. The rules of good typesetting, however, indicate that, except for the last page of a chapter, there should be the same amount of text on each page. 

It is therefore advisable to avoid large blank vertical spaces appearing; and {\em floating} objects are the main mechanism for achieving this. A \quotation{floating object} is one that does not have to be located at an exact point in the document, but can {\em move} or {\em float} around it. The idea is to allow \ConTeXt\ to decide on the best place, from a pagination point of view, to locate such objects, even authorising them to move to another page; but always trying not to move too far away from the point of inclusion in the source file.

Therefore, there are no objects that have to be floats {\em per se}. But there are objects that will occasionally need to be floats. The decision, in any case, is up to the author or the person in charge of typesetting, if they are two different people.

Undoubtedly, allowing the exact placement of an indivisible object to change, very much facilitates the task of typesetting nicely balanced pages; but the problem that goes with this is that since we don't know exactly where such an object will end up at the time we are writing the original, it is difficult to make reference to it. So, for example, if I have just put a command in my document that inserts an image and in the next paragraph I want to describe it and write something about it like: \quotation{As you can see from the previous figure}, when the figure {\em floats} it could well be placed {\em after} what I have just written and the result is an inconsistency: the reader is looking for an image {\em before} the text that refers to it and can't find it because after floating, the image has ended up after that reference.

This is fixed by {\em numbering} floating objects (after distributing them in categories), so that in instead of referring to an image as \quotation{the previous image} or \quotation{the next image}, we will refer to it as \quotation{image 1.3}, since we can use \ConTeXt's internal reference mechanism to ensure that the image number is always kept up to date (see \in{section}[sec:references]). The numbering of these kinds of objects, on the other hand, makes it easier to quite easily create an index of them (index of tables, graphs, images, equations, etc.). For how to do this, see (\in{section}[sec:lists]).

The mechanism for dealing with floating objects in \ConTeXt\ is quite sophisticated and so abstract at times that it may not make it suitable for beginners. Therefore, in this chapter I will start by explaining it using two particular cases: images and  tables. Then I will try to generalise somewhat so that we can understand how to extend the mechanism to other kinds of objects.

\stopsection

\startsection
  [title={External images},reference=sec:extimage]  % TODO Garulfo modified. added ref

As the reader at this stage knows (since it has been explained in \in{section}[sec:ctx]), \ConTeXt\ is perfectly integrated with MetaPost and can generate images and graphics that are {\em programmed} in much the same way as text transformations are programmed. There is also an extension module for \ConTeXt\footnote{\ConTeXt\ extension modules give it additional utilities but are not included in this introduction.} that allows it to work with TiKZ.\footnote{This is a graphics programming language intended to work with \TeX-based systems. It is a \quotation{recursive acronym} from the German sentence \quotation{TiKZ ist keinen Zeichenprogramm} which translated means: \quotation{TiKZ is not a drawing program}. Recursive acronyms are a kind of programmers' joke. Leaving aside MetaPost (which I do not know how to use), I believe that TiKZ is a great system for programming graphics with.} But such images are not dealt with in this introduction (as this would probably force its length to be doubled). I am referring here to the use of external images, which reside in a file on our hard drive or are downloaded directly from the Internet by \ConTeXt.

\startsubsection
  [title={Directly inserting images}]
  \PlaceMacro{externalfigure}

To directly insert an image (not as a floating object) we use the \tex{externalfigure} command whose syntax is

\type{\externalfigure [Name] [Configuration]}

where 

\startitemize

\item {\em Name} can be either the name of the file containing the image, or the web address of an image found on the Internet, or a symbolic name we have previously associated with an image using the \PlaceMacro{useexternalfigure}\tex{useexternalfigure} command whose format is similar to that of \tex{externalfigure} although it takes a first argument with the symbolic name that will be associated with the image in question.

\item {\em Configuration} is an optional argument that allows us to apply certain transformations to the image before it is inserted into our document. We will examine this argument more closely in  \in{section}[sec:configimage].

\stopitemize

The image formats allowed are pdf, mps, jpg, png, jp2, jbig, jbig2, jb2, svg, eps, gif or tif. \ConTeXt\ can directly manage eight of these, while the rest (svg, eps, gif or tif) need to be converted with an external tool before opening them, that changes according to the format and therefore must be installed on the system so that \ConTeXt\ can manipulate these kinds of files.

\startSmallPrint

  Among the formats supported by \tex{externalfigure} are also some video formats. In particular: QuickTime (extension .mov), Flash Video   (extension .flv) and MPeg~4 (extension .mp4). But most PDF players do not know how to handle PDF files with video embedded in them. I can't say much about this, as I haven't done any tests.

\stopSmallPrint

There is no need to indicate the file extension: \ConTeXt\ will search for a file with the specified name and one of the extensions for the known image formats. If there are several candidates, first the PDF format is used if there is one, and in its absence the MPS format (graphics generated by MetaPost). In the absence of these two, the following order is followed: jpeg, png, jpeg~2000, jbig and jbig2.

\startSmallPrint

  If the actual format of the image does not correspond to the extension of the file that stores it, \ConTeXt\ cannot open it unless we indicate the actual format of the image using the {\tt method} option.

\stopSmallPrint

If the image is not placed on its own outside of a paragraph, but is integrated into a text paragraph, and its height is greater than the line spacing, the line will be adjusted to prevent the image from overlapping the previous lines, as in the example that accompanies this line\externalfigure[cow-brown][width=2.5em].

By default, \ConTeXt\ searches for the images in the working directory, in its parent directory and in that directory's parent directory. We can indicate the location of a directory containing the images we will work with using the {\tt directory} option of the \tex{setupexternalfigures} command, which would add that directory to the search path. If we want the search to be performed only in the image directory, we have to set the {\tt location} option as well. So, for example, so that our document looks for all the images we need in the \MyKey{img} directory, we should write:

\starttyping
  \setupexternalfigures
    [directory=img, location=global]
\stoptyping

\startSmallPrint

  In the {\tt directory} option in \tex{setupexternalfigures}, we can include more than one directory, separating them with commas; but in this case we need to enclose the directories within curly brackets. For example \MyKey{directory=\{img, \lettertilde/imágenes\}}.

  In {\tt directory} we always use the \quote{/} character as the separator between directories; including in Microsoft Windows whose operating system uses the \quote{\backslash} as its directories separator.

\stopSmallPrint

\tex{externalfigure} is also able to use images hosted on the Internet. So, for example, the following snippet will insert the CervanTeX logo directly from the Internet into the document. This is the \TeX\ Spanish-speaking user group:\footnote{Internet addresses are very long, and there is not much space available to display the double-column example. Therefore, in order to make the order in the left column fit properly, I have inserted a line break within the web address. If someone wants to copy and paste the example, it will not work if this line break is not deleted.}

\startDoubleExample
\starttyping
\externalfigure
[http://www.cervantex.es/files/
cervantex/cervanTeXcolor-small.jpg]
\stoptyping

\externalfigure
[http://www.cervantex.es/files/cervantex/cervanTeXcolor-small.jpg]

\stopDoubleExample

\startSmallPrint

  When a document containing a remote file is first compiled, it is downloaded from the server and stored in the LuaTeX cache directory. This cached file is used during subsequent compilations. Normally, the remote image is downloaded again if the image in the cache is older than 1~day. To change this threshold see the \goto{\ConTeXt\ wiki}[url(https://wiki.contextgarden.net/Using_Graphics)].

\stopSmallPrint

If \ConTeXt\ does not find the image that should be inserted, no error is generated, but instead of the image a text block will be inserted with information about the image that should go there. The size of this block will be the image size (if known by \ConTeXt\) or, otherwise, a standard size. There is an example of this in \in{section}[sec:startcombination].

\stopsubsection

\startsubsection
  [
    reference=sec:placefigure,
    title={Inserting an image with \tex{placefigure}},
  ]
  \PlaceMacro{placefigure}

Images can be directly inserted. But it is preferable to do this with \tex{placefigure}. This command causes \ConTeXt:

\startitemize

\item to know that an image is being inserted that must be incorporated into the list of images in the document that can then be used, if we wish, to produce an index of images.

\item to assign a number to the image, thus facilitating internal references to it.

\item to add a title to the image, creating a text block between the image and its title that means these cannot be separated.

\item to automatically set the white space (horizontal and vertical) needed for the image to be viewed correctly.

\item to position the image in the place indicated, making the text flow around it if necessary.

\item to convert the image to a floating object if it is possible, taking into account its size and location specifications.\footnote{This latter is my conclusion, given that among the placement options there are ones like {\tt force} or \Conjecture {\tt split} that go against the true notion of the floating object.}

\stopitemize
  
\stopfigure

The syntax for this command is as follows:

\type{\placefigure[Options] [Label] {Title} {Image}}

The various arguments have the following meanings:

\startitemize

\item {\em Options} are a set of indications that generally refer to where to place the image. Since these options are the same in this and other commands, I will explain them together later (in \in{section}[sec:placingobjects]). For now, I will use the {\tt here} option for examples. It tells \ConTeXt\ that, as far as possible, it should place the image exactly at the point in the document where the command that inserts it is found.

\item {\em Label} is a text string to refer internally to this object so we can make reference to it (see \in{section}[sec:references]).

\item {\em Title} is the title text to be added to the image.

\item {\em Image} is the command that inserts the image.

\stopitemize

For example

\starttyping
  \placefigure
    [here]
    [fig:texknuth]
    {\TeX\ logo and photo of {\sc Knuth}}
    {\externalfigure[https://i.ytimg.com/vi/8c5Rrfabr9w/maxresdefault.jpg]}
\stoptyping

  \placefigure
    [here]
    [fig:texknuth]
    {\TeX\ logo and photo of {\sc Knuth}}
    {
      \externalfigure
        [https://i.ytimg.com/vi/8c5Rrfabr9w/maxresdefault.jpg]
        [scale=600]
    }

As we can see in the example, by inserting the image (which, by the way,  has been done directly from an image hosted on the Internet), there are some changes regarding what happens when using the \tex{externalfigure} command directly. Vertical space is added, the image is centred and a title added. Those are {\em external} changes obvious at first sight. From an internal point of view the command has also produced other no less important effects:

\startitemize

\item First of all, the image has been inserted into the \quotation{list of images} which \ConTeXt\ maintains internally for objects inserted into the document. This, in turn, means that the image will appear in the image index that can be generated with \tex{placelist[figure]} (see \in{section}[sec:lists]), although there are two specific commands to generate the image index which are
  \PlaceMacro{placelistoffigures}\tex{placelistoffigures} or
  \PlaceMacro{completelistoffigures}\tex{completelistoffigures}.

\item Secondly, the image has been linked to the label that was added as the second argument to the \tex{placefigure} command, which means that from now on we can make internal references to it using that label (see \in{section}[sec:references]).

\item Finally, the image has become a float, which means that if, for typesetting needs (pagination) it needed to move, \ConTeXt\ would alter its placement.

\stopitemize

Actually, \tex{placefigure}, despite its name, is not only used for inserting images. We can insert anything with it, including text. However, the text or other items inserted into the document with \tex{placefigure}, will be treated {\em as if they were an image}, even though they are not; they will be added to the list of images internally managed by \ConTeXt, and we can apply transformations similar to the ones we use for images such as scaling or rotating, framing, etc. Thus the following example:

\placefigure
  [here, force]
  [fig:testtext]
  {Using \backslash placefigure for text transformations}
  {\rotate[rotation=180]{\framed{\tfd Test text}}}

which is achieved as follows:

\starttyping
\placefigure
  [here, force]
  [fig:testtext]
  {Using \backslash placefigure for text transformations}
  {\rotate[rotation=180]{\framed{\tfd Test text}}}
\stoptyping

\stopsubsection

\startsubsection
  [title={Inserting images integrated into a text block}]

  Except for very small images, which can be integrated into a line without too much disruption to paragraph spacing, images are usually inserted into a paragraph that contains only them (or put in other words, the image can be thought of as a paragraph in its own right). If the image is inserted with \tex{placefigure} and its size allows, depending on what we have indicated regarding its placement (see \in{section}[sec:placingobjects]), \ConTeXt\ will allow the text from the previous and subsequent paragraphs to flow around the image. However, if we want to ensure that a certain image will not be separated from a certain text, we can use the {\tt \PlaceMacro{startfiguretext}figuretext} environment whose syntax is as follows:

\starttyping
  \startfiguretext
    [Options]
    [Label]
    {Title}
    {Image}

    ... Text

  \stopfiguretext
\stoptyping

The environment's arguments are exactly the same as for \tex{placefigure} and have the same meaning. But here the options are no longer options for placement of a floating object, but indications regarding the integration of the image into the paragraph; so, for example, \MyKey{left} here means that the image will be placed on the left while text will flow to the right, while \MyKey{left, bottom} will mean that the image must be placed on the lower left side of the text associated with it.

The text written within the environment is what will flow around the image.

\stopsubsection

\startsubsection
  [
    reference=sec:configimage,
    title={Configuration and transformation of images inserted},
  ]

\startsubsubsection
  [title={Insert command options that cause some transformation of the image}]
  \PlaceMacro{setupexternalfigures}

The final argument in the \tex{externalfigure} command allows us to carry out certain adjustments to the image inserted. We can make these adjustments:

\startitemize

\item In general for all images to be inserted in the document; or for all images to be inserted from a certain point. In this case we make the adjustment with the \tex{setupexternalfigures} command.

\item For a specific image that we want to insert several times in the document. In this case the adjustment is made in the last argument of the \tex{useexternalfigure} command that associates an external figure with a symbolic name.

\item At the exact moment when we insert a specific image. In this case the adjustment is made in the \tex{externalfigure} command itself.

\stopitemize

The changes in the image that can be achieved by this route are the following:

\startdescription{Changing the size of the image.}

We can do this:

\startitemize

\item {\em By assigning a precise width or height}, something done with the {\tt width} and {\tt height} options respectively; if only one of the two values is adjusted, the other is automatically adapted to maintain the proportion.

  We can assign a precise height or width, or indicate it as a percentage of page height or line width. For example: 

  \type{width=.4\textwidth}

  will ensure that the image has a width equal to 40\% of the line width.

\item {\em Scaling the image}: The {\tt xscale} option will scale the image horizontally; {\tt yscale} will do so vertically, and {\tt scale} will do it horizontally and vertically. These three options expect a number representative of the scaling factor multiplied by 1000. That is to say: {\tt scale=1000} will leave the image in its original size, while {\tt scale=500} will reduce it by half, and {\tt scale=2000} will double its size.

  A conditional scaling, which is applied only if the image exceeds certain dimensions, is obtained with the {\tt maxwidth} and {\tt maxheight} options that take a dimension. For example {\tt maxwidth=.2\backslash textwidth} will scale the image only if it turns out to be more than 20\% of the line width.

\stopitemize  

\stopdescription

\startdescription{Rotating the image.}

  To rotate the image we use the {\tt orientation} option which takes a number representative of the number of degrees of rotation that will be applied. The rotation is done in a counter-clockwise direction.

  \startSmallPrint

    The wiki implies that the rotations that can be achieved with this option must be multiples of 90 (90, 180 or 270) but to achieve a different rotation we would have to use the \tex{rotate} command. However, I have not had any problem applying a 45 degree rotation to an image with only {\tt orientation=45}, without the need to use the \tex{rotate} command.

  \stopSmallPrint
  
\stopdescription

\startdescription{Framing the image.}

  We can also surround the image with a frame using the  {\tt frame=on} option, and configure its colour ({\tt framecolor}), the distance between the frame and the image ({\tt frameoffset}), the thickness of the line that draws the frame ({\tt rulethickness}) or the shape of its corners ({\tt framecorner}) which can be rounded ({\tt round}) or rectangular.

\stopdescription

\startdescription{Other configurable aspects of images.}

  In addition to the aspects already seen, which imply a transformation in the image to be inserted, using \tex{setupexternalfigures} we can configure other aspects, such as where to look for the image ({\tt directory} option), whether the image should be searched for only in the indicated directory ({\tt location=global}) or whether it should also include the working directory and its parent directories ({\tt location=local}), and whether the  image will or will not be interactive ({\tt interaction}), etc.
  
\stopdescription

\stopsubsubsection

\startsubsubsection
  [title={Specific commands for transforming an image}]
 
There are three commands in \ConTeXt\ that produce some transformation in an image and can be used in combination with \tex{externalfigure}. These are:

\startitemize
  
\item {\em Mirror image}: achieved with the \PlaceMacro{mirror}\tex{mirror} command.

\item {\em Clipping}: this is achieved with the \PlaceMacro{clip}\tex{clip} command when the width ({\tt width}), height ({\tt height}), horizontal offset ({\tt hoffset}) and vertical offset ({\tt voffset}) dimensions are given. For example:

\starttyping
\clip
  [width=2cm, height=1cm, hoffset=3mm, voffset=5mm]
  {\externalfigure[logo.pdf]}
\stoptyping

\item {\em Rotation.}
  A third command able to apply transformations to an image is the \PlaceMacro{rotate}\tex{rotate} command. It can be used in conjunction with \tex{externalfigure} but normally this would not be necessary given that the latter has, as we have seen, the {\tt orientation} option that produces the same result.

\startSmallPrint

\stopSmallPrint

\stopitemize

The typical use of these commands is with images, but they actually act on {\em boxes}. That's why we can apply them to any text simply by enclosing it in a box (which the command does automatically), that will produce curious effects like the following:

\startDoubleExample
\vbox{\starttyping
  \mirror{Test text}\\
  \rotate[rotation=20] {Test text}
\stoptyping}

\vbox{  \mirror{Test text}\\
  \rotate[rotation=20]
     {Test text}}

\stopDoubleExample

\stopsubsubsection

\stopsubsection

\stopsection

\startsection
  [
    reference=sec:tables,
    title={Tables},
  ]

\startsubsection
  [title={General ideas about tables and their placement in the document}]

Tables are structured objects that contain text, formulas or even images arranged in a series of {\em cells} that allow us to graphically see the relationship between the contents of each cell. To do this, the information is organised into rows and columns: all data (or entries) in the same row have a certain relationship to each other, as well as all data (or entries) in the same column. A cell is the intersection of a row with a column, as shown in \in{figure}[fig:table].

\placefigure
  [right]
  [fig:table]
  {Image of a simple table}
  {\externalfigure[tablas_eng][width=.6\textwidth]}
  
Tables are ideal for displaying text or data that are related to each other, because as each one is locked in its own cell, even if its content, or the content of the remaining cells changes, the relative position of one with respect to the others will not change.

Of all the tasks involved in typesetting a text, the creation of tables is the only one that, in my opinion, is easier to do in a graphic program (word processor type) than in \ConTeXt. Because it's easier {\em to draw} the table (which is what you do in a word processing program) than {\em describe it} which is what we do when we work with \ConTeXt. Every row change and column change requires the presence of a command, and that means that it takes much longer to implement the table, instead of simply saying how many rows and columns we want.

\ConTeXt\ has three different mechanisms for producing tables; the {\tt tabulate} environment which is recommended for simple tables and which is the most directly inspired by \TeX\ tables; the so-called {\em natural tables}, inspired by HTML tables, suitable for tables with special design needs where, for example, not all rows have the same number of columns; and the so-called {\em extreme tables}, clearly based on XML and recommended for medium or long tables which take up more than one page. Of the three, I will explain only the first. The natural tables are reasonably well explained in \quotation{\ConTeXt\ Mark IV an excursion}, and for {\em extreme tables} there is a document about them in the \suite- documentation.

Something similar to what happens with images occurs in tables: we can simply write the necessary commands at some point in the document to generate a table and it will be inserted at that exact point, or we can use the \PlaceMacro{placetable}\tex{placetable} command to insert a table. This has some advantages:

\startitemize

\item \ConTeXt\ numbers the table and adds it to the list of tables allowing internal references to the table (through its numbering), including it in an eventual index of tables.

\item We will gain flexibility in table placement within the document, thus facilitating the task of pagination.

\stopitemize

The format of \tex{placetable} is similar to what we saw for
\tex{placefigure} (see \in{section}[sec:placefigure]):

\type{\placetable[Options] [Label] {Title} {table}}

I refer to sections \in{}[sec:placingobjects] and \in{}[sec:confcaptions] regarding options relating to table placement and configuring the title. Among the options there is one, however, that seems to be designed exclusively for tables. This is the \MyKey{split} option which, when set, authorises \ConTeXt\ to extend the table over two or more pages, in which case the table can no longer be a floating object.

In general terms we can set the configuration for tables with the \PlaceMacro{setuptables}\tex{setuptables} command. Also, as with images, it is possible to generate an index of tables with
\PlaceMacro{placelistoftables}\tex{placelistoftables} or \PlaceMacro{completelistoftables}\tex{completelistoftables}. In this regard see \in{section}[sec:variouslists].

\stopsubsection

\startsubsection
  [title={Simple tables with the {\tt tabulate} environment}]
\PlaceMacro{starttabulate}

The simplest tables are those achieved with the {\em
  tabulate} environment whose format is:

\vbox{\starttyping
  \starttabulate[Table column layout]
    ... % Table contents
    ...
    ...
  \stoptabulate
\stoptyping}

Where the argument taken in square brackets describes (in code) the number of columns the table will have, and (sometimes indirectly) indicates their width. I say that the argument describes the design {\em in code}, because at first glance it seems very cryptic: it consists of a sequence of characters, each with a special meaning. I will explain it little by little and in steps, because I think that this way it is easier to understand.

\startSmallPrint

  This is the typical case in which the huge number of aspects that we can configure means we need a lot of text to describe it. This seems to be devilishly difficult. In fact, for most of the tables that are built in practice, points 1 and 2 are enough. The rest are extra possibilities that it is useful to know exist, but are not essential to know to typeset a table.

\stopSmallPrint

\startitemize[n]

\item {\bf Columns delimiter}: the \MyKey{\|} character is used to delimit table columns. So, for example, \MyKey{[\|lT\|rB\|]} will describe a table with two columns, one of which would have the characteristics associated with the indicators \MyKey{l} and \MyKey{T} (which we will see immediately following) and the second column will have the characteristics associated with \MyKey{r} and \MyKey{B}. A simple three-column table aligned to the left, for example, would be described as \MyKey{[\|l\|l\|l\|]}.

\item {\bf Determining the basic nature of the cells in a column:} The first thing to determine when we build our table is if we want the content of each cell to be written on a single line, or if, on the contrary, if the text of any column is too long we want our table to distribute it over two or more lines. In the {\tt tabulate} environment that question is not decided cell by cell but is considered a characteristic of the columns.

  \startitemize[a]

  \item {\em One line cells:} If the contents of the cells in a column, regardless of their length, are to be written on a single line, we must specify the alignment of the text in the column, which can be left (\MyKey{l}, from {\em left}), right (\MyKey{r}, from {\em right}) or centred
    (\MyKey{c}, from {\em center}).

    \startSmallPrint
      
      In principle, these columns will be as wide as necessary to fit the widest cell. But we can limit the width of the column with the  \MyKey{w(Width)} specifier. For example, \MyKey{[\|rw(2cm)\|c\|c\|]}  will describe a table with two columns, the first aligned to the right and with an exact width of 2 centimetres, and the other two centred and with no width limitation.

      It should be noted that the width limitation in single-line columns may cause the text in one column to overlap text in the next column. So my advice is that when we need fixed-width columns, always use multi-line cell columns.

    \stopSmallPrint
    
  \item {\em Cells that can take up more than one line if needed}: the \MyKey{p} specifier generates columns in which the text in every cell will occupy as many lines as needed. If we simply specify \MyKey{p}, the width of the column will be the full width available. But it is also possible to indicate \MyKey{p(Width)}, in which case the width will be that expressly specified. Thus the following examples:

\starttyping
\starttabulate[|l|r|p|]
\starttabulate[|l|p(4cm)|]
\starttabulate[|r|p(.6\textwidth)|]
\starttabulate[|p|p|p|]
\stoptyping

The first example will create a table with three columns, the first and second of a single line, aligned, to the left and right respectively, and the third, which will occupy the remaining width and the height required to house all its contents. In the second example, the second column will measure exactly four centimetres wide, whatever its content; but if it does not fit in that space, it will take up more than one line. The third example calculates the width of the second column in proportion to the maximum width of the line, and in the last example, there will be three columns that will be the width available in equal parts.

  \stopitemize

  \startSmallPrint

    Note that, in reality, if a cell is a quadrilateral, what the \MyKey{p}  specifier does is authorise a variable height for the cells in a column, depending on the length of the text.
    
  \stopSmallPrint

\item {\bf Adding indications to the description of the column, about the font style and variant to be used}: once the basic nature of the column (width and height, automatic or fixed, of the cells) has been decided, we can still add, in the description of the contents of the column, a character representative of the {\em format} in which it must be written. These characters can be \MyKey{B} for bold, \MyKey{I} for italic, \MyKey{S} for slanted, \MyKey{R} for Roman style lettering or \MyKey{T} for  {\em typewriter} style lettering.

\head {\bf Other additional aspects that can be specified in the description of table columns}:

  \startitemize[1]

  \item {\em Columns with maths formulas}: the \MyKey{m} and \MyKey{M} specifiers enable maths mode in a column without the need to specify it in each of its cells. The cells in this column will not be able to hold normal text.

    \startSmallPrint

      Although \TeX, \ConTeXt's predecessor, came into existence for typesetting any kind of maths, until now I have hardly said anything about writing maths. In the maths mode (which I will not be explaining) \ConTeXt\ alters our normal rules and even uses different fonts. The maths mode has two varieties: one we could call {\em linear} in that the formula is housed within a line containing normal text (\MyKey{m} indicator), and the {\em complete maths mode} that displays formulas in an environment where there is no normal text. The main difference between the two modes, in a table, is basically the size which the formula will be written in and the horizontal and vertical space surrounding it.

    \stopSmallPrint

  \item {\em Add extra horizontal white space around the contents of the cells in a column}: with the \MyKey{in}, \MyKey{jn} and \MyKey{kn} indicators we can add extra white space to the left of the column contents (\MyKey{in}), to the right (\MyKey{jn}) or to both sides (\MyKey{kn}). In all three cases \MyKey{n} represents the number by which to multiply the white space that would normally be left without one of these specifiers (by default the average is an {\em em}). So, for example,    \MyKey{\|j2r\|} will indicate that we are faced with a column that will be aligned to the right, and in which we want a blank space of 1 {\em em}'s width.

  \item {\em Adding text before or after the contents of each cell in a column}. The {\tt b\{Text\}} and {\tt a\{Text\}} specifiers cause the text between curly brackets to be written before (\MyKey{b}, from {\em before}) or after (\MyKey{a}, from {\em after}) the cell's contents.

  \item {\em Applying a format command to the entire column}. The \MyKey{B}, \MyKey{I}, \MyKey{S}, \MyKey{R} \MyKey{T} indicators we mentioned previously do not cover all the format possibilities: e.g. there is no indicator for small caps, or for {\em sans serif}, or that affects the font size. With the \MyKey{f\backslash Command} indicator we can specify a format command that is automatically applied to all cells in a column. For example, \MyKey{\|lf\backslash sc\|} will typeset the column's contents in small caps.

  \item {\em Applying any command to all the cells in the column}. Finally, the \MyKey{h\backslash Command} indicator will apply the specified command to all cells in the column.

\stopitemize

\stopitemize

In \in{table}[tbl:examplestabulate] some examples of table format specification strings are shown.

\placetable
  [here]
  [tbl:examplestabulate]
  {Some examples of how to specify the format of the columns in {\tt tabulate}}
{\starttabulate[|lT|p(.6\textwidth)|]
\HL
\NC{\bf\rm Format specifier}
\NC{\bf Meaning}
\NR
\HL
\NC \|l\|
\NC Generates a column whose width is automatically left-aligned.
\NR
\NC \|rB\|
\NC Generates a column whose width is automatically right-aligned, and in bold.
\NR
\NC \|cIm\|
\NC Generates a column enabled for maths content. Centred and in italics.
\NR
\NC \|j4cb\{---\}\|
\NC This column will have contents centred, will begin with an em dash (---) and will add 2 {\em ems} white space to the right.
\NR
\NC \|l\|p(.7\tex{textwidth})\|
\NC generates two columns: the first is left-aligned and width automatic. The second takes up 70\% of the total width of the line.
\NR
\HL
\stoptabulate}

Once the table has been designed, its contents need to be input. To explain how to do this I will start by describing how a table should be filled in  where we have lines separating rows and columns:

\startitemize

\item We always start by drawing a horizontal line. In a table this is done with the \PlaceMacro{HL}\tex{HL} command (from {\em Horizontal Line}).

\item Then we write the first line: at the beginning of each cell we must indicate that a new cell begins and that a vertical line must be drawn. This is done with the \PlaceMacro{VL}\tex{VL} command (from {\em Vertical Line}). So we start with this command, and we write the content of each cell. Every time we change cells we repeat the \tex{VL} command.

\item At the end of a row, we expressly indicate that a new row is going to be started with the \PlaceMacro{NR}\tex{NR} command (from {\em Next Row}). After it we repeat the \tex{HL} to draw a new horizontal line.

\item And so, one by one, we write all the rows of the table. When we finish we add, as an extra, a \tex{NR} command and another \tex{HL} to close the grid with the bottom horizontal line.

\stopitemize

If we do not want to draw the table grid, we remove the \tex{HL} commands and replace the \tex{VL} commands with \PlaceMacro{NC}\tex{NC} (from
{\em New Column}).

It's not especially difficult when we get the hang of it, although when we look at the source code for a table it's hard to get an idea what it will look like. In \in{table}[tbl:tablecommands] we see the commands that can (and must) be used within a table. There are some that I have not explained, but I think the description I have given is enough.

\placetable
  [here, force]
  [tbl:tablecommands]
  {Commands to be used within a table}
{\starttabulate[|l|p(.6\textwidth)|]
\HL
\NC {\bf Command}
\NC {\bf Meaning}
\NR
\HL
\NC \tex{HL}
\NC Inserts a horizontal line
\NR
\NC \tex{NC}
\NC Begins a new column
\NR
\NC \tex{NR}
\NC Begins a new row
\NR
\NC \tex{VL}
\NC Inserts a vertical line delimiting a column (used in place of \tex{NC})
\NR
\NC \PlaceMacro{NN}\tex{NN}
\NC Begins a column in maths mode (used in place of \tex{NC})
\NR
\NC \PlaceMacro{TB}\tex{TB}
\NC Adds some extra vertical space between two rows
\NR
\NC \PlaceMacro{NB}\tex{NB}
\NC Indicates that the next row starts an indivisible block within which there cannot be a page break
\NR
\HL
\stoptabulate}

And now, as an example I will transcribe the code with which \in{table}[tbl:tablecommands] was written.

\starttyping
\placetable
  [here]
  [tbl:tablecommands]
  {Commands to be used within a table}
{\starttabulate[|l|p(.6\textwidth)|]
\HL
\NC {\bf Command}
\NC {\bf Meaning}
\NR
\HL
\NC \tex{HL}
\NC Inserts a horizontal line
\NR
\NC \tex{NC}
\NC Begins a new column
\NR
\NC \tex{NR}
\NC Begins a new row
\NR
\NC \tex{VL}
\NC Inserts a vertical line delimiting a column (used in place of \tex{NC})
\NR
\NC \tex{NN}
\NC Begins a column in maths mode (used in place of \tex{NC})
\NR
\NC \tex{TB}
\NC Adds some extra vertical space between two rows
\NR
\NC \tex{NB}
\NC Indicates that the next row starts an indivisible block within which there cannot be a page break
\NR
\HL
\stoptabulate}
\stoptyping

The reader will notice that in general I have used one (or two) lines of text for each cell. In a real source file I would have only used a line of text for each cell; in the example I have split the lines that are too long. Using a single line per cell makes it easier for me to write the table because what I do is to write the contents of each cell, without row or columns separation commands. When everything is written, I select the text from the table and ask my text editor to insert \quotation{\tex{NC }} at the beginning of each line. After that, every two lines (because the table has two columns) I insert a line that adds the \tex{NR} command, because every two columns starts a new row. Finally, by hand, I insert the \tex{HL} commands at the points where I want a horizontal line to appear. It takes me almost longer to describe it than to do it!

But also see how, within a table, we can use \ConTeXt's ordinary commands. In particular in this table we continually use \tex{tex} which is explained in \in{section}[sec:verbatim].

\stopsubsection

\stopsection

\startsection
  [title={Aspects common to images, tables and other floating objects}]

We already know that images and tables do not have to be floating objects, but they are good candidates to be so, although they have to be inserted in the document by means of the \tex{placefigure} or \tex{placetable} commands. In addition to these two commands, and with the same structure, in \ConTeXt\ we have the \PlaceMacro{placechemical}\tex{placechemical} command (to insert formulas chemicals), the \PlaceMacro{placegraphic}\tex{placegraphic} command (to insert graphics) and the \PlaceMacro{placeintermezzo}\tex{placeintermezzo}  command for inserting a structure that \ConTeXt\ calls {\em Intermezzo} and which I suspect refers to framed text fragments. All these commands are in turn concrete applications of a more general command that is \PlaceMacro{placefloat}\tex{placefloat} whose syntax is the following:

\type{\placefloat [Name] [Options] [Label] {Title} {Contents}}

Note that \tex{placefloat} is identical to \tex{placefigure} and \tex{placetable} except for the first argument that in \tex{placefloat} takes the name of the floating object. This is because {\em each type of floating object can be inserted into the document with two different commands}: \tex{placefloat[TypeName]} or \tex{placeTypeName}. In other words: \tex{placefloat[figure]} and \tex{placefigure} are exactly the same command, just as \tex{placefloat[table]} is the same command as \tex{placetable}.

I will therefore speak from now on of \tex{placefloat}, but please note that everything I say will also apply to \tex{placefigure} or \tex{placetable} which are specific applications of said command.

The \tex{placefloat} arguments are:

\startitemize

\item {\em Name}. refers to the floating object in question. It can be some predetermined floating object ({\tt figure, table, chemical, intermezzo}) or a floating object created by ourselves using \tex{definefloat} (see \in{section}[sec:definefloat]).

\item {\em Options}. A series of symbolic words that tell \ConTeXt\ how it should insert the object. The great majority of these refer to {\em where} to insert it. We will see this in the next section.

\item {\em Label}. A label for future internal references to this object.

\item {\em Title}. The title text to be added to the object. Regarding its configuration, see \in{section}[sec:confcaptions].

\item {\em Contents}. This depends, of course, on the type of object. For images it is usually a \tex{externalimage} command; for tables, the commands that will create the table; for {\em intermezzi}, a framed text fragment; etc.

\stopitemize

The first three arguments, which are introduced in square brackets, are optional. The last two (which are introduced between curly brackets) are mandatory, although they can be empty. So, for example:
\cmd{placefloat\{\}\{\}} will insert:

\placefloat{}{}

in the document.

\startitemize

  {\bf Note:} We see that \ConTeXt\ considered that the object to be inserted was an image, since it was numbered as an image and included in the list of \Conjecture images. This makes me assume that images are the default floating objects.

\stopitemize

\startsubsection
  [
    reference=sec:placingobjects,
    title={Floating object insertion options},
  ]

The {\em Options} argument in \tex{placefigure}, \tex{placetable} and \tex{placefloat} controls different aspects regarding the insertion of these types of objects. Mainly the place on the page where the object will be inserted. Here several values are supported, each of a different nature:

\startitemize

\item Some of the insertion places are established in relation to page elements ({\tt top, bottom inleft, inright, inmargin, margin, leftmargin, rightmargin, leftedge, rightedge, innermargin, inneredge, outeredge, inner, outer}). It must, of course, be an object that can fit in the area where it is intended to be placed and space must have been reserved for that element in the page layout. Regarding this, see section \in{}[sec:page-elements] and \in{}[sec:pagelayout].

\item Other possible insertion places are more related to the text surrounding the object, and are an indication of where the object should be placed so that the text flows around it. Fundamentally the {\tt left} and {\tt right} values.

\item The {\tt here} option is interpreted as a recommendation to keep the object at the point in the source file where it is located. This {\em recommendation} will not be respected if the pagination requirements do not allow it. This indication is reinforced if we add the {\tt force} option which means exactly that: force the insertion of the object at that point. Note that by forcing the insertion at a particular point, the object will no longer be floating.

\item Other possible options relate to the page on which the object is to be inserted: \MyKey{page} inserts it on a new page; \MyKey{opposite} inserts it on the page opposite the current one; \MyKey{leftpage} on an even page; \MyKey{rightpage} on an odd page.
  
\stopitemize

There are some options that are not related to the location of the object.
Among them:

\startitemize

\item {\tt none}: This option suppresses the title.

\item {\tt split}: This option allows the object to extend over more than one page. It must, of course, be an object that is divisible by nature, such as a table. When this option is used and the object is split, it can no longer be said to be floating.

\stopitemize

\stopsubsection

\startsubsection
  [
    reference=sec:confcaptions,
    title={Configuring floating object titles},
  ]

Unless we use the \MyKey{none} option in \tex{placefloat}, by default, floating objects are associated with a title that consists of three elements:

\startitemize

\item The name of the type of object in question This name is exactly that of the object type; so if, for example, we define a new floating object called \quotation{sequence} and we insert a \quotation{sequence} as a floating object, the title will be \quotation{Sequence 1}. Simply capitalise the name of the object.

  \startSmallPrint

    Despite what has just been said, if the main language of the document  is not English, the English name for predefined objects, like for example the \MyKey{figure} or \MyKey{table} objects, will be translated; So, for example, the \MyKey{figure} object in documents in Spanish are called \MyKey{Figura}, while the \MyKey{table} object is called \MyKey{Tabla}. These Spanish names for predefined objects can be changed with \tex{setuplabeltext} as explained in \in{section}[sec:labels].
    
  \stopSmallPrint

\item Its number. By default the objects are numbered by chapters, and so the first table in Chapter 3 will be table \quote{3.1}.

\item Its contents. Introduced as an argument of \tex{placefloat}.

\stopitemize

With \PlaceMacro{setupcaptions}\tex{setupcaptions} or \PlaceMacro{setupcaption}\tex{setupcaption[Object]} we can change the numbering system and the appearance of the title itself. The first command will affect all the titles of all objects, and the second will affect only the title of a particular type of object:

\startitemize

\item As for the numbering system, this is controlled by the {\tt number}, {\tt way}, {\tt prefixsegments} and {\tt numberconversion} options:

  \startitemize

  \item {\tt number} can adopt the {\tt yes}, {\tt no} or
    {\tt none} values and controls whether there will be a number or not.

  \item {\tt way} indicates whether the numbering will be sequential throughout the document ({\tt way=bytext}), or whether it will recommence at the beginning of each chapter ({\tt way=bychapter}) or section ({\tt way=bysection}). In the case of a restart, it is appropriate to coordinate the value of this option with the {\tt prefixsegments} option.

  \item {\tt prefixsegments} indicates if the number will have a {\em
    prefix}, and what this will be. Thus {\tt prefixsegments=chapter} causes the number of objects to always start with the chapter number, while {\tt prefixsegments=section} will precede the object number with the section number. 

  \item {\tt numberconversion} controls the kind of numeration. The values for this option can be: Arabic numbers
    (\MyKey{numbers}), lower case (\MyKey{a},
    \MyKey{characters}), upper case (\MyKey{A},
    \MyKey{Characters}), small caps \MyKey{KA}), upper case Roman numerals (\MyKey{I}, \MyKey{R},
    \MyKey{Romannumerals}), lower case (\MyKey{i},
    \MyKey{r}, \MyKey{romannumerals} or small caps
    (\MyKey{KR})).

  \stopitemize

\item The appearance of the title itself is controlled by numerous options. I will list them, but for a detailed explanation of the meaning of each one of them, I refer to \in{section}[sec:titlestyle] where the control of the appearance of the sectioning commands is explained, as the options are largely the same. The options in question are:

  \startitemize

  \item To control the format of all the elements of the title,
    {\tt style, color, command}.

  \item To control the format only of the name for the kind of object:
    {\tt headstyle, headcolor, headcommand, headseparator}.

  \item To control only the numbering format: {\tt
    numbercommand}.

  \item To control only the format of the title itself:
    {\tt textcommand}.

  \stopitemize

\item We can also control other aspects such as the distance between the different elements that make up the title, the width of the title, its placement in relation to the object, etc. I refer here to the information in \goto{\ConTeXt\ wiki}[url(wiki)] regarding the options that can be configured with this command.

\stopitemize

\stopsubsection

\startsubsection
  [
    reference=sec:startcombination,
    title={Combined insertion of two or more objects},
  ]

To insert two or more different objects in the document, such that \ConTeXt\ keeps them together and deals with them as a single object, we have the \PlaceMacro{startcombination}\tex{startcombination} environment whose syntax is:

\type{\startcombination[Ordering] ... \stopcombination}

where {\em Ordering} indicates how the objects should be ordered: if they all need to be ordered horizontally, {\em Ordering} only indicates the number of objects to be combined. But if we want to combine the objects in two or more rows, we will have to indicate the object number per row, followed by the number of rows, and separating both numbers by the * character. For example:

\starttyping
\startcombination[3*2]
  {\externalfigure[test1]}
  {\externalfigure[test2]}
  {\externalfigure[test3]}
  {\externalfigure[test4]}
  {\externalfigure[test5]}
  {\externalfigure[test6]}
\stopcombination
\stoptyping

which will produce the following alignment of images.

\startcombination[3*2]
  {\externalfigure[test1]}
  {\externalfigure[test2]}
  {\externalfigure[test3]}
  {\externalfigure[test4]}
  {\externalfigure[test5]}
  {\externalfigure[test6]}
\stopcombination

In the previous example, the images I have combined actually do not exist, which is why, instead of the images, \ConTeXt\ has generated text boxes with information about them.

See, on the other hand, how each element to be combined within \tex{startcombination}, is enclosed within curly brackets.

In fact, \tex{startcombination} not only allows us to connect and align images, but any kind of {\em box} such as texts inside a \tex{startframedtext} environment, tables, etc. To configure the combination we can use the \tex{setupcombination} command and we can also create pre-configured combinations using \PlaceMacro{definecombination}\tex{definecombination}.

\stopsubsection

\startsubsection
  [title={General configuration of floating objects}]

We have already seen that with \tex{placefloat} we can control the location of the floating object being inserted and some other details. It is also possible to configure:

\startitemize

\item The global characteristics of a particular type of floating object. This is done with  \PlaceMacro{setupfloat}\cmd{setupfloat[Name of type of floating object]}.

\item The global characteristics of all floating objects in our document.  This is done with \PlaceMacro{setupfloats}\tex{setupfloats}.  

\stopitemize

Bear in mind that in the same way that \tex{placefloat[figure]} is equivalent to \tex{placefigure}, \tex{setupfloat[figure]} is equivalent to \tex{setupfigures}, and \tex{setupfloat[table]} is equivalent to \tex{setuptables}.

Regarding the configurable options for these, I refer to the \ConTeXt\ official list of commands (\in{section}[sec:qrc-setup-en]).

\stopsubsection

\startsection
  [
    reference=sec:definefloat,
    title={Defining additional floating objects},
  ]
  \PlaceMacro{definefloat}

The \tex{definefloat} command allows us to define our own floating objects. Its syntax is:

\type{\definefloat [Singular name] [Plural name] [Configuration]}

Where the {\em Configuration} argument is an optional argument that allows us to already indicate the configuration of this new object at the time of its creation. We can also do it later with \tex{setupfloat[Name in the singular]}.

Since we are ending our introduction with this section, I am going to take advantage of it to go a little deeper into the apparent {\em jungle} of \ConTeXt\ commands which, once understood, is not so much of a {\em jungle} but is, in fact, quite rational.

Let's start by asking ourselves what a floating object really is for\ConTeXt, the answer being that it is an object with the following characteristics:

\startitemize

\item That it has a certain margin of freedom with respect to its location on the page.

\item That is has a {\em list} associated with it, that allows it to number these kinds of objects and, eventually, to generate an index of them.

\item That it has a title

\item That, when the object can really float, it must be treated as an inseparable unit, meaning (in \TeX\ terminology) {\em enclosed in a box}.

\stopitemize

In other words, the floating object is actually made up of three elements: the object itself, the list associated with it, and the title. To control the object itself we only need one command to set up its location and another to insert the object into the document; to set up the list aspects, general list control commands are sufficient, and to set up the title aspects, the general title control commands.

And this is where the genius of \ConTeXt\ comes in: a simple command to control floating objects (\tex{setupfloats}), and a simple command to insert floating objects: \tex{placefloat}, could have been designed:  but what \ConTeXt\ does is to:

\startitemize[n]

\item Design a command to link a name to a specific floating object configuration. This is the \tex{definefloat} command, which does not actually link one name, but two names, one in the singular and one in the plural.

\item Create, together with the global floating objects configuration command, a command that allows us to configure only a specific type of object: \tex{setupfloat[Object]}.

\item Add to the floating object location command, (\tex{placefloat}), an argument that allows us to differentiate between one or other type: (\tex{placefloat[Object]}).

\item Create commands, including the object name, for all actions of a floating object. Some of these commands (which are actually clones of other more general commands) will use the object's name in the singular and others will use it in the plural.
  
\stopitemize

Therefore, when we create a new floating object and tell \ConTeXt\ what its name is in the singular and the plural, \ConTeXt:

\startitemize

\item Reserves a space in memory to store the specific configuration of that object type.

\item Creates a new list with the singular name of that object type, since floating objects are associated with a list.

\item Creates a new kind of \quotation{title} linked to this new object type, in order to maintain a customised configuration of these titles.

\item And finally, it creates a group of new commands specific to that new object type, whose name is actually a synonym for the more general command.
  
\stopitemize

In \in{table}[tbl:floatcommands] we can see the commands that are automatically created when we define a new floating object, as well as the more general commands they are synonyms of:

\placetable
  [here]
  [tbl:floatcommands]
  {Commands that are automatically created when a new floating object is created}
{\switchtobodyfont[small]
\starttabulate[|lT|lT|lT|]
\HL
\NC{\bf\rm Command}
\NC{\bf\rm Synonym of}
\NC{\bf\rm Example}
\NR
\HL
\NC\backslash completelistof<PluralName>
\NC\backslash completelist[PluralName]
\NC\backslash completelistoffigures
\NR
\NC\backslash place<SingularName>
\NC\backslash placefloat[SingularName]
\NC\backslash placefigure
\NR
\NC\backslash placelistof<PluralName>
\NC\backslash placelist[PluralName]
\NC\backslash placelistoffigures
\NR
\NC\backslash setup<SingularName>
\NC\backslash setupfloat[SingularName]
\NC\backslash setupfigure
\NR
\HL
\stoptabulate
}

\startSmallPrint

  Actually, some additional commands are created which are synonymous with the previous ones and since I have not included them in the explanation of the chapter, I have omitted them from \in{table}[tbl:floatcommands]: \tex{start<NameSingular>}, \tex{start<NameSingular>text} and  \tex{startplace<NameSingular>}.

\stopSmallPrint

I have used the command used for images as an example of the commands created when defining a new floating object; and I did so because images, like tables and the rest of the floats predefined by \ConTeXt, are actual cases of \tex{definefloat}:

\starttyping
\definefloat[chemical][chemicals]
\definefloat[figure][figures]
\definefloat[table][tables]
\definefloat[intermezzo][intermezzi]
\definefloat[graphic][graphics]
\stoptyping

Finally, we see that in reality \ConTeXt\ in no way controls any kind of material included in each particular floating object; it presumes that this is the author's job. This is why we can also insert text with the \tex{placefigure} or \tex{placetable} commands. However, the text that is input with \tex{placefigure} is included in the list of images, and if input with \tex{placetable}, in the list of tables.

\stopsection

\stopchapter

\stopcomponent

%%% Local Variables:
%%% mode: ConTeXt
%%% mode: auto-fill
%%% coding: utf-8-unix
%%% TeX-master: "../introCTX_fra.tex"
%%% End:
%%% vim:set filetype=context tw=75 : %%%

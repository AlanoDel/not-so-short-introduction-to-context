%%% Fichero:     b02_FirstDoc.mkiv
%%% Autor:       Joaquín Ataz López
%%% Inicio:      Marzo 2020
%%% Terminación: Abril 2020
%%% Contenido:   En una introducción me parece oportuno empezar
%%%              poniendo un ejemplo. En «The TeX Book» Knuth empieza
%%%              mostrando algunos ejemplos de la ejecución de tex. El
%%%              problema es que ConTeXt al compilar no espera que el
%%%              usuario resuelva los posibles errores _durante la
%%%              compilación_, por lo que me quedo casi sin materia
%%%              para el capítulo. Tampoco he conseguido averiguar
%%%              para qué sirven gran parte de las opciones de
%%%              ConTeXt. Y en cuanto a los errores y su tratamiento,
%%%              salvo en el caso de comandos inexistentes, nunca
%%%              estoy seguro de si un fallo generará o no un error de
%%%              compilación. Según he podido comprobar las opciones
%%%              inexistentes o con valores inadecuados nunca generan
%%%              errores de compilación, y los entornos no
%%%              oportunamente cerrados, sólo lo hacen a veces.
%%%              Para el contenido de este capítulo me he inspirado un
%%%              poquito en el capítulo II de «El libro de LaTeX», de
%%%              Bernardo, José Manual y compañía.
%%%
%%% Editado: Emacs + AuTeX - Y a veces con vim + context-plugin
%%%


\environment introCTX_env

\startcomponent b02_PremierDoc

\startchapter
  [reference=cap:firstdoc,
   title=Notre premier fichier source]

\TocChap

Ce chapitre est consacré à la mise en oeuvre de notre première expérience. Il expliquera la structure de base d'un document \ConTeXt\ ainsi que les meilleures stratégies pour faire face aux éventuelles erreurs.


\startsection
  [title=Préparation de l'expérience \\ outils nécessaires]

Pour écrire et compiler un premier fichier source, nous devrons avoir les outils
suivants installés sur notre système.

\startitemize[n]

\item {\bf un éditeur de texte} pour écrire notre fichier de test. Il existe de nombreux éditeurs de texte et il est difficilement concevable qu'un système informatique n'en ait pas déjà un d'installé. Nous pouvons utiliser n'importe lequel d'entre eux : il existe des systèmes simples, d'autres complexes, des puissants, d'autres simples, des payants, des gratuits, des spécialisés pour \TeX, des généralistes, etc. Si nous avons l'habitude d'utiliser un éditeur spécifique, il est préférable de poursuivre avec lui ; si nous n'avons pas l'habitude de travailler avec des éditeurs de texte, mon conseil est, dans un premier temps, de choisir un éditeur simple, afin de ne pas ajouter à la difficulté de l'apprentissage de \ConTeXt\ la difficulté d'apprendre à utiliser l'éditeur. Bien qu'il soit également vrai que, souvent, les programmes les plus difficiles à maîtriser sont aussi les plus puissants.


J'ai écrit ce texte avec GNU Emacs, qui est l'un des éditeurs généralistes les plus puissants et les plus polyvalents qui existent ; il est vrai qu'il a ses particularités et aussi ses détracteurs, mais en général il y a plus de \quotation{\em Emacs Lovers} que de \quotation{\em Emacs Haters}. Pour travailler avec des fichiers \TeX\ ou l'un de ses dérivés, il existe une extension pour GNU Emacs, appelée AucTeX, qui fournit à l'éditeur quelques fonctionnalités supplémentaires très intéressantes, même si AucTeX est plus développé pour \LaTeX\ que pour \ConTeXt. GNU Emacs en combinaison avec AucTeX peut être une bonne option si l'on ne sait pas quel éditeur choisir ; tous deux sont des programmes à code source ouvert, et ils sont disponibles pour tous les systèmes d'exploitation. En fait, dire que GNU Emacs est un {\em logiciel libre} est un euphémisme, car ce programme incarne mieux que tout autre l'esprit de ce qu'est et signifie le {\em logiciel libre}. Après tout, son principal développeur était {\sc Richard Stallman}, fondateur et idéologue du projet GNU et de la {\em Free Software Foundation}.

En plus de GNU Emacs + AucTeX, {\em Scite} et {\em TexWorks} sont d'autres bonnes options si vous ne savez pas quel éditeur choisir. Le premier, bien qu'il s'agisse d'un éditeur à usage généraliste, non conçu spécifiquement pour travailler avec des fichiers \ConTeXt, est facilement personnalisable et, comme c'est l'éditeur généralement utilisé par les développeurs de \ConTeXt\, \suite- contient les fichiers de configuration de cet éditeur conçus et utilisés par {\sc Hans Hagen} lui-même. {\em TexWorks} est, quant à lui, un éditeur de texte rapide, spécialisé dans le traitement des fichiers \TeX\ et de ses langages dérivés. Il est assez facile à configurer pour fonctionner avec \ConTeXt\ et \suite- prévoit également de fournir des fichiers configurations.

Qu'il s'agisse d'un éditeur ou d'un autre, ce qu'il ne faut pas faire, c'est utiliser, comme éditeur de texte, un {\em logiciel de traitement de texte} tel que, par exemple, OpenOffice Writer ou Microsoft Word. Ces programmes, qui sont à mon avis trop lents et trop lourds, peuvent certes enregistrer un fichier en \quotation{texte pur}, mais ils ne sont pas conçus pour cela et nous finirions probablement par enregistrer notre fichier dans un format binaire incompatible avec \ConTeXt.

\item {\bf Une distribution \ConTeXt} pour traiter notre fichier de test. S'il existe déjà une installation \TeX\ (ou \LaTeX) sur votre système, il est possible qu'une version de \ConTeXt\ soit déjà installée. Pour le vérifier, il suffit d'ouvrir un terminal et de taper dans celui-ci


\startterminal
$ context --version
\stopterminal

\startDemoC
$ context --version
\stopDemoC

  \startSmallPrint

    {\bf NOTA} ceux pour qui l'utilisation du terminal est nouvelle, les deux premiers caractères que j'ai indiqué ( \quotation{\type{$}}) n'ont pas à être tappé dans le terminal par l'utilisateur. Je les utilise pour représenter ce qu'on appelle l'{\em invite} du terminal (le prompt en anglais),
qui indique que le terminal attend nos instructions.

\stopSmallPrint

Si une version de \ConTeXt\ est déjà installée, vous devriez obtenir un résultat similaire au suivant :

\startterminal
$ context --version
mtx-context     | ConTeXt Process Management 1.03
mtx-context     |
mtx-context     | main context file: /usr/share/texmf/tex/context/base/mkiv/context.mkiv
mtx-context     | current version: 2020.03.10 14:44
mtx-context     | main context file: /usr/share/texmf/tex/context/base/mkiv/context.mkxl
mtx-context     | current version: 2020.03.10 14:44
\stopterminal

\startDemoC
$ context --version
mtx-context     | ConTeXt Process Management 1.03
mtx-context     |
mtx-context     | main context file: /usr/share/texmf/tex/context/base/mkiv/context.mkiv
mtx-context     | current version: 2020.03.10 14:44
mtx-context     | main context file: /usr/share/texmf/tex/context/base/mkiv/context.mkxl
mtx-context     | current version: 2020.03.10 14:44
\stopDemoC


Dans la dernière ligne, nous sommes informés de la date à laquelle la version installée a été publiée. Si elle est très ancienne, nous devons le mettre à jour ou installer une nouvelle version. Je recommande d'installer la distribution appelée \suite- dont les instructions d'installation se trouvent sur le \goto{wiki de \ConTeXt}[url(wiki)]. Les indications sont également incluses dans l'\in{annexe}[instalación_suite].


\item {\bf Un programme de visualisation de fichier PDF} afin de visualiser le résultat de notre expérience à l'écran. Sur les systèmes Windows et Mac OS, la visionneuse omniprésente est Adobe Acrobat Reader. Il n'est pas installé par défaut (ou ne l'était pas lorsque j'ai cessé d'utiliser Microsoft Windows, il y a plus de 15 ans). L'installation se fait la première fois que vous essayez d'ouvrir un fichier PDF, il est donc généralement déjà installé. Sur les systèmes Linux/Unix, il n'y a pas de version mise à jour d'Acrobat Reader, mais il n'est pas nécessaire non plus, car il existe littéralement des dizaines de très bons visualisateurs de PDF gratuits. De plus, dans ces systèmes, il y en a presque toujours un installé par défaut. Mon préféré, pour sa vitesse et sa facilité d'utilisation, est MuPDF ; bien qu'il ait quelques inconvénients comme, par exemple, de ne pas montrer l'index des signets, de ne pas permettre les recherches de texte qui incluent des caractères inexistants dans l'alphabet anglais (comme les voyelles accentuées ou les eñes) ou de ne pas permettre de sélectionner le texte, d'envoyer le document à l'imprimante ; c'est juste un visualisateur, mais très rapide et très confortable. Lorsque j'ai besoin de certains de ces fonctionnalités absentes de MuPDF, j'utilise généralement Okular ou qPdfView. Mais, encore une fois, la question est une affaire de goût : chacun peut choisir celui qu'il préfère.

\stopitemize

Nous pouvons choisir l'éditeur, nous pouvons choisir le visualisateur de PDF, nous pouvons choisir la distribution \ConTeXt... Bienvenue dans le monde du {\em logiciel libre} !


\stopsection

\startsection
  [title=L'experience elle-même]

\startsubsubsubject
  [title=Rédaction du fichier source]


Si nous disposons déjà des outils mentionnés dans la section précédente, nous devons ouvrir notre éditeur de texte et créer un fichier appelé \quotation{la-maison-sur-le-port.tex} pour notre exemple. Comme contenu du fichier, nous allons écrire ce qui suit :


\typefile[code]{../img/la-maison-sur-le-port.tex}

\startDemoI
% Première ligne du document

\mainlanguage[fr]    % Langue français

\setuppapersize[S5]  % Format du papier

\setupbodyfont       % Police = Latin Modern, 12 points
  [modern,18pt]

\setuphead           % Format des titres de chapitre
  [chapter]
  [style=\bfc]

\starttext           % Début du contenu du document

\startchapter
  [title=La maison sur le port]

Il y avait des    chansons
Les hommes        venaient y boire et rêver
Dans la maison    sur le port
Où les filles     riaient fort
Où le vin faisait chanter chanter chanter

Les pêcheurs      vous le diront
Ils y venaient    sans façon
Avant de partir   retirer leurs filets
Ils venaient      se réchauffer près de nous
Dans la maison    sur le port

\stopchapter

\stoptext            % Fin du document
\stopDemoI

\attachment
  [file=../img/la-maison-sur-le-port.tex,
    title={View Source Code},
    name=la-maison-sur-le-port.tex,
    author=Test]


Durant l'écriture, certains aspects n'ont aucune importance, notamment si vous ajoutez ou supprimez des espaces blancs ou des sauts de ligne. Ce qui est important, c'est que chaque mot suivant le caractère \quotation{\tt \backslash} soit écrit très exactement de la même façon qu'il l'est dans l'exemple, ainsi que le contenu des crochets. Il peut y avoir des variations dans le reste.


\stopsubsubsubject

\startsubsubsubject
  [title=Encodage du fichier]

  Une fois le texte précédent écrit, nous enregistrons le fichier sur le disque. Cela n'est dorénavant qu'une vérification à faire, mais il faut nous assurer que l'encodage du fichier est bien UTF-8. Cet encodage est aujourd'hui la norme et constitue l'encodage par défaut sur la plupart des systèmes Linux/Unix. Néanmois, je ne sais pas si c'est la même chose sous Mac OS ou Windows et il est encore tout à fait possible que l'encodage ANSI soit utilisé. En tout cas, si nous ne sommes pas sûrs, depuis l'éditeur de texte lui-même, nous pouvons voir avec quel encodage le fichier sera enregistré et, si nécessaire, le modifier. La manière de procéder dépend, bien entendu, de l'éditeur avec lequel nous travaillons. Dans GNU Emacs, par exemple, en appuyant simultanément sur les touches CTRL-X puis Return suivi de \quotation{f}, dans la dernière ligne de la fenêtre (que GNU Emacs appelle mini-buffer) un message apparaîtra nous demandant un nouvel encodage et nous informant de l'encodage actuel. Dans les autres éditeurs, nous pouvons généralement accéder à l'encodage dans le menu \quotation{Enregistrer sous}.

Après avoir vérifié que l'encodage est correct et enregistré le fichier sur le disque, nous fermerons l'éditeur pour nous concentrer sur l'analyse de ce que nous avons écrit.

\stopsubsubsubject

\startsubsubsubject
  [title=Regardons le contenu de notre premier fichier source pour \ConTeXt]

  La première ligne commence par le caractère \quotation{\tt \%}. C'est un caractère réservé qui indique à \ConTeXt\ de ne pas traiter le texte qui le suit et ce jusqu'à la fin de la ligne sur laquelle il se trouve. Cette fonctionnalité est utilisée pour écrire des commentaires dans le fichier source que seul l'auteur pourra lire, car ils ne seront pas incorporés au document final. Dans cet exemple, je l'ai par exemple utilisé pour attirer l'attention sur certaines lignes, en expliquant ce qu'elles font.

  Les lignes suivantes commencent par le caractère \quotation{\tt \backslash} qui est un autre des caractères réservés de \ConTeXt\ et indique que ce qui suit est le nom d'une commande. L'exemple comprend plusieurs commandes couramment utilisée dans un fichier source \ConTeXt\ : la langue dans laquelle le document est écrit, le format du papier, la police à utiliser dans le document et la mise en forme appliquée aux titres de chapitres. Plus tard, dans d'autres chapitres, nous détailleront ces commandes, pour le moment je veux juste que le lecteur voit à quoi elles ressemblent : elles commencent toujours par le caractère \quotation{\tt \backslash}, suivi du nom de la commande, et ensuite, entre parenthèses ou accolades, selon le cas, les données dont la commande a besoin pour produire ses effets. Entre le nom de la commande et les crochets ou accolades qui l'accompagnent, il peut y avoir des espaces vides ou des sauts de ligne. C'est à l'auteur de choisir la façon dont le code source est le plus clair et lisible.

Sur la 9\high{ème} ligne de notre exemple (je ne compte que les lignes qui ont du texte) se trouve la commande importante \tex{starttext} : elle indique à \ConTeXt\ que le contenu du document commence à partir de cet endroit; et à la dernière ligne de notre exemple, nous voyons la commande \tex{stoptext} qui indique la fin du contenu. Tout ce qui suit cette dernière commande ne sera pas traité. Ce sont deux commandes très importantes sur lesquelles je reviendrai très bientôt. Entre les deux se trouve donc le contenu à proprement parler de notre document qui, dans notre exemple, consiste en la première strophe de la chanson "\quotation{La maison sur le port}, dont les paroles sont de {\sc Amalia Rodrigues}, et qui a été reprise notamment par {\sc Sanseverino}. Je l'ai écrit en prose afin de mieux observer le formatage des paragraphes effectué par \ConTeXt.


\stopsubsubsubject

\startsubsubsubject
  [title=Traitement du document source, ou compilation]

 \index{compilation}

Pour l'étape suivante, après s'être assuré que \ConTeXt\ a été correctement installé dans notre système, nous devons ouvrir un terminal dans le répertoire où se trouve notre fichier \quotation{la-maison-sur-le-port.tex}.

\startSmallPrint


De nombreux éditeurs de texte vous permettent de compiler le document sur lequel vous travaillez sans ouvrir un terminal. Cependant, la procédure {\em canonique} pour traiter un document avec \ConTeXt\ implique de le faire à partir d'un terminal, en exécutant directement le programme. Je vais procéder de cette manière (ou supposer qu'il en est ainsi) tout au long de ce document pour plusieurs raisons ; la première est que je n'ai aucun moyen de savoir avec quel éditeur chaque lecteur travaille. Mais le plus important est que, depuis le terminal, nous aurons accès à la {\em sortie} de \MyKey{context}, c'est à dire les messages émis par le programme.

\stopSmallPrint

Si la distribution \ConTeXt\ que nous avons installée est \suite-, nous devons tout d'abord exécuter le {\em script} qui indique au terminal les chemins et l'emplacement des fichiers dont \ConTeXt\ a besoin pour fonctionner. Sur les systèmes Linux/Unix, cela se fait en tapant la commande suivante :

% TODO à revoir avec LMTX

\startterminal
$ source ~/context/tex/setuptex
\stopterminal

\startDemoC
$ source ~/context/tex/setuptex
\stopDemoC


en supposant que nous avons installé \ConTeXt\ dans un répertoire appelé \MyKey{context}.

\startSmallPrint

En ce qui concerne l'exécution du {\em script} qui vient d'être mentionné, voir ce qui est dit dans l'\in{annexe}[instalación_suite] concernant l'installation de \suite-.

\stopSmallPrint

Une fois que les variables nécessaires à l'exécution de \MyKey{context} ont été chargées en mémoire, nous pouvons l'exécuter. Cela se fait en tapant dans le terminal :

\startterminal
$ context la-maison-sur-le-port
\stopterminal

\startDemoC
$ context la-maison-sur-le-port
\stopDemoC


Notez que bien que le fichier source s'appelle \quotation{la-maison-sur-le-port.tex} dans l'appel à \MyKey{context} nous avons omis l'extension du fichier. Si nous avions appelé au fichier source, par exemple, \MyKey{la-maison-sur-le-port.mkiv} (ce que je fais habituellement pour savoir que ce fichier est écrit pour Mark~IV), il aurait fallu indiquer expressément l'extension du fichier à compiler en tapant \MyKey{context la-maison-sur-le-port.mkiv}.


Après avoir exécuté \MyKey{context} dans le terminal, plusieurs dizaines de lignes s'affichent à l'écran, informant de ce que fait \ConTeXt. Les informations s'affichent à une vitesse impossible à suivre par un être humain, mais ne vous inquiétez pas, car en plus de l'écran, ces informations sont également stockées dans un fichier auxiliaire, avec l'extension \MyKey{.log} qui est généré avec la compilation et que nous pourrons consulter tranquililement plus tard si nécessaire.

Après quelques secondes, si nous avons bien écrit le texte de notre fichier source, sans faire d'erreur grave, l'émission de messages vers le terminal se terminera. Le dernier des messages nous informera du temps nécessaire à la compilation. La première fois qu'un document est compilé, cela prend toujours un peu plus de temps, car \ConTeXt\ doit construire à partir de zéro certains fichiers contenant les informations de notre document. Par la suite ils seront juste réutilisés et complétés pour les compilations suivantes qui iront plus vite. Le message du temps passé indique que la compilation est terminée. Si tout s'est bien passé, trois fichiers supplémentaires apparaîtront dans le répertoire où nous avons exécuté \MyKey{context}~:


\startitemize[packed]

\item la-maison-sur-le-port.pdf
\item la-maison-sur-le-port.log
\item la-maison-sur-le-port.tuc

\stopitemize


Le premier est le résultat de notre traitement, c'est-à-dire : le fichier PDF déjà formaté. Le deuxième est le fichier  dans lequel sont stockées toutes les informations qui ont été affichées à l'écran pendant la compilation ; le troisième est un fichier auxiliaire que \ConTeXt\ génère pendant la compilation et qui est utilisé pour construire les index et les références croisées. Pour le moment, si tout a fonctionné comme prévu, nous pouvons supprimer les deux fichiers (\MyKey{la-maison-sur-le-port.log} et \MyKey{la-maison-sur-le-port.tuc}). S'il y a eu un problème, les informations contenues dans ces fichiers peuvent nous aider à localiser la source du problème et à déterminer comment le résoudre.


Si nous n'avons pas obtenu ces résultats, c'est probablement dû à un ou plusieurs de points qui suivent :


\startitemize [packed]

\item soit nous n'avons pas installé correctement notre distribution \ConTeXt, auquel cas en tapant la commande  \MyKey{context} dans le terminal, un message \quotation{commande inconnue} sera apparu.

\item soit notre fichier n'a pas été encodé en UTF-8 et cela a généré une erreur de compilation.

\item soit peut-être que la version de \ConTeXt\ installée sur notre système est Mark~II. Dans cette version, vous ne pouvez pas utiliser l'encodage UTF-8 sans l'indiquer explicitement dans le fichier source lui-même. Nous pourrions corriger le fichier source pour qu'il compile bien, mais, puisque cette introduction se réfère à Mark~IV, cela n'a pas de sens de continuer à travailler avec Mark~II : la meilleure chose à faire est d'installer \suite-.

\item Soit nous avons fait une erreur en écrivant dans le fichier source le nom de certaines commandes ou leurs données associées.

\stopitemize


\startSmallPrint
  Si après l'exécution de  \MyKey{context}, le terminal a commencé à émettre des messages, mais s'est ensuite arrêté sans que le {\em prompt} ne réapparaisse, avant de continuer, il faut appuyer sur CTRL-X pour interrompre l'exécution de \ConTeXt\ qui a été interrompue par l'erreur.

\stopSmallPrint

En cas de problème, nous devrons donc vérifier chacune de ces possibilités, et les corriger, jusqu'à ce que la compilation se déroule correctement.


\placefigure
  [here]
  [la-maison-sur-le-port]
  {La maison sur le port}
  {\midaligned{\framed{\externalfigure[la-maison-sur-le-port.pdf][width=13cm]}}}


La \in{figure}[la-maison-sur-le-port] montre le contenu de \quotation{la-maison-sur-le-port.pdf}. Nous pouvons voir que \ConTeXt\ a numéroté la page, numéroté le chapitre et écrit le texte dans la police indiquée. Il a également réparti le mot  \quotation{venaient} entre la sixième et la septième ligne, ainsi que le mot  \quotation{maison} la septième et la guitième ligne. \ConTeXt, par défaut, active la césure (division syllabique) des mots afin de répartir les blancs (les espaces vides entre les mots) de façon la plus homogène possible. C'est pourquoi il est si important d'informer \ConTeXt\ de la langue du document, car les modèles de césure varient selon la langue. Dans notre exemple, c'est l'objectif de la première commande du fichier source (\tex{mainlanguage[fr]}).


En bref : \ConTeXt\ a transformé le fichier source et a généré un fichier dans lequel nous avons un document formaté selon les instructions qui étaient incluses dans le fichier source. Les commentaires en ont disparu et, en ce qui concerne les commandes, ce que nous avons maintenant n'est pas leur nom, mais le résultat de leur application par \ConTeXt.

\stopsubsubsubject

\stopsection

\startsection
  [title=La structure de notre fichier d'exemple]
\index{préambule}

Dans un projet aussi simple que notre exemple, développé dans un seul fichier source, la structure de celui-ci est très simple et est marquée par les commandes \tex{starttext} ... \tex{stoptext}. Tout ce qui se trouve entre la première ligne du fichier et la commande \tex{starttext} constitue le {\em préambule}. Le contenu du document lui-même est inséré entre les commandes \tex{starttext} et \tex{stoptext}. Dans notre exemple, le préambule comprend quatre commandes de configuration globale : une pour indiquer la langue de notre document (\tex{mainlanguage}), une autre pour indiquer la taille des pages (\tex{setuppapersize}) qui dans notre cas est \quotation{S5}, représentant les proportions d'un écran d'ordinateur, une troisième commande (\tex{setupbodyfont})
qui nous permet d'indiquer la police de caractère et sa taille, et une quatrième (\tex{setuphead}) qui nous permet de configurer l'apparence des titres des chapitres.

Le corps du document est encadré par les commandes \tex{starttext} et \tex{stoptext}. Ces commandes indiquent, respectivement, le point de départ et le point final du texte à traiter : entre elles, nous devons inclure tout le texte que nous voulons que \ConTeXt\ traite, ainsi que les commandes qui ne doivent pas affecter le document entier mais seulement des fragments de celui-ci. Pour le moment, nous devons supposer que les commandes \tex{starttext} et \tex{stoptext} sont obligatoires dans tout document ConTEXt, bien que plus tard, en parlant des projets multifichiers (\in{section}[sec-proyectos]) nous verrons qu'il y a quelques exceptions.


\stopsection

\startsection
  [title=Quelques détails supplémentaires sur la façon d'exécuter \MyKey{context}]

La commande \MyKey{context}  avec laquelle nous avons procédé au traitement de notre premier fichier source est, en fait, un {\em script} {\sc Lua}, c'est-à-dire : un petit programme {\sc Lua} qui, après avoir effectué quelques vérifications et opérations, appelle \LuaTeX pour traiter le fichier source.

Nous pouvons appeler \MyKey{context} avec plusieurs options. Les options sont saisies immédiatement après le nom de la commande, précédées de deux traits d'union. Si nous voulons saisir plus d'une option, nous les séparons par un espace blanc. L'option \MyKey{help} nous donne une liste de toutes les options, avec une brève explication de chacune d'elles :

\startterminal
$ context --help
\stopterminal

\startDemoC
$ context --help
\stopDemoC

Parmi les options les plus intéressantes, citons les suivantes :


\description{\tt interface} Comme je l'ai dit dans le chapitre d'introduction, l'interface de \ConTeXt\ est traduite en plusieurs langues. Par défaut, c'est l'interface anglaise qui est utilisée, mais cette option nous permet de lui demander d'utiliser la version néerlandaise (nl), française (fr), italienne (it), allemande (de) ou roumaine (ro).

\description{\tt purge, purgeall} Supprime les fichiers auxiliaires générés pendant le traitement.

\description{\tt result=Name} indique le nom que doit porter le fichier PDF résultant. Par défaut, ce sera le même que le fichier source à traiter, avec l'extension \MyKey{.pdf}.

\description{\tt usemodule=list} Charge les modules qui sont indiqués avant d'exécuter \ConTeXt\ (un module est une extension de \ConTeXt, qui ne fait pas partie de son noyau, et qui lui fournit une utilité supplémentaire).

\description{\tt useenvironment=list} Charge les fichiers d'environnement qui sont spécifiés avant de lancer \ConTeXt\ (un fichier d'environnement est un fichier contenant des instructions de configuration).

\description{\tt version} indique la version de \ConTeXt.

\description{\tt help} affiche des informations d'aide sur les options du programme.

\description{\tt noconsole} Supprime l'envoi de messages à l'écran pendant la compilation. Toutefois, ces messages seront toujours enregistrés dans le fichier \MyKey{.log}.

\description{\tt nonstopmode} Exécute la compilation sans s'arrêter sur les erreurs. Cela ne signifie pas que l'erreur ne se produira pas, mais que lorsque \ConTeXt\ rencontre une erreur, même si elle est récupérable, il continuera la compilation jusqu'à ce qu'elle se termine ou jusqu'à ce qu'il rencontre une erreur irrécupérable.

\description{\tt batchmode} Il s'agit d'une combinaison des deux options précédentes. Il fonctionne sans interruption et omet les messages à l'écran.

Pour les premières utilisation et pour l'apprentissage de \ConTeXt, je ne pense pas que ce soit une bonne idée d'utiliser les trois dernières options, car lorsqu'une erreur se produit, nous n'aurons aucune idée de l'endroit où elle se trouve ou de ce qui l'a produite. Et, croyez-moi chers lecteurs, tôt ou tard, une erreur de compilation se produira.


\stopsection

\startsection
  [title=Traitement des erreurs]

En travaillant avec \ConTeXt, il est inévitable que, tôt ou tard, des erreurs se produisent dans la compilation. En gros, nous pouvons regrouper les erreurs dans l'une des quatre catégories suivantes~:

\startitemize[n]

\item {\bf Erreurs de frappe}. Elles se produisent lorsque nous orthographions mal le nom d'une commande. Dans ce cas, nous envoyons au compilateur une commande qu'il ne comprend pas. Par exemple, lorsque, au lieu d'écrire la commande \tex{TeX}, nous écrivons \tex{Tex} avec le \quotation{X} final en minuscule, puisque \ConTeXt\ fait la différence entre les majuscules et les minuscules et considère donc que \quotation{TeX} et \quotation{Tex} sont des mots différents ; ou si les options utilisée pour une commande, au lieu de les mettre entre crochets, sont mises entre accolades, ou si nous essayons d'utiliser un des caractères réservés comme s'il s'agissait d'un caractère normal, etc.

\item {\bf Erreurs par omission}. Dans \ConTeXt\ il y a des instructions qui démarrent une tâche, dont il faut indiquer explicitement quand la fermer ; comme le caractère réservé \quotation{\$} qui active le mode mathématique, qui est maintenu jusqu'à ce qu'on le désactive, et si on oublie de le désactiver, une erreur sera générée dès qu'on trouvera un texte ou une instruction qui n'a pas de sens dans le mode mathématique. Il en va de même si nous commençons un bloc de texte au moyen du caractère réservé \quotation{\{} ou d'une commande \tex{startUnTruc} et que, par la suite, la fermeture explicite n'est pas trouvée (\quotation{\}} ou \tex{stopUnTruc}).

\item {\bf Erreurs de conception}. J'appelle ainsi les erreurs qui se produisent lorsque vous appelez une commande qui nécessite certains arguments, sans les fournir, ou lorsque la syntaxe d'appel de la commande n'est pas correcte.

\item {\bf Erreurs situationnelles}. Certaines commandes sont destinées à ne fonctionner que dans certains contextes ou environnements, et sont donc inconnues en dehors de ceux-ci. Cela se produit, en particulier, avec le mode mathématique : certaines commandes \ConTeXt\ ne fonctionnent que lors de l'écriture de formules mathématiques et si elles sont appelées dans d'autres contextes, elles génèrent une erreur.

\stopitemize


Que faire lorsque  \MyKey{context} nous avertit, pendant la compilation, qu'une erreur s'est produite ? La première chose est, évidemment, d'identifier quelle est l'erreur. Pour ce faire, nous devrons parfois analyser le fichier \MyKey{.log} généré pendant la compilation ; mais encore plus souvent il suffira de remonter dans les messages produits par \MyKey{context} dans le terminal où il est exécuté.

\placefloat
  [here]
  [middle]
  [msgerror]
  {Sortie d'affichage en cas d'erreur de compilation}
  {
\startterminal[numbering=]
tex error       > tex error on line 14 in file la-maison-sur-le-port_bug.tex: ! Undefined control sequence

l.14 \startext
                       % Début du contenu du document

 4
 5     \setuppapersize[S5]  % Format du papier
 6
 7     \setupbodyfont       % Police = Latin Modern, 12 points
 8       [modern,18pt]
 9
10     \setuphead           % Format des titres de chapitre
11       [chapter]
12       [style=\bfc]
13
14 >>  \startext           % Début du contenu du document
15
16     \startchapter
17       [title=La maison sur le port]
18
19     Il y avait des    chansons
20     Les hommes        venaient y boire et rêver
21     Dans la maison    sur le port
22     Où les filles     riaient fort
23     Où le vin faisait chanter chanter chanter
24

mtx-context     | fatal error: return code: 256
\stopterminal
}

\startDemoC
tex error       > tex error on line 14 in file la-maison-sur-le-port_bug.tex: ! Undefined control sequence

l.14 \startext
                       % Début du contenu du document

 4
 5     \setuppapersize[S5]  % Format du papier
 6
 7     \setupbodyfont       % Police = Latin Modern, 12 points
 8       [modern,18pt]
 9
10     \setuphead           % Format des titres de chapitre
11       [chapter]
12       [style=\bfc]
13
14 >>  \startext           % Début du contenu du document
15
16     \startchapter
17       [title=La maison sur le port]
18
19     Il y avait des    chansons
20     Les hommes        venaient y boire et rêver
21     Dans la maison    sur le port
22     Où les filles     riaient fort
23     Où le vin faisait chanter chanter chanter
24

mtx-context     | fatal error: return code: 256
\stopDemoC


Par exemple, si dans notre fichier de test, \MyKey{la-maison-sur-le-port.tex}, par erreur, au lieu de \tex{starttext} nous avions écrit \tex{startext} (avec un seul \quotation{t}), ce qui, par ailleurs, est une erreur très courante, lors de l'exécution de \MyKey{context la-maison-sur-le-port}, lorsque la compilation était arrêtée, dans l'écran du terminal nous pouvions voir l'information montrée dans la \in{figure}[msgerror].

  Nous pouvons y voir les lignes de notre fichier source numérotées, et à l'une d'entre elles, dans notre cas la ligne 14, entre le numéro et le texte de la ligne le compilateur a ajouté \quotation{>>} pour indiquer que c'est dans cette ligne qu'il a trouvé l'erreur. Le numéro de la ligne est également indiqué plus haut, avnat l'affichage des lignes, dans une ligne commençant par \MyKey{tex error}. Le fichier \MyKey{la-maison-sur-le-port.log} nous donnera plus d'indices. Dans notre exemple, il ne s'agit pas d'un très gros fichier, car la source que nous compilions est très petite ; dans d'autres cas, il peut contenir une quantité écrasante d'informations. Mais nous devons nous y plonger. Si nous ouvrons  \MyKey{la-maison-sur-le-port.log} avec un éditeur de texte, nous verrons que ce fichier enregistre tout ce que fait \ConTeXt. Nous devons y chercher une ligne qui commence par un avertissement d'erreur \quotation{\tt tex error}, pour cela nous pouvons utiliser la fonction de recherche de texte de l'éditeur. Nous trouveerons les lignes d'erreur suivantes~:

\startterminal[numbering=]
tex error       > tex error on line 14 in file la-maison-sur-le-port_bug.tex: ! Undefined control sequence

l.14 \startext
                       % Début du contenu du document
\stopterminal

\startSmallPrint

{\bf Note~:} La première ligne informant de l'erreur, dans le fichier  \MyKey{la-maison-sur-le-port.log} est très longue. Pour que cela soit présentable ici, en tenant compte de la largeur de la page, j'ai supprimé une partie du chemin indiquant l'emplacement du fichier.
\stopSmallPrint


Si nous prêtons attention aux trois lignes du message d'erreur, nous voyons que la première nous indique à quel numéro de ligne l'erreur s'est produite (ligne 14) et de quel type d'erreur il s'agit : \quotation{Undefined control sequence}, ou, ce qui revient au même : \quotation{Unknown control sequence}, c'est-à-dire une commande inconnue. Les deux lignes suivantes du fichier journal nous montrent la ligne 14, qui commence à l'endroit où l'erreur s'est produite. Donc il n'y a pas de doute, l'erreur est dans \tex{startext}. Nous le lirons attentivement et, avec de l'attention et de l'expérience, nous nous rendrons compte que nous avons écrit \quotation{startext} et non \quotation{starttext} (avec un double \quotation{t}).

Pensez que les ordinateurs sont très bons et très rapides pour exécuter des instructions, mais très maladroits pour lire nos pensées, et que le mot \quotation{startext} n'est pas le même que \quotation{starttext}. Dans le second cas, le programme sait comment l'exécuter ; dans le premier cas, il ne sait pas quoi faire.



D'autres fois, la localisation de l'erreur ne sera pas aussi facile. En particulier lorsque l'erreur est qu'une tâche a été lancée et que sa fin n'a pas été expressément spécifiée. Parfois, au lieu de chercher l'expression \quotation{tex error} dans le fichier \MyKey{.log}, vous devez chercher un astérisque. Ce caractère au début d'une ligne du fichier journal représente, non pas une erreur fatale, mais un avertissement. Et les avertissements peuvent être utiles pour localiser l'erreur.


Et si les informations du fichier \MyKey{.log} ne sont pas suffisantes, il faudra aller, petit à petit, localiser l'endroit de l'erreur. Une bonne stratégie pour cela consiste à changer l'emplacement de la commande \tex{stoptext} dans le fichier source. Rappelez-vous que \ConTeXt\ arrête de traiter le texte dès qu'il trouve cette commande. Par conséquent, si, dans mon fichier source, j'écris, plus ou moins à la hauteur du milieu, un \tex{stoptext} et que je compile, seule la première moitié sera traitée ; si l'erreur se répète, je saurai qu'elle se trouve dans la première moitié du fichier source, si elle ne se répète pas, cela signifie que l'erreur se trouve dans la deuxième moitié... et ainsi, petit à petit, en changeant l'emplacement de la commande \tex{stoptext}, nous pouvons localiser l'emplacement de l'erreur.


Une autre astuce consiste à mettre en commentaires le paquets de lignes douteuses avec le caractère \quotation{\%} (certain éditeur de texte propose une fonction pour commenter et décommenter tout un paquet de ligne automatiquement).

Une fois que nous l'avons localisée, nous pouvons essayer de la comprendre et de la corriger ou, si nous ne pouvons pas comprendre pourquoi l'erreur se produit, au moins, ayant localisé le point où elle se trouve, nous pouvons essayer d'écrire les choses d'une manière différente pour éviter que l'erreur se reproduise.i

% TODO Garulfo : la remarque me semble plus source de complication dans l'esprit du lecteur que source d'éclaircissement


Ce dernier point, bien sûr, uniquement si nous sommes les auteurs ; si nous nous limitons à composer le texte de quelqu'un d'autre, nous ne pourrons pas le modifier et nous devrons continuer à enquêter jusqu'à ce que nous découvrions les raisons de l'erreur et sa possible solution.


Dans la pratique, lorsqu'on crée un document relativement volumineux avec \ConTeXt, ce que l'on fait habituellement, c'est de le compiler de temps en temps, au fur et à mesure de la rédaction du document, de sorte que si une erreur se produit, nous savons plus ou moins clairement quelle est la partie du document qui vient d'ětre introduite depuis la précédente compilation et qui engendre la nouvelle erreur.


\stopsection

\stopchapter

\stopcomponent

%%% Local Variables:
%%% mode: ConTeXt
%%% mode: auto-fill
%%% TeX-master: "../introCTX_fra.tex"
%%% coding: utf-8-unix
%%% End:
%%% vim:set filetype=context tw=75 : %%%

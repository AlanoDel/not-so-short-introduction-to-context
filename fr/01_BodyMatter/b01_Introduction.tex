%%% File:        b01_Introduction.mkiv
%%% Author:      Joaquín Ataz-López
%%% Begun:       March 2020
%%% Concluded:   March 2020
%%% Contents:    First chapter of the introduction to ConTeXt: a general
%%%              overview of the system. The contents were partly
%%%              by the presentation of LaTeX
%%%              by Kopka and Daly in Chapter 1 of their Guide to
%%%              LaTeX
%%%
%%% Edited with: Emacs + AuTeX - And at times with vim + context-plugin
%%%

\startcomponent b01_Introduction

\environment introCTX_env_00

%==============================================================================

\startchapter[title=\ConTeXt\ : une vue d'ensemble,reference=cap:panorama]

\TocChap

\startsection [title=Qu'est-ce que \ConTeXt\ ?]

\CONTEXT\ est un {\em système de composition}, c'est à dire un ensemble complet d'outils visant à donner à l'utilisateur un contrôle le plus complet précis possible sur la présentation et l'apparence d'un document électronique destiné à être imprimé sur papier ou affiché à l'écran. Ce chapitre expliquera ce que cela signifie. Mais d'abord, mettons en évidence certains des caractéristiques de \CONTEXT .

\startitemize

\index{Mark~II}
\index{Mark~IV}
\item Il existe deux versions de \CONTEXT\, appelées respectivement Mark~II et
  Mark~IV. \CONTEXT\ Mark~II est {\em gelé}, c'est-à-dire qu'il est considéré
  comme finalisé, qu'aucun changement ou nouvelle fonctionnalité ne devrait y
  être introduit. Une nouvelle version ne sera publiée que si un bogue est
  détecté et doit être corrigé. \CONTEXT\ Mark~IV, en revanche, est toujours en
  développement, de nouvelles versions intègrent périodiquement des
  améliorations et fonctionnalité supplémentaires.  Mais, bien que toujours en
  cours de développement, c'est un langage très mature, dans lequel les
  changements introduits par les nouvelles versions sont très subtils et
  n'affectent que le fonctionnement de bas niveau du système. Pour l'utilisateur
  moyen ces changements sont totalement transparents, c'est comme s'ils
  n'existaient pas. Pour finir, bien que les deux versions aient beaucoup en
  commun, il y a aussi quelques incompatibilités entre elles. Cette introduction
  se concentre donc uniquement sur le \CONTEXT\ Mark~IV.

\item \CONTEXT\ est un logiciel libre. Le programme lui-même (c'est-à-dire
  l'ensemble des outils logiciels qui composent \CONTEXT) est distribué sous la
  {\em Licence Publique Générale GNU}. La documentation est fournie sous une
  licence {\em Creative Commons} qui vous permet de la copier et de la
  distribuer librement.

\item \CONTEXT\ n'est pas un programme de traitement de texte ou d'édition de
  texte, mais un ensemble d'outils conçus pour {\em transformer} un texte que
  nous avons précédemment écrit avec notre éditeur de texte préféré. Par
  conséquent, lorsque nous travaillons avec \CONTEXT\ :

  \startitemize

  \item Nous commençons par écrire un ou plusieurs fichiers texte avec n'importe
    quel éditeur de texte.

  \item Dans ces fichiers, en plus du texte qui constitue le contenu réel du
    document, il y a une série d'instructions, instructions qui indiquent à
    \CONTEXT\ à quoi doit ressembler le document final généré à partir des
    fichiers texte originaux. L'ensemble complet des instructions \CONTEXT\
    constitue en fait un {\em langage} ; et puisque ce langage permet de {\em
      programmer} la transformation typographique d'un texte, on peut dire que
    \CONTEXT\ est un {\em langage de programmation typographique}.

    
  \item Une fois les fichiers sources écrits, ils seront traités par un
    programme (également appelé \MyKey{context}\footnote{\CONTEXT\ désigne donc
      à la fois un langage et un programme (ainsi qu'un ensemble d'autres outils
      formant le système complet). Par conséquent, dans un texte comme cette
      introduction, se pose le problème de devoir parfois faire la distinction
      entre les deux aspects. J'ai donc adopté la convention typographique
      consistant à écrire \quote{\CONTEXT} avec son logo (\CONTEXT) lorsque je
      veux me référer exclusivement à la langue, ou indistinctement à la langue
      et au programme. Et lorsque je veux me référer exclusivement au
      programme j'écrirai \MyKey{context} tout en minuscules et avec une police
      de caractères à espacement constant, typique des terminaux d'ordinateur et
      des machines à écrire, que j'utiliserai aussi pour les exemples et
      pour faire référence aux instructions du langage.}), qui, à partir de
    ceux-ci, générera un fichier PDF prêt à être envoyé à une imprimante ou à
    être affiché à l'écran.

  \stopitemize
  
\item Dans \CONTEXT, nous devons donc faire la différence entre le document que nous rédigeons et le document que  \CONTEXT\ génère. Pour éviter tout doute, dans cette introduction, j'appellerai  {\em fichier source} le document texte qui contient les instructions de formatage, et {\em document final} le fichier PDF généré par ConTEXt à partir du fichier source.
  
\stopitemize

Dans la suite, les points fondamentaux ci-dessus seront développés un peu plus.

\stopsection

%-------------------------------------------------------------------------------

\startsection [title=La composition typographique de document]

\index{rédaction}
\index{composition}

Ecrire un document (livre, article, chapitre, brochure, dépliant, imprimé, affiche...) et le mettre en page, dit également le composer, sont deux activités très différentes.

L'écriture du document c'est sa rédaction, qui est effectuée par l'auteur, qui décide de son contenu et de sa structure. Le document créé directement par l'auteur, tel qu'il l'a écrit, s'appelle un {\em manuscrit}. Le manuscrit, par sa nature même, n'est accessible qu'à l'auteur et aux personnes à qui l'auteur permet de le lire. Sa diffusion au-delà de ce cercle intime nécessite que le manuscrit soit {\em publié}. De nos jours, publier quelque chose --- au sens étymologique de \quote{rendre accessible au public} --- est aussi simple que de le mettre sur l'internet, à la disposition de quiconque peut le localiser et souhaite le lire. Mais jusqu'à une date relativement récente, l'édition était un processus qui impliquait des coûts, dépendait de certains professionnels spécialisés dans ce domaine et n'était accessible qu'aux manuscrits qui, en raison de leur contenu ou de leur auteur, étaient considérés comme particulièrement intéressants. Et aujourd'hui encore, nous avons tendance à réserver le mot {\em publication} au type de {\em publication professionnelle} par lequel le manuscrit subit une série de transformations de son apparence visant à améliorer la {\em lisibilité du document}. C'est cette série de transformations qui est appelée {\em composition} ou encore {\em composition typographique}.


L'objectif de la composition est ---- en général, et en laissant de côté les textes publicitaires qui cherchent à attirer l'attention du lecteur --- de réaliser des documents avec une {\em lisibilité} maximale, entendue comme la qualité d'un texte imprimé qui invite à la lecture, ou qui la facilite, et qui fait que le lecteur s'y sente à l'aise. De nombreux aspects y contribuent ; certains, bien sûr, sont liés au {\em contenu} du document (qualité, clarté, standardisation...), mais d'autres dépendent de questions telles que le type et la taille de la police utilisée, la répartition des espaces blancs sur la page, la séparation visuelle entre les paragraphes, etc., on encore d'autres outils, moins graphiques ou visuels, telles que l'existence ou non dans le document de certaines aides à la lecture comme les en-têtes ou les pieds de page, les index, les glossaires, les caractères gras, les titres dans les marges, etc. On pourrait appeler \quote{art de la composition} ou \quote{art de l'impression} la connaissance et la manipulation correcte de toutes les ressources dont dispose un compositeur, un imprimeur.


Historiquement, et jusqu'à l'avènement des ordinateurs, les tâches et les rôles du rédacteur et du compositeur sont restés clairement différenciés. L'auteur écrivait à la main ou, depuis le milieu du XIX\high{e} siècle, sur une \quote{machine à écrire} dont les ressources typographiques étaient encore plus limitées que celles de l'écriture manuscrite ; puis il remettait ses originaux à l'éditeur ou à l'imprimeur qui se chargeait de les transformer pour en tirer le document imprimé.


De nos jours, grâce à l'informatique, il est plus facile pour l'auteur lui-même de définir la composition jusqu'aux moindres détails. Mais pour autant les qualités d'un bon auteur ne sont pas les mêmes que celles d'un bon compositeur. L'auteur doit avoir une bonne connaissance du sujet traité, savoir le structurer, l'exposer avec clarté, avec créativité, avec un rythme. etc. Le compositeur typographe doit avoir une bonne connaissance de l'environnement graphique et conceptuel à sa disposition, un goût de l'esthétique pour les utiliser harmonieusement, de façon cohérente avec le sujet traité, avec les tendances du moment.

Avec un bon logiciel de traitement de texte \footnote{Par une convention assez ancienne, on fait une distinction entre les logiciels d'édition de texte et les logiciels de  {\em traitement de texte}. Les premiers manipulent des fichiers de texte brut, et les seconds des fichiers de texte au format binaire permettant une plus grande complexité.}, il est possible d'obtenir une composition raisonnablement bonne. Mais les traitements de texte ne sont généralement pas conçus pour la composition et leurs résultats, même s'ils sont corrects, ne sont pas comparables à ceux obtenus avec d'autres outils spécifiquement conçus pour contrôler la composition des documents. En fait, les traitements de texte sont l'évolution des machines à écrire, et leur utilisation, dans la mesure où ces outils masquent la différence entre la rédaction du texte (la paternité) et sa composition, tend à produire des textes parfois moins structurés et moins bien optimisés typographiquement.

Au contraire, les outils tels que  \ConTeXt\ sont l'évolution de l'imprimerie ; ils offrent beaucoup plus de possibilités de composition et, surtout, il n'est pas possible d'apprendre à les utiliser sans acquérir également, en cours de route, de nombreuses notions liées à la composition, contrairement aux traitements de texte, qui peuvent être utilisés pendant de nombreuses années sans apprendre un seul mot de typographie.


\stopsection

\startsection [title=Les langages de balisage]
\index{langages de balisage+balises}
\index{langages de balisage+markup}

Avant l'arrivée de l'informatique, comme je l'ai déjà dit, l'auteur préparait son manuscrit à la main ou à la machine à écrire et le remettait à l'éditeur ou à l'imprimeur, qui était chargé de transformer le manuscrit en texte final imprimé. Bien que l'auteur soit relativement peu intervenu dans cette transformation, il l'a fait dans une certaine mesure, par exemple en indiquant que certaines lignes du manuscrit étaient les titres de ses différentes parties (chapitres, sections...) ; ou en indiquant que certains fragments devaient être mis en valeur typographiquement d'une certaine manière. Ces indications étaient faites par l'auteur dans le manuscrit lui-même, parfois expressément, et d'autres fois au moyen de certaines conventions qui, avec le temps, se sont développées ; ainsi, par exemple, les chapitres commençaient toujours sur une nouvelle page, en insérant plusieurs lignes vierges avant le titre, en le soulignant, en l'écrivant en majuscules ; ou en encadrant le texte à mettre en valeur entre deux soulignements, en augmentant l'indentation d'un paragraphe, etc.

  L'auteur, en somme, {\em indiquait} dans le texte original quelques éléments concernant la composition typographique du texte. L'éditeur ensuite inscrivait à son tour de nouvelles indication pour l'imprimeur, comme par exemple la police et la taille de caractères.


  Aujourd'hui, dans un monde informatisé, nous pouvons continuer à faire de même pour la génération de documents électroniques, au moyen de ce que l'on appelle un {\em langage de balisage}. Dans ce type de langage, on utilise une série de marques ou d'indications ou encore de {\em balises} que le programme traitant le fichier qui les contient sait interpréter. Le langage de balisage le plus connu au monde aujourd'hui est sans doute le HTML, car la plupart des pages web sont basées sur ce langage. Un fichier HTML contient le texte d'une page web, ainsi qu'une série de marques qui indiquent au programme de navigation avec lequel la page est chargée, comment elle doit être affichée. L'ensemble des balises HTML compréhensibles par les navigateurs web, ainsi que les instructions sur la manière et l'endroit où les utiliser, est appelé \quote{langage HTML}, qui c'est un langage de balisage. Mais en plus du HTML, il existe de nombreux autres langages de balisage ; en fait, ceux-ci sont en plein essor et ainsi, le XML, qui est le langage de balisage par excellence, est aujourd'hui absolument omniprésent et est utilisé pour presque tout : pour la conception de bases de données, pour la création de langages spécifiques, pour la transmission de données structurées, pour les fichiers de configuration d'applications, et ainsi de suite. Il existe également des langages de balisage conçus pour la conception graphique (SVG, TikZ ou MetaPost), les formules mathématiques (MathML), la musique (Lilypond et MusicXML), la finance, la géographie, etc. Il y a aussi, bien sûr, ceux destinés à la transformation typographique des textes, et parmi eux se distinguent \TeX\ et ses dérivés.



En ce qui concerne les balises {\em typographiques}, qui indiquent l'apparence que doit avoir un texte, il en existe deux types, que nous pourrions distinguer comme d'un côté les {\em balises purement typographique} (ou encore graphiques) et de l'autre les {\em balises sémantiques} (ou encore conceptuelles, logiques). Les balises purement typographiques se limitent à indiquer précisément quelles ressources typographiques doivent être utilisées pour afficher un certain texte ; par exemple, lorsque nous indiquons qu'un certain texte doit être en gras ou en italique, de telle ou telle couleur. Le balisage sémantique, quant à lui, indique la fonction d'un texte donné dans l'ensemble du document, par exemple lorsque nous indiquons qu'il s'agit d'un titre, d'un sous-titre, d'une citation. En général, les documents qui utilisent de préférence ce deuxième type de balisage sont plus cohérents et plus faciles à composer, car la différence entre la paternité et la composition y est à nouveau claire : l'auteur indique que cette ligne est un titre, ou que ce fragment est un avertissement, ou une citation ; et le compositeur décide comment mettre en valeur typographiquement tous les titres, avertissements ou citations ; ainsi, d'une part, la cohérence est garantie, puisque tous les fragments remplissant la même fonction auront la même apparence, et, d'autre part, on gagne du temps, puisque le format de chaque type de fragment ne doit être indiqué qu'une seule fois.


\stopsection

\startsection [title=\TeX\ et ses dérivés]
  \index{moteurs \TeX}

\TeX\   a été développé à la fin des années 1970 par  {\sc Donald E. Knuth}, professeur de théorie de la programmation à l'université de Stanford, qui l'a utilisé pour composer ses propres publications et ainsi que pour donner un exemple de {\em programmation littéraire}, une approche de la programmation où le code source du logiciel est systématique commenté et documenté. Avec \TeX, {\sc Knuth} a également développé un langage de programmation supplémentaire appelé \MetaFont, pour la conception de caractères typographiques, avec lequel il a créé une police qu'il a nommée {\em Computer Modern}, qui, en plus des caractères habituels de toute police, comprenait également un ensemble complet de \quote{glyphes} \footnote{En typographie, un glyphe est une représentation graphique d'un caractère, de plusieurs caractères ou d'une partie d'un caractère et est l'équivalent actuel du type d'impression (la pièce mobile en bois ou en plomb qui portait la gravure de la lettre).} pour l'écriture des mathématiques. Il a ajouté à tout cela quelques utilitaires supplémentaires et c'est ainsi qu'est né le système de composition appelé \TeX, qui, pour sa puissance, la qualité de ses résultats, sa flexibilité d'utilisation et ses vastes possibilités, est considéré comme l'un des meilleurs systèmes informatiques pour la composition de textes. Il a été pensé pour des textes dans lesquels il y avait beaucoup de mathématiques, mais on a vite vu que les possibilités du système le rendaient adapté à tous les types de textes.

\reference[ref:cajas]{}  En interne, il fonctionne comme la machine à écrire d'une presse à imprimer, car tout y est {\em boîte}. Les lettres sont contenues dans des boîtes, les blancs sont aussi des boîtes. Un mot est une boîte enfermant les boîtes de ses lettres. Une ligne est une boîte enfermant les boîtes de ses mots et des blancs entre ces mots. Un paragraphe est une boîte contenant l'ensemble des boîtes de ses lignes. Et ainsi de suite. Tout cela avec une précision extraordinaire apportée au traitement des mesures. Il suffit de penser que la plus petite unité que \TeX\ traite est 65,536 fois plus petite que le point typographique, avec lequel on mesure les caractères et les lignes, qui est généralement la plus petite unité traitée par la plupart des programmes de traitement de texte. Cette plus petite unité traité par \TeX\ est d'environ 0,000005356 millimètre.


% He copiado y pegado la épsilon acentuada, de "Aprender ConTeXt",
% de Pablo Rodríguez, pero no se por qué razón, no la procesa. Por
% lo tanto utilizo \definecharacter para crear una epsilon
% acentuada.

\definecharacter etilde {\buildtextaccent ´ {\lower.2ex\hbox{\epsilon}}}


Le nom \TeX\ vient de la racine du mot grec \tau\etilde\chi\nu\eta,
écrit en lettres capitales ({\tfx ΤÉΧΝΗ}). Par conséquent, comme la dernière lettre du nom n'est pas un \quotation{X}  latin, mais le  \quotation{\chi} grec, il faut prononcer \quote{Tec}. Ce mot grec, quant à lui, signifiait à la fois \quote{art} et \quote{technique}, c'est pourquoi {\sc Knuth} l'a choisi comme nom pour son système. Le but de ce nom, écrit-il, \quote{est de rappeler qu'il s'occupe principalement de manuscrits techniques de haute qualité. Elle met l'accent sur l'art et la technologie, tout comme le mot grec sous-jacent}.
Par convention établie par Knuth, le nom de est à écrire :

\startitemize

\item Dans des textes formatés typographiquement, comme le présent texte, en utilisant le logo que j'ai utilisé jusqu'à présent : Les trois lettres sont en majuscules, avec le  \quotation{E} central légèrement décalé vers le bas pour faciliter un rapprochement entre le  \quotation{T} et le  \quotation{X}  ; c'est-à-dire :  \quotation{\TeX} .
Pour rendre plus facile l'écriture d'un tel logo, Knuth a inclus
dans une instruction qui l'inscrit dans le document final :
TeXTeX.


  \startSmallPrint

    Pour rendre plus facile l'écriture d'un tel logo,  {\sc Knuth} a inclus
dans une instruction qui l'inscrit dans le document final : \PlaceMacro{TeX}\tex{TeX}.

  \stopSmallPrint

\item Dans un texte non formaté (tel qu'un e-mail ou un fichier texte), le  \quotation{T} et le  \quotation{X} sont en majuscules, et le  \quotation{e} du milieu est en minuscules ; par exemple :  \quotation{TeX}.

\stopitemize

Cette convention est suivie dans tous les dérivés de \TeX\ qui l'incluent dans leur propre nom, comme par exemple  \ConTeXt, qui lorsqu'il est écrit en mode texte doit être écrit  \quotation{ConTeXt}.


\startsubsection [reference=sec:motores,title=Les moteurs \TeX]

  Le programme \TeX\ est un logiciel libre : son code source est à la disposition du public et chacun peut l'utiliser ou le modifier à sa guise, à la seule condition que, si des modifications sont introduites, le résultat ne puisse être appelé \quotation{\TeX}. C'est la raison pour laquelle, au fil du temps, certaines adaptations du programme sont apparues, qui lui ont apporté différentes améliorations, et qui sont généralement appelées  {\em moteurs \TeX} (engine en anglais). En dehors du programme original, les principaux moteurs \TeX\ sont, par ordre chronologique d'apparition  \pdfTeX, \eTeX, \XeTeX\ et \LuaTeX. Chacun d'entre eux est censé intégrer les améliorations de son prédécesseurs. Ces améliorations, en revanche, jusqu'à l'apparition de  \LuaTeX , n'ont pas affecté le langage lui-même, mais seulement les fichiers d'entrée, les fichiers de sortie, la gestion des polices et le fonctionnement de bas niveau des macros.


\startSmallPrint

  La question du choix du moteur  \TeX\ à utiliser fait l'objet d'un débat animé dans l'univers  \TeX. Je ne m'y attarderai pas ici, car \ConTeXt\ Mark~IV ne fonctionne qu'avec \LuaTeX. En fait, dans le monde de \ConTeXt\, la discussion sur les moteurs devient une discussion sur l'utilisation de Mark~II (qui fonctionne avec \pdfTeX et \XeTeX) ou Mark~IV (qui fonctionne avec \LuaTeX).


\stopSmallPrint

\stopsubsection


%===============================================================================

\startsubsection [title=Les formats dérivés de \TeX]

  Le noyau, ou cœur, de \TeX\ contient seulement un ensemble d'environ 300 instructions, appelées {\em primitives}, qui conviennent aux opérations de composition et aux fonctions de programmation très basiques. Ces instructions sont pour la plupart de très {\em bas niveau}, ce qui, en terminologie informatique, signifie qu'elles se rapprochent des opérations élémentaires de l'ordinateur, dans un langage machine peu approprié aux êtres humains, du type  \quotation{déplacer ce caractère de 0,000725 millimètre vers le haut}.

Pour cette raison, {\sc Knuth} a rendu \TeX\ extensible, c'est-à-dire disposant d'un mécanisme permettant de définir des instructions de plus haut niveau, dans un langage plus facilement compréhensibles par les êtres humains. Ces instructions, qui au moment de l'exécution sont décomposées en instructions élémentaires, sont appelées {\em macros}. Par exemple, l'instruction \TeX\ qui imprime votre logo  (\tex{TeX}), est décomposée lors de son exécution en :

\placefigure [here,none] [] {}{
\startDemoC
T 
\kern -.1667em 
\lower .5ex 
\hbox {E} 
\kern -.125em 
X
\stopDemoC}


On comprend là qu'il est beaucoup plus facile pour un être humain de comprendre et mémoriser la simple commande \quotation{\PlaceMacro{TeX}\type{\TeX}} dont l'exécution effectue l'ensemble des opérations typographiques nécessaires à l'impression du logo.


\startSmallPrint

La différence entre les {\em macros} et les  {\em primitives} n'est vraiment importante que du point de vue du développeur de \TeX. Du point de vue de l'utilisateur, ce sont toutes des {\em instructions} ou, si vous préférez, des {\em commandes}. {\sc Knuth} les appelait des {\em séquences de contrôle}.

\stopSmallPrint


Cette possibilité d'étendre le langage par le biais de  {\em macros} est l'une des caractéristiques qui ont fait de \TeX\ un outil si puissant. En fait,  {\sc Knuth} lui-même a conçu environ 600 macros qui, avec les 300 primitives, constituent le format appelé \quotation{Plain \TeX}. Il est assez courant de confondre \TeX\ lui-même avec Plain \TeX\ et, en fait, presque tout ce qui est dit ou écrit sur \TeX, se réfère en fait à Plain \TeX. Les livres qui prétendent être sur \TeX\  (y compris le livre fondateur \quotation{{\em The \TeX Book}}), font en fait référence à Plain \TeX\ ; et ceux qui pensent qu'ils manipulent directement \TeX\ manipulent en fait Plain \TeX.


Plain  \TeX\ est ce que l'on appelle dans la terminologie \TeX\ un {\em format}, constitué d'un vaste ensemble de macros, ainsi que de certaines règles syntaxiques sur la manière et la façon de les utiliser. En plus de Plain  \TeX, d'autres formats ont été développés au fil du temps, notamment \LaTeX\ un vaste ensemble de macros pour \TeX\ conçu en 1985 par {\sc Leslie Lamport}, qui est probablement le dérivé de \TeX\ le plus utilisé dans le monde universitaire, technologique et mathématique. \ConTeXt\ est (ou a commencé à être), de même que \LaTeX\, un format dérivé de \TeX.

Normalement, ces {\em formats} sont accompagnés d'un programme qui charge dans la mémoire de l'ordinateur les macros qui les composent avant d'appeler \MyKey{tex} (ou l'un des autres moteurs précédemment listés) pour traiter le fichier source. Mais bien que tous ces formats exécute finalement \TeX, comme chacun possède ses instructions et ses règles syntaxiques spécifiques, du point de vue de l'utilisateur, nous pouvons les considérer comme des {\em langages différents}. Ils sont tous inspirés de \TeX, mais différents de \TeX, et différents les uns des autres.

\stopsection

\startsection [title=\ConTeXt, reference=sec:ctx]

  En fait, si \ConTeXt\ a commencé comme un  {\em format} de \TeX, aujourd'hui il est beaucoup plus que cela. \ConTeXt comprend :


\startitemize[n]

\item Un très large ensemble de macros de \TeX. Si Plain \TeX\ se compose d'environ 900 instructions, il en compte près de 3500 ; et si l'on ajoute les noms des différentes options que ces commandes prennent en charge, on parle d'un vocabulaire d'environ 4000 mots. Le vocabulaire est aussi large car la stratégie de \ConTeXt pour faciliter son apprentissage est d'inclure de nombreux synonymes des commandes et des options.


  \startSmallPrint

Ce qui est prévu, pour obtenir un certain effet, c'est de fournir à l'utilisateur l'ensemble des façons dont celui ci pourrait chercher à appeler cet effet. Par exemple, pour obtenir simultanément un caractère gras (en anglais {\em bold}) et italique (en anglais {\em italic}), \ConTeXt propose trois instructions identiques en terme de résultat : \type{\bi}, \type{\italicbold} y \type{\bolditalic}.

  \stopSmallPrint

\item Un autre ensemble assez complet de macros pour \MetaPost, un langage de programmation graphique dérivé de  \MetaFont, qui, lui-même, est le langage de conception de caractères que {\sc
    Knuth} a co-développé avec \TeX.


\item Plusieurs {\em scripts} développés en  {\sc Perl} (les plus anciens),  {\sc Ruby} (certains également anciens et d'autres moins) et {\sc Lua} (les plus récents).


\item Une interface qui intègre \TeX, \MetaPost, {\sc Lua} et XML, permettant d'écrire et de traiter des documents dans n'importe lequel de ces langages, ou qui mélangent des éléments de certains d'entre eux.

\stopitemize

\startSmallPrint

  Vous n'avez pas compris grand-chose à l'explication ci-dessus ? Ne vous inquiétez pas. J'ai utilisé beaucoup de jargon informatique et mentionné beaucoup de programmes et de langages. Mais il n'est pas nécessaire de savoir d'où viennent les différents composants pour les utiliser. L'important, à ce stade de l'apprentissage, est de garder à l'esprit qu'il intègre de nombreux outils d'origines diverses qui forment un  {\em système de composition typographique}.


\stopSmallPrint


C'est en raison de cette intégration d'outils d'origines différentes que l'on caractérise \ConTeXt\ de \quotation{technologie hybride} dédié à la composition typographique de documents. C'est également ce qui fait de \ConTeXt\ un système extraordinairement avancé et puissant.

Mais bien que \ConTeXt\ soit bien plus qu'un ensemble de macros pour \TeX, ses fondamentaux restent basés sur \TeX, et donc ce document, qui se veut n'être qu'une {\em introduction}, se concentre sur cet aspect.


\ConTeXt\ en revanche est beaucoup plus moderne que \TeX. Lorsque \TeX\ a été conçu, l'informatique commençait à peine à émerger, et on était encore loin d'entrevoir ce que serait (ce qui allait devenir) l'Internet et le monde du multimédia. En ce sens,  \ConTeXt\ intègre naturellement certains éléments qui ont toujours constitué une sorte de corps étranger, tels que l'inclusion de graphiques externes, le traitement des couleurs, les hyperliens dans les documents électroniques, l'hypothèse d'un format de papier adapté d'un affichage sur écran, etc.


\stopsubsection

\startsubsection 
  [ 
    reference=sec:historiactx, 
    title=Une brève histoire de \ConTeXt
  ]

\ConTeXt{} est né vers 1991. Il a été créé par {\sc Hans Hagen} et {\sc Ton Otten} au sein d'une société néerlandaise de conception et de composition de documents appelée \quotation{{\em Pragma Advanced
  Document Engineering}}, souvent abrégée en Pragma ADE. Il s'agissait au départ d'un ensemble de macros \TeX\ en néerlandais, officieusement connu sous le nom de {\em Pragmatex}, et destiné aux employés non techniques de l'entreprise, qui devaient gérer les nombreux détails de la mise en page des documents à éditer, et qui n'étaient pas habitués à utiliser des langages de balisage et des interfaces dans une autre langue que le néerlandais.

La première version de \ConTeXt{} a donc été écrite en néerlandais. L'idée était de créer un nombre suffisant de macros avec une interface uniforme et cohérente. Vers 1994, le {\em paquet} était suffisamment stable pour qu'un manuel d'utilisation soit écrit en néerlandais, et en 1996, à l'initiative de {\sc Hans Hagen}, le nom \quotation{\ConTeXt{}} a été utilisé pour s'y référer. Ce nom est censé signifier  \quotation{Texte avec \TeX} (en utilisant la préposition latine "con" qui a la même signification qu'en espagnol), mais il joue en même temps avec le terme  \quotation{Contexte}, qui en néerlandais (comme en anglais) s'écrit  \quotation{context}. Derrière ce nom, il y a donc un triple jeu de mots entre \quotation{\TeX}, \quotation{texte} et \quotation{contexte}.



\startSmallPrint

  Par conséquent, bien que  \ConTeXt\ soit dérivé de \TeX\ (prononcé  \quotation{Tec}), il ne devrait pas être prononcé \quotation{Contect} afin de ne pas perdre ce jeu de mots.

\stopSmallPrint

L'interface a commencé à être traduite en anglais vers 2005, donnant lieu à la version connue sous le nom de  \ConTeXt\ Mark~II, où le \quotation{II}  s'explique par le fait que dans l'esprit des développeurs, la version \quotation{I} est la version précédente en néerlandais, même si elle n'a jamais vraiment été appelée ainsi. Après la traduction de l'interface en anglais, le système a commencé à être utilisé en dehors des Pays-Bas, et l'interface a été traduite dans d'autres langues européennes comme le français, l'allemand, l'italien et le roumain. La documentatino \quotation{officielle} de \ConTeXt{} est généralement écrite sur la version anglaise, et c'est donc sur cette version que nous travaillons dans ce document, même si l'auteur de ce document (c'est-à-dire moi) est plus à l'aise en espagnol qu'en anglais.


Dans sa version initiale,  \ConTeXt\ Mark~II fonctionnait avec le {\em moteur \TeX} \pdfTeX. Plus tard, lorsque le nouveau moteur \XeTeX\ est apparu, \ConTeXt\ Mark~II a été modifié pour en permettre l'utilisation, qui présentait de nombreux avantages par rapport à \pdfTeX. Des années plus tard encore, lorsque le moteur \LuaTeX a été développé, il a été décidé de reconfigurer le fonctionnement interne de \ConTeXt\ Mark~II  pour intégrer toutes les nouvelles possibilités offertes par ce dernier moteur. C'est ainsi qu'est né  \ConTeXt\ Mark~IV, qui a été présenté en 2007, immédiatement après la présentation de \LuaTeX. La décision d'adapter \ConTeXt\ à \LuaTeX a très probablement été influencée par le fait que deux des trois principaux développeurs de  \ConTeXt{},  {\sc Hans Hagen} et {\sc Taco Hoekwater}, font également partie de l'équipe de développement de \LuaTeX. Par conséquent, \ConTeXt\ Mark~IV et \LuaTeX\ sont nés simultanément et ont été développés à l'unisson. Il existe une synergie entre \ConTeXt{} et \LuaTeX\ et qui n'existe avec aucun autre dérivé de \TeX\ ; et je ne pense pas qu'aucun d'entre eux ne profite des possibilités de \LuaTeX\ comme \ConTeXt{} le fait.


Entre Mark~II et Mark~IV, il existe de nombreuses différences, bien que la plupart d'entre elles soient {\em internes}, c'est-à-dire qu'elles concernent le fonctionnement de la macro à un bas niveau, de sorte que du point de vue de l'utilisateur, la différence n'est pas perceptible : le nom et les paramètres de la macro sont les mêmes. Il existe cependant quelques différences qui affectent l'interface et vous obligent à faire les choses différemment selon la version avec laquelle vous travaillez. Ces différences sont relativement peu nombreuses, mais elles affectent des aspects très importants comme, par exemple, l'encodage du fichier d'entrée, ou la gestion des polices installées dans le système.


\startSmallPrint

Cependant, il serait apprécié qu'il existe quelque part un document expliquant (ou listant) les différences significatives entre Mark~II et \Conjecture Mark~IV. Dans le wiki de  \ConTeXt, par exemple, il existe parfois {\em deux syntaxes} (souvent identiques) pour chaque commande. Je suppose que l'une est la version Mark~II et l'autre la version Mark~IV ; et à deviner, je suppose également que la première version est la version Mark~II. Mais en pratique rien n'est indiqué à ce sujet sur le wiki.


\stopSmallPrint

Le fait que, pour les utilisateur, les différences au niveau du langage soient relativement peu nombreuses, signifie que dans de nombreux cas, plutôt que de parler de deux versions, nous parlons de deux \quotation{saveurs} de \ConTeXt{}. Mais qu'on les appelle d'une manière ou d'une autre, le fait est qu'un document préparé pour Mark~II peut ne pas être compatible d'une compilation avec Mark~IV et vice versa ; et si le document mélange les deux versions, il est fort probable qu'il ne se compilera bien avec aucune d'entre elles ; ce qui signifie que l'auteur du fichier source doit commencer par décider s'il l'écrit pour Mark~II ou Mark~IV.


\startSmallPrint

Si nous avons a travailler avec différentes versions de \ConTeXt{}, une bonne astuce pour facilement distinguer les versions des fichiers sources consiste à utiliser une extension différente dans le nom des fichiers. Ainsi, par exemple, mes fichiers écrits pour Mark~II sont nommés \MyKey{.mkii} et ceux écrits pour Mark~IV sont nommés \MyKey{.mkiv}. Il est vrai que  \ConTeXt{} s'attend à ce que tous les fichiers sources aient l'extension  \MyKey{.tex}, mais vous pouvez changer l'extension tant que lorsque vous invoquez un fichier, vous indiquez explicitement son extension, si elle n'est pas celle par défaut.

\stopSmallPrint


La distribution de \ConTeXt{} que vous installez à partir de leur wiki, \suite-, inclut les deux versions, et pour éviter toute confusion ---je suppose--- propose une commande distincte pour compiler avec chacune d'entre elles. Mark~II compile avec la commande  \MyKey{texexec} et Mark~IV avec la commande  \MyKey{context}.


\startSmallPrint

  En réalité, aussi bien \MyKey{context} que \MyKey{texexec} sont des {\em scripts} qui lancent, avec différentes options, \MyKey{mtxrun} qui, à son tour, est un  {\em script} {\sc Lua}.

\stopSmallPrint

% TODO

A ce jour, Mark~II est gelé et Mark~IV est toujours en cours de développement, ce qui signifie que les nouvelles versions de Mark~II ne sont publiées que lorsque des bogues ou des erreurs sont détectés, tandis que les nouvelles versions de Mark~IV sont publiées régulièrement ; parfois même deux ou trois par mois ; bien que dans la plupart des cas, ces  \quotation{nouvelles versions} n'introduisent pas de changements notables dans le langage, et se limitent à améliorer d'une manière ou d'une autre l'implémentation de bas niveau d'une commande, ou à mettre à jour l'un des nombreux manuels qui sont inclus dans la distribution. Néanmoins, si nous avons installé la version de développement --- qui est celle que je recommande et celle qui est installée par défaut avec \suite-{} ---, il est approprié de mettre à jour notre installation de temps en temps (voir l'\in{annexe}[instalación_suite] concernant la mise à jour de la version installée de \suite-).


\startSmallPrint

\startsubsubsubsubject 
  [title=LMTX et autres implémentations alternatives de Mark~IV]

Les développeurs de \ConTeXt{} sont soucieux de la qualité du logiciel et n'ont cessé de faire évoluer Mark~IV ; de nouvelles versions sont testées et expérimentées. Celles-ci, en général, ne diffèrent de Mark~IV que sur très peu de points, et ne présentent pas d'incompatibilité de compilation comme cela existe entre Mark~IV et Mark~II, ce qui traduit la maturité du langage du point de vue utilisateur.

Ainsi, quelques variantes de Mark~IV ont été développées, appelées respectivement Mark~VI, Mark~IX et Mark~XI. Je n'ai pu trouver qu'une petite référence à Mark~VI dans le wiki de \ConTeXt{} où il est indiqué que sa seule différence avec Mark~IV est la possibilité de définir des commandes en assignant aux paramètres non pas un nombre, comme c'est traditionnel dans \TeX, mais un nom, comme cela se fait habituellement dans presque tous les langages de programmation.

Plus important que ces petites variantes ---je pense--- est l'apparition dans l'univers de \ConTeXt{} d'une nouvelle version, appelée LMTX, nom qui est un acronyme pour \LuaMetaTeX\ : un nouveau {\em moteur} de \TeX\ qui est une version simplifiée et optimisée de \LuaTeX, développé en vue d'économiser les ressources informatiques et d'offir une solution \TeX\ aussi minimaliste que possible ; c'est-à-dire que LMTX nécessite moins de place sur disque dur, moins de mémoire et moins de puissance de traitement que \ConTeXt\ Mark~IV.

LMTX a été présenté au printemps 2019 et l'on suppose qu'il n'impliquera aucune altération externe du langage Mark~IV. Pour l'auteur du document, il n'y aura aucune différence dans la conception ; mais au moment de la compilation, vous pouvez choisir entre compiler avec  \LuaTeX, ou compiler avec \LuaMetaTeX. Une procédure pour attribuer un nom de commande différent à chacune des installations (\in{section} [sec:alias]) est expliquée dans l'\in{annexe}[instalación_suite], relative à l'installation de \ConTeXt.


\stopsubsubsubject

\stopSmallPrint

\stopsubsection

\startsubsection [title=\ConTeXt\ versus \LaTeX]

  Comme le format  dérivé de \TeX\ le plus populaire est \LaTeX\, la comparaison entre celui-ci et \ConTeXt\ est inévitable.

  Il s'agit bien sûr de langages différents mais, d'une certaine manière, liés par leur origine commune \TeX\ ; la parenté est donc similaire à celle qui existe entre, par exemple, l'espagnol et le français : des langues qui partagent une origine commune (le latin) qui utilise des syntaxes {\em similaires} et de nombreux mots se correspondants assez directement. Mais au-delà de cet air de famille,  \LaTeX\ et \ConTeXt\  diffèrent dans leur philosophie et leur mise en œuvre, puisque les objectifs initiaux de l'un et de l'autre sont, en quelque sorte, contradictoires.

  \LaTeX\  vise à faciliter l'utilisation de \TeX, en éloignant l'auteur des détails typographiques spécifiques pour l'inciter à se concentrer sur le contenu, et laisser les détails de la composition entre les mains de \LaTeX. En d'autres termes, la simplification de l'utilisation de  \TeX\ est obtenue au prix d'une limitation de son immense flexibilité, par la prédéfinition de nombreux formats de base et la réduction du nombre de choix typographiques que l'auteur doit déterminer.

A l'opposé de cette philosophie,  \ConTeXt\ est ne au sein d'une entreprise dédiée à la composition de documents. Par conséquent, loin d'essayer d'isoler l'auteur des détails de la composition, ce qu'il tente de faire, c'est de lui donner un contrôle absolu et complet sur ceux-ci. Pour ce faire, \ConTeXt\ fournit une interface homogène  et cohérente qui reste beaucoup plus proche de l'esprit original \TeX\ que \LaTeX.


Cette différence de philosophie et d'objectifs fondamentaux se traduit à son tour par une différence de mise en œuvre. Parce que \LaTeX, qui tend à simplifier au maximum, n'a pas besoin d'utiliser toutes les ressources de  \TeX. Son cœur est, d'une certaine manière, assez simple. Par conséquent, lorsque vous souhaitez étendre ses possibilités, vous devez construire un  {\em paquet}. Cet {\em ensemble de paquets} associé à \LaTeX est à la fois une force et une faiblesse : une force, car l'énorme popularité de \LaTeX, ainsi que la générosité de ses utilisateurs, impliquent que pratiquement tous les besoins qui se présentent ont déjà été soulevés par quelqu'un, et qu'il existe un paquet qui y réponde ; mais aussi une faiblesse, car ces paquets sont souvent incompatibles entre eux, et leur syntaxe n'est pas toujours homogène, ce qui signifie que l'utilisation de \LaTeX\ exige une plongée continue dans les milliers de paquets existants pour trouver ceux dont nous avons besoin et les faire fonctionner ensemble.

Contrairement à la simplicité du noyau de \LaTeX\ et son extensibilité par le biais de paquets, \ConTeXt\ est conçu pour intégrer et rendre accessibles toutes --- ou presque toutes --- les possibilités typographiques de \TeX, de sorte que sa conception est beaucoup plus monolithique, mais, en même temps, il est aussi plus modulaire : le noyau  \ConTeXt\ permet de faire presque tout et il est garanti qu'il n'y aura pas d'incompatibilités entre les différentes commandes, il n'y a pas besoin de rechercher les extensions dont vous avez besoin (elles sont déjà présentes), et la syntaxe du langage est homogène entre les différents commande.


Il est vrai que  \ConTeXt\ propose des  {\em modules} d'extension dont on pourrait considérer qu'ils ont une fonction similaire à celle des paquets de  \LaTeX, mais la vérité est que la fonction des deux est très différente : les modules de \ConTeXt\ sont conçus exclusivement pour accueillir des fonctionnalités supplémentaires qui, parce qu'ils sont en phase expérimentale, n'ont pas encore été incorporés dans le noyau, ou pour permettre à des développeurs en dehors de l'équipe de développement de \ConTeXt\ de les proposer.

Je ne pense pas que l'une de ces deux {\em philosophies} puisse être considérée comme préférable à l'autre. La réponse dépend plutôt du profil de l'utilisateur et de ce qu'il souhaite. Si l'utilisateur ne veut pas s'occuper de questions typographiques, mais simplement produire des documents standardisés de très haute qualité, il serait probablement préférable pour lui d'opter pour un système comme \LaTeX\ ; au contraire, l'utilisateur qui aime expérimenter, ou qui a besoin de contrôler chaque détail de ses documents, ou qui doit concevoir un design spécial pour un certain document, ferait probablement mieux d'utiliser un système comme \ConTeXt, où l'auteur dispose de tous les contrôle ; avec le risque, bien sûr, qu'il ne sache pas correctement l'utiliser.

\stopsubsection

\startsubsection 
  [title=La logique de travail avec \ConTeXt]

Lorsque nous travaillons avec \ConTeXt, nous commençons toujours par écrire un fichier texte (que nous appellerons  {\em fichier source}), dans lequel nous inclurons, en plus du contenu de notre document final à proprement parler, les instructions (en langage \ConTeXt) qui indiquent exactement comment nous voulons que le document soit composé : quel aspect général nous voulons donner à ses pages et paragraphes, quelles marges nous souhaitons appliquer à certains paragraphes spéciaux, quelles types de police doit être utilisé, quels fragments souhaitons nous afficher avec une police différente, etc. Une fois que nous avons écrit le fichier source, depuis un terminal, nous executerons le programme \MyKey{context}, qui le traitera et, à partir de celui-ci, générera un fichier différent, dans lequel le contenu de notre document aura été formaté selon les instructions qui étaient incluses dans le fichier source. Ce nouveau fichier peut être envoyé à l'imprimante, affiché à l'écran, hébergé sur Internet ou distribué à nos contacts, amis, clients, professeurs, étudiants... bref, à tous ceux pour qui nous avons écrit le document.

C'est-à-dire que lorsqu'il travaille avec \ConTeXt, l'auteur agit sur un fichier dont l'apparence n'a rien à voir avec celle du document final : le fichier avec lequel l'auteur travaille directement est un fichier texte qui n'est pas formaté typographiquement. À cet égard, son fonctionnement est très différent de celui des programmes dits de {\em traitement de texte}, qui affichent l'aspect final du document édité au fur et à mesure de sa saisie. Pour ceux qui sont habitués aux  {\em traitements de texte}, le fonctionnement de l'application peut sembler étrange au début, et il peut même falloir un certain temps pour s'y habituer. Cependant, une fois que vous vous y serez habitué, vous comprendrez que cette autre façon de travailler, faisant la différence entre le fichier de travail et le résultat final, est en fait un avantage pour de nombreuses raisons, parmi lesquelles je soulignerai, sans ordre particulier, les suivantes :

\startitemize[n,broad]

\item car les fichiers texte sont plus  \quotation{légers} à manipuler que les fichiers binaires des traitements de texte et que leur édition nécessite moins de ressources informatiques ; ils sont moins sujets à la corruption et ne deviennent pas illibles si la version du programme avec lequel ils ont été créés change. Ils sont également compatibles avec n'importe quel système d'exploitation et peuvent être édités avec de nombreux éditeurs de texte, de sorte que pour travailler avec eux, il n'est pas nécessaire de disposer d'un logiciel d'édition particulier : n'importe quel autre programme d'édition de texte fera l'affaire, et  chaque système d'exploitation informatique propose un voire des programmes d'édition de texte.

\item car la différenciation entre le document de travail et le document final permet de distinguer ce qui est le contenu réel du document de ce qui sera son apparence, permettant à l'auteur de se concentrer sur le contenu dans la phase de création, et sur l'apparence dans la phase de composition.

\item car il vous permet de modifier très rapidement et très précisément l'apparence du document, puisque celle-ci est déterminée par des commandes facilement identifiables.

\item car cette facilité à changer l'apparence, d'autre part, permet de générer facilement plusieurs versions différentes à partir d'un seul contenu : par exemple une version optimisée pour l'impression sur papier, et une autre pour l'affichage sur écran, ajustée à la taille de celui-ci et, peut-être, incluant des hyperliens qui n'ont pas d'utilité dans un document imprimé sur papier.

\item car il est également facile d'éviter les erreurs typographiques courantes dans les traitements de texte comme, par exemple, l'extension de l'italique au-delà du dernier caractère à utiliser, les erreurs d'application de style..

\item car, puisque le fichier de travail ne sera pas distribué et qu'il est  \quotation{pour nos yeux seulement}, il est possible d'incorporer des annotations et des observations, des commentaires et des avertissements pour nous-mêmes, pour des révisions ou des versions futures, avec la tranquillité d'esprit de savoir qu'ils n'apparaîtront pas dans le fichier formaté qui sera distribué.

\item car la qualité que l'on peut obtenir en traitant simultanément l'ensemble du document est bien supérieure à celle que l'on peut obtenir avec un programme qui doit prendre des décisions typographiques à la volée, au fur et à mesure de la rédaction du document.

\item etcétéra.


\stopitemize


Tout cela signifie que, d'une part, lorsque l'on travaille avec \ConTeXt, une fois que l'on a pris le coup de main, on est plus efficace et productif, et que, d'autre part, la qualité typographique que l'on obtiendra est bien supérieure à celle que l'on obtiendrait avec les {\em logiciels de traitement de texte}. Et s'il est vrai que, en comparaison, ces derniers sont plus faciles à utiliser, en réalité ils ne le sont {\em pas beaucoup}. Car s'il est vrai que \ConTeXt se compose, comme je l'ai déjà dit, d'environ 3500 instructions, un utilisateur normal n'a pas à toutes les connaître. Pour faire ce que l'on fait habituellement avec les traitements de texte, il suffira de connaître les instructions qui permettent d'indiquer la structure du document, quelques instructions relatives aux ressources typographiques courantes, comme le gras ou l'italique, et, éventuellement, comment générer une liste, ou une note de bas de page. Au total, pas plus de 15 ou 20 instructions nous permettront de faire presque toutes les choses que l'on fait avec un traitement de texte. Le reste des instructions nous permet de faire différentes choses qui, normalement, sont très difficiles voire impossibles à faire avec un logiciel de traitement de texte. Ainsi, si l'apprentissage de \ConTeXt\ est plus difficile que celui d'un logiciel de traitement de texte, c'est parce que l'on peut faire beaucoup plus de choses avec.

\stopsubsection

\startsubsection 
  [title=Obtenir de l'aide sur \ConTeXt]
\index{aide et ressources}
%\adaptlayout[+2]

Tant que nous sommes des débutants, le meilleur endroit pour trouver de l'aide sur \ConTeXt\ est sans aucun doute son  \goto{wiki}[url(wiki)], qui regorge d'exemples et dispose d'un bon moteur de recherche, même s'il nécessite bien sûr de bien comprendre l'anglais. Nous pouvons aussi chercher de l'aide sur Internet, mais ici le jeu de mots sur lequel repose \ConTeXt\  nous jouera un sale tour car une recherche d'informations sur \quotation{contexte} renverrait des millions de résultats et la plupart d'entre eux n'auraient aucun rapport avec ce que nous recherchons. Pour rechercher des informations sur \ConTeXt, vous devez ajouter quelque chose au nom \quotation{context} ; par exemple, \quotation{tex}, \quotation{luatex}, \quotation{Mark IV} , \quotation{Hans Hagen} (un des créateurs de  \ConTeXt), \quotation{Pragma ADE}, ou quelque chose de similaire (par exemple une autre commande souvent utilisée dans le cas de figure qui vous préoccupe). Il peut également être utile de rechercher des informations par le nom wiki : \quotation{contextgarden}.


Après en avoir appris un peu plus sur \ConTeXt, et si l'on maîtrise bien l'anglais, on peut consulter l'un des nombreux documents inclus dans \suite- ou demander de l'aide~:

\startitemize[packed]
\item soit sur
  \goto{TeX -- LaTeX Stack Exchange} [url(https://tex.stackexchange.com/)]
  et en particulier \goto{les questions tagguées \quotation{\ConTeXt}} [url(https://tex.stackexchange.com/questions/tagged/context)]
  \item soit sur la liste de diffusion propre à \ConTeXt\
\goto{NTG-context}[url(https://mailman.ntg.nl/mailman/listinfo/ntg-context)]
et son \goto{moteur de recherche} [url(https://www.mail-archive.com/ntg-context@ntg.nl/index.html)].
\stopitemize

Cette dernière liste diffusion implique les personnes les plus compétents sur \ConTeXt, mais les règles d'une bonne éducation de \quotation{cybercitoyen} exigent qu'avant de poser une question, on ait essayé par tous les moyens de trouver la réponse par soi-même dans les documentations déjà existantes.

\stopsubsection

\stopsubsection

\stopchapter

\stopcomponent

%%% Local Variables:
%%% mode: ConTeXt
%%% mode: auto-fill
%%% TeX-master: "../introCTX_fra.tex"
%%% coding: utf-8-unix
%%% End:

 

